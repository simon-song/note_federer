\chapter{Borel and Suslin Sets}

Notes on section 2.2 Borel and Suslin Sets of chapter 2 General Measure 
Theory of H. Federer's classics Geometric Measure Theory.


%%%%%%%%%%%%%%%%%%%%%%%%%%%%%%%%%%%%%%%%%%%%%
%%         Federer 2.2.1
%%%%%%%%%%%%%%%%%%%%%%%%%%%%%%%%%%%%%%%%%%%%%
\section{Borel Families}

\begin{definition} \label{D:borel_fam}
A set $F$ is called a \textbf{Borel family} 
\index{Borel family}
\footnote{Federer 2.2.1, also called $\sigma$-algebra}
with respect to a set $X$ iff
$\emptyset\in F\subset 2^X$ and the following three conditions hold:
\begin{enumerate}
  \item[(1)] If $A\in F$, then $X\setminus A\in F$.
  \item[(2)] If $G\subset F$ and $G$ is countable, then $\cup G\in F$.
  \item[(3)] If $\emptyset\neq G\subset F$ and $G$ is countable, then $\cap G\in F$.
\end{enumerate}
\end{definition}
Clearly these conditions are redundant, because (1) implies (2) and (3) are
equivalent. And $2^X$ and $\{\emptyset,X\}$ are the largest and smallest Borel
families with respect to $X$. The intersection of any set of Borel families is a
Borel family. For every $S\subset 2^X$ there exists a smallest Borel family
containing $S$ called the Borel family generated by $S$.


%%%%%%%%%%%%%%%%%%%%%%%%%%%%%%%%%%%
\begin{proposition} \label{P:borel1}
If $S\subset 2^X$, then $S$ is contained in a smallest family $F$ satisfying (2)
and (3) (i.e. closed under countable union and countable intersection). In case
that $A\in S$ implies $X\setminus A\in F$, then $F$ equals the Borel family
generated by $S$.
\end{proposition}
\begin{proof}
Let $H=\{A: A\in F,\, X\setminus A\in F\}$, then $H\subset F$. Now we verify 
that $H$ is a Borel family. Condition (1) is trivial. To verify (2), note that 
if $A_1, A_2, \cdots\in H$, then $A_1,A_2,\cdots\in F$ and 
$X\setminus A_1,X\setminus A_2,\cdots\in F$, thus
\[
  \bigcup_{i=1}^{\infty}A_i \in F,
\]
and 
\[
  X\setminus \bigcup_{i=1}^{\infty} A_i 
    = \bigcap_{i=1}^{\infty} X\setminus A_i \in F.
\]
Similarly we can verify conditon (3). Since $H$ is a Borel family containing
$S$, we have $F\subset H$. Hence $F=H$.
\end{proof}


%%%%%%%%%%%%%%%%%%%%%%%%%%%%%%%%%%%
%%     topology
%%%%%%%%%%%%%%%%%%%%%%%%%%%%%%%%%%%
\begin{definition} \label{D:topo}
A \textbf{topological space} 
\index{topological space} 
\footnote{Engelking, General Topology, 1.1, p.11}
is a pair $(X,O)$ of a set $X$ and a family $O$ of
subsets of $X$ satisfying the following conditions:
\begin{enumerate}
  \item[(O1)] $\emptyset\in O$, and $X\in O$.
  \item[(O2)] If $U_1\in O$ and $U_2\in O$, then $U_1\cap U_2\in O$.
  \item[(O3)] If $A\subset O$ then $\cup A\in O$.
\end{enumerate}
The set $X$ is called a space, the elements of $X$ are called points of the
space, and the subsets of $X$ belonging to $O$ are called open in the space, and
the family $O$ of open subsets of $X$ is called a topology in $X$. A subset
$A\subset X$ is called closed iff $X\setminus A$ is open.
\end{definition}


%%%%%%%%%%%%%%%%%%%%%%%%%%%%%%%%%%%
%%  base
%%%%%%%%%%%%%%%%%%%%%%%%%%%%%%%%%%%
\begin{definition} 
\footnote{Engelking, General Topology, 1.1, p.12}
A family $B\subset O$ is called a \textbf{base for a topological space $(X,O)$}
\index{base for a topological space}
if every non-empty open subset of X can be represented as the union of a
subfamily of $B$. A family $B(x)$ of neighbourhoods of $x$ is called a 
\textbf{base for a topological space $(X,O)$ at the point $x$}
if for any neighbourhood $V$ of $x$ there exists a $U\in B(x)$ such that 
$x\in U\subset V$. Suppose that for every $x\in X$ a base $B(x)$ for $(X,O)$ at 
$x$ is given, the collection $\{B(x)\}_{x\in X}$ is called a 
\textbf{neighbourhood systems for the topological space $(X,O)$}. 
\index{neighbourhood systems} 
A family $\mathcal{P}\subset\mathcal{O}$ is called a \textbf{subbase for a
topological space} $(X,\mathcal{O})$ 
\index{subbase for a topological space}
if the family of all finite intersections $U_1\cap U_2\cap \dots\cap U_k$, where
$U_i\in\mathcal{P}$ for $i=1,2,\dots,k$ is a base for $(X,\mathcal{O})$.
\end{definition}


%%%%%%%%%%%%%%%%%%%%%%%%%%%%%%%%%%%
%% properties of neighborhood system 
%%%%%%%%%%%%%%%%%%%%%%%%%%%%%%%%%%%
\begin{proposition} \label{P:neigh_sys}
\footnote{Engelking, General Topology, 1.1, p.13}
Any neighborhood system $\{B(x)\}_{x\in X}$ of topological space $(X,O)$ has the
following properties:
\begin{enumerate}
  \item[(BP1)] For every $x\in X$, $B(x)\neq \emptyset$, and for every 
    $U\in B(x)$, $x\in U$.
  \item[(BP2)] If $x\in U\in B(y)$, then there exists a $V\in B(x)$ such that
    $V\subset U$.
  \item[(BP3)] For any $U_1,U_2\in B(x)$ there exists a $U\in B(x)$ such that
    $U\subset U_1\cap U_2$.
\end{enumerate}
\end{proposition}
\begin{proof}
Property $(BP1)$ follows directly from the definition of a base at $x$. Property
$(BP2)$ and $(BP3)$ also follows from the definition of a base at $x$, because
$U\in B(y)$ and $U_1\cap U_2$ are open sets containing $x$.
\end{proof}


%%%%%%%%%%%%%%%%%%%%%%%%%%%%%%%%%%%
%%   generating topology 
%%%%%%%%%%%%%%%%%%%%%%%%%%%%%%%%%%%
Let $X$ be an arbitrary set, by generating a topology on $X$ we mean selecting a
family $O$ of subsets of $X$ such that satisfies conditions $(O1)-(O3)$, i.e., a
family $O$ such that the pair $(X,O)$ is a topological space. Often it is more
convenient not to describe the family $O$ of open sets directly.

\begin{proposition}[Topology generated by a neighborhood system] 
\label{P:top_gen}
\footnote{Engelking, General Topology, 1.2.3, p. 21}
Suppose we are given a set $X$ and a collection $\{B(x)\}_{x\in X}$ of famililes
of subsets of $X$ which has properties $(BP1)-(BP3)$. Let $O$ be the family of
all subsets of $X$ that are unions of subfamilies of $\bigcup_{x\in X}B(x)$. 
  \footnote{More clearly, $U\in O$ if $U=\bigcup_{x\in X} D(x)$ where 
    $D(x)\subset B(x)$.}
The family $O$ satisfies conditions $(O1)-(O3)$ and the collection 
$\{B(x)\}_{x\in X}$ is a neighborhood system for the topological space $(X,O)$.

The topology $O$ is called the topology generated by the neighborhood system 
$\{B(x)\}_{x\in X}$.
\index{topology generated by a neighborhood system}
\end{proposition}
\begin{proof}
To prove that condition $(O1)$ is satisfied, first notice that directly from 
definition of $O$ we have $\emptyset \in O$. And from property
$BP(1)$ for any $x$, $B(x)\neq \emptyset$, hence $X=\bigcap_{x\in X} U_x$ where
$U_x\in B(x)$ and $U_x\neq \emptyset$, thus $X\in O$.

Take $U_1,U_2\in O$; we then have 
$U_1=\bigcup_{x\in X} \bigcup_{s\in S} U_s(x)$ and
$U_2=\bigcup_{y\in X} \bigcup_{t\in T} U_t(y)$, where $U_s(x)\in B(x)$ and 
$U_t(y)\in B(y)$ for any $s\in S$ and $t\in T$. Hence we have
\[
  U_1\bigcap U2 = \bigcup_{x,y\in X, s\in S, t\in T} 
                  \left( U_s(x)\bigcap U_t(y) \right).
\]
To show that condition $(O2)$ is satisfied, it is enough to prove that 
$U_s(x)\bigcap U_t(y)$ is the union of subfamilies of $\bigcup_{x\in X}B(x)$. 
From properties $BP(2)$ and $BP(3)$, for every $z\in U_s(x)\cap U_t(y)$ there
exists a $U(z)\in B(z)$ such that $U(z)\subset U_s(x)\cap U_t(y)$, hence
\[
  U_s(x)\bigcap U_t(y)=\bigcup_{z\in U_s(x)\cap U_t(y)} U(z).
\]

Condition $(O3)$ is satisfied by the definition of the family $O$.
\end{proof}


%%%%%%%%%%%%%%%%%%%%%%%%%%%%%%%%%%%
%%  continuous function in topology
%%%%%%%%%%%%%%%%%%%%%%%%%%%%%%%%%%%
\begin{definition} \label{D:cont}
Let $(X,O)$ and $(Y,O')$ be two topological spaces, a mapping $f:X\to Y$ is
called \textbf{continuous} 
\index{continuous function between topological spaces} 
\footnote{Engelking, General Topology, 1.4, p.27}
if $f^{-1}(U)\in O$ for any $U\in O'$, i.e. the
inverse image of any open subset of $Y$ is open in $X$.
\end{definition}


%%%%%%%%%%%%%%%%%%%%%%%%%%%%%%%%%%%
%%  property of continuous function 
%%%%%%%%%%%%%%%%%%%%%%%%%%%%%%%%%%%
\begin{proposition} \label{P:cont}
For a mapping of a topological space $(X,O)$ to a topological space $(Y,O')$ the
following conditions are equivalent:
  \footnote{Engelking, General Topology, Proposition 1.4.1, p.28}
\begin{enumerate}
  \item[(i)] The mapping $f$ is continuous.
  \item[(iii)] There are neighbourhood systems $\{B(x)\}_{x\in X}$ and 
     $\{D(y)\}_{y\in Y}$ for $X$ and $Y$ respectively, such that for every
     $x\in X$ and a $V\in D(f(x))$ there exists a $U\in B(x)$ satisfying
     $f(U)\subset V$.
\end{enumerate}
\end{proposition} 

\begin{proof}
First we prove $(i)\Rightarrow (iii)$. Since $f$ is continuous, for every 
$x\in X$ and for every $V\in D(f(x))$ there exists a $U'\in O$ such that 
$U'=f^{-1}(V)$. It is obvious that $x\in U'$, thus there exists a $U\in B(x)$
such that $U\subset U'$. And it is obvious $f(U)\subset V$.

Next we prove $(iii)\Rightarrow (i)$. For every $V\in O'$ and every 
$x\in f^{-1}(V)$, we have $f(x)\in V$, thus from the definition of neighborhood
system there exists $V(f(x))\in D(f(x))$ such that $V(f(x))\subset V$. 
From $(iii)$ there exists a $U_x\in B(x)$ such that $f(U_x)\subset V(f(x))$.
Hence $U_x\subset f^{-1}(V(f(x))\subset f^{-1}(V)$, and
\[
  f^{-1}(V)=\bigcup_{x\in f^{-1}(V)} U_x,
\]
and $f^{-1}(V)\in O$ follows because topology $O$ is closed under arbitrary 
union.
\end{proof}


    
    


%%%%%%%%%%%%%%%%%%%%%%%%%%%%%%%%%%%
%%    metric 
%%%%%%%%%%%%%%%%%%%%%%%%%%%%%%%%%%%
\begin{definition} \label{D:metric}
A \textbf{metric space}
\index{metric space}
\footnote{Engelking, General Topology, 4.1, p.248}
is a pair $(X,\rho)$ consisting of a set $X$ and a function $\rho$
definied on set $X\times X$, assuming non-negative real values and satisfying
the following conditions:
\begin{enumerate}
  \item[(M1)] $\rho(x,y)=0$ iff $x=y$.
  \item[(M2)] $\rho(x,y)=\rho(y,x)$ for all $x,y\in X$.
  \item[(M3)] $\rho(x,y)+\rho(y,z)\ge \rho(x,z)$ for all $x,y,z\in X$.
\end{enumerate}
The set $X$ is called a space, the elments of $X$ are called points, the
function $\rho$ is called a metric and the number $\rho(x,y)$ is called the distance
between $x$ and $y$.
\end{definition}


%%%%%%%%%%%%%%%%%%%%%%%%%%%%%%%
%%  open balls
%%%%%%%%%%%%%%%%%%%%%%%%%%%%%%%
\begin{definition}
\footnote{Engelking, General Topology, 4.1, p.248}
Let $(X,\rho)$ be a metric space, $x_0$ be a point of $X$ and $r$ be a positive
number, the set $B(x_0,r)=\{x\in X: \rho(x_0,x)<r\}$ is called the 
\textbf{open ball with center $x_0$ and radius $r$} 
\index{open ball}
or, briefly, the $r$-ball
about $x_0$. for a set $A\subset X$ and a positive number $r$, by the $r$-ball
about $A$ we mean the set $B(A,r)=\{\bigcup_{x\in A} B(x,r)\}$.
\end{definition}


%%%%%%%%%%%%%%%%%%%%%%%%%%%%%%%
%%  topology induced by a metric
%%%%%%%%%%%%%%%%%%%%%%%%%%%%%%%
\begin{lemma} \label{L:top_metric}
\footnote{Engelking, General Topology, 4.1, pp.248-249}
Let $(X,\rho)$ be a metric space, 
for every $x\in X$ we define $\mathcal{B}(x)=\{B(x,r):r>0\}$, i.e. the
set of open balls at point $x$, then the collection of open sets at each point
$\{\mathcal{B}(x)\}_{x\in X}$ satisfies properties (BP1)-(BP3) in 
\ref{P:neigh_sys}. 
Thus, by Proposition \ref{P:top_gen} we generate a topology
$\mathcal{O}$ on $X$. We call this topology $\mathcal{O}$ on the set $X$ the
\textbf{topology induced by the metric $\rho$}.
\index{topology induced by metric}
\end{lemma}
\begin{proof}
It is trivial to prove that $\{\mathcal{B}(x)\}_{x\in X}$ satisfies properties
(BP1) and (BP3). To prove that it satisfies property (BP2), suppose 
$x_1\in B(x_0,r)$ where $r>0$, let $r_1=r-\rho(x_1,x_0)$, and we have $r_1>0$ 
from the definition of open ball. And the open ball $B(x_1,r_1)\subset B(x_0,r)$
because by virtue of property (M3) of the metric $\rho$, for any 
$x\in B(x_1,r_1)$ we have
\[
  \rho(x,x_0) \le \rho(x,x_1)+\rho(x_1,x_0) < r_1+\rho(x_1,x_0) = r.
\]
\end{proof}


%%%%%%%%%%%%%%%%%%%%%%%%%%%%%%%%%%%
\begin{proposition}
Let $(X,\rho)$ be a metric space, any open set $O\subset X$ (in the topology 
induced by metric $\rho$) is a union of open balls, and for any point $x\in O$ 
there exists an open ball $B(x,r)$ such that $B(x,r)\subset O$.
\end{proposition}
\begin{proof}
The first part is a direct consequence of Lemma \ref{L:top_metric} since the
collection of open balls is a base of the induced topology. 
To prove the second part we only need
to verify the property holds for any open balls. For any open ball $B(y,R)$ in
$X$ and any point $x\in B(y,R)$, open ball $B(x,R-\rho(x,y))$ is apparently a
subset of $B(y,R)$ because for any $z\in B(x,R-\rho(x,y))$ we have 
$\rho(z,x)<R-\rho(x,y)$ and thus
\[
  \rho(z,y) \leq \rho(z,x)+\rho(x,y) < R.
\]
\end{proof}


%%%%%%%%%%%%%%%%%%%%%%%%%%%%%%%%%%%
%%    continuous function: metric 
%%%%%%%%%%%%%%%%%%%%%%%%%%%%%%%%%%%
\begin{proposition} \label{P:cont2}
A mapping $f$ of a space $X$ with the topology induced by a metric $\rho$ to a
space $Y$ with the topology induced by a metric $\sigma$ is continuous iff for
every $x\in X$ and any $\epsilon>0$ there exists a $\delta>0$ such that 
$\sigma(f(x),f(x'))<\epsilon$ whenever $\rho(x,x')<\delta$.
\footnote{Engelking, General Topology, 4.1.8, p.253}
\end{proposition}
\begin{proof}
Directly from the equivalence of (i) and (iii) in Proposition \ref{P:cont}.
\end{proof}


%%%%%%%%%%%%%%%%%%%%%%%
%% distance to a set
%%%%%%%%%%%%%%%%%%%%%%%
\begin{definition}
The distance $\rho(x,A)$ from a point $x$ to a set $A$ in a metric space
$(X,\rho)$ is defined by
\[
  \rho(x,A)=\inf\{\rho(x,a): a\in A\}, 
\]
if $A\neq \emptyset$, and $\rho(x,\emptyset)=1$.
\footnote{Engelking, General Topology, p.253}
\end{definition}


%%%%%%%%%%%%%%%%%%%%%%%%%%%%%%%%%%%%
\begin{proposition} \label{P:dist}
For a pair of points $x,y$ and a set $A$ in a metric space $(X,\rho)$ we have
\footnote{Engelking, General Topology, 4.1.9, p.254}
\[
  |\rho(x,A)-\rho(y,A)|\le \rho(x,y).
\]
\end{proposition}
\begin{proof}
From defintions we have
\begin{align*}
  \rho(x,y) + \rho(y,A) 
    &= \rho(x,y) + \inf\{\rho(y,a): a\in A\}  \notag \\
    &= \inf\{\rho(y,a)+\rho(x,y): a\in A\}  \notag \\
    &\ge \inf\{\rho(x,a): a\in A\}  \notag \\
    &= \rho(x,A) \notag \\
\end{align*}
And by symmetry we also have $\rho(x,y) + \rho(x,A)\ge\rho(y,A)$. 
This completes the proof.
\end{proof}

%%%%%%%%%%%%%%%%%%%
\begin{theorem} \label{T:dist_cont}
For every set $A$ in a metric space $(X,\rho)$ assigning to each point $x\in X$
the distance $\rho(x,A)$ defines a continous function on $X$.
\footnote{Engelking, General Topology, 4.1.10, p.254}
\end{theorem}
\begin{proof}
Directly from Propositions \ref{P:dist} and \ref{P:cont2}.
\end{proof}


%%%%%%%%%%%%%%%%%%%%%%%%%%%%%%%%%%%%%
%%  closure in a topological space
%%%%%%%%%%%%%%%%%%%%%%%%%%%%%%%%%%%%%
\begin{definition}
\footnote{Engelking, General Topology, 1.1, p.13}
For every set $A$ in a topological space $(X,\mathcal{O})$ we define the 
\textbf{closure} 
\index{closure} 
$\overline{A}$ of $A$ as the smallest closed set containing $A$. 
Hence a set is closed iff it is equal to its closure.
\end{definition}

%%%%%%%%%%%%%%%%%%%%%%%%%%%%%%%%%%%%%
\begin{proposition} \label{P:clos}
\footnote{Engelking, General Topology, 1.1.1, p.13}
For every set $A\subset X$ in a topological space $(X,\mathcal{O})$ the
following conditions are equivalent:
\begin{enumerate}
  \item[(i)] The point $x$ belongs to $\overline{A}$.
  \item[(ii)] For every neighborhood $U$ of $x$ we have $U\cap A\neq\emptyset$.
\end{enumerate}
\end{proposition}
\begin{proof}
First we prove $(ii)\Rightarrow (i)$.
\footnote{Actually we prove that if (i) is not true then (ii) is not true.}
Assume $x\notin\overline{A}$, then $x\in X\setminus \overline{A}$. Since $\overline{A}$ is a 
closed set, $X\setminus \overline{A}$ is an open set and 
$X\setminus \overline{A} \cap A=\emptyset$.

Next we prove $(i)\Rightarrow (ii)$. Assume that there exists a neighborhood $U$
of $x$ such that $U\cap A=\emptyset$, then $x\notin\overline{A}$ otherwise
$\overline{A}\setminus U$ is a closed set and 
$A\subset \overline{A}\setminus U \subset\overline{A}$ which is in conflict with the fact
that $\overline{A}$ is the smalleset closed set containing $A$.
\end{proof}


%%%%%%%%%%%%%%%%%%%%%%%%%%%%%%%%%%%%%%%%%%%
%%  convergent sequence in a metric space
%%%%%%%%%%%%%%%%%%%%%%%%%%%%%%%%%%%%%%%%%%%
\begin{definition}
A sequence $x_1,x_2,\cdots$ of points of a metric space $(X,\rho)$
\textbf{converges}
\index{convergence}
\footnote{Engelking, General Topology, 4.1, pp.249-250}
to a point $x\in X$ if the sequence of real numbers
$\rho(x,x_1),\rho(x,x_2),\cdots$ converges to zero; the point $x$ is called the
limit of the sequence $x_1,x_2,\cdots$ and is denoted by $\lim{x_i}$.
\end{definition}

\begin{proposition} \label{P:conv}
A point $x$ belongs to the closure $\overline{A}$ of a set $A\subset X$ with respect
to the topology induced on $X$ by a metrc $\rho$ iff there exists a sequence of
points of $A$ that converges to $x$.
\footnote{Engelking, General Topology, 4.1.1, p.250}
\end{proposition}
\begin{proof}
Assume that $x\in \overline{A}$. For every natural number $i$ take a point 
$x_i\in A\cap B(x,1/i)$; clearly we have $\rho(x,x_i)<1/i$ and $x=\lim{x_i}$.
On the other hand, if $x\notin \overline{A}$, then there exists an $r>0$ such that
$A\cap B(x,r)=\emptyset$; hence we have $\rho(x,a)\ge r$ for every $a\in A$ and
there is no sequence of points of $A$ that converges to to $x$.
\end{proof}




%%%%%%%%%%%%%%%%%%%%%%%%%%%%%
%%  closure in a metric space
%%%%%%%%%%%%%%%%%%%%%%%%%%%%%
\begin{corollary} \label{C:closure_m}
For every set $A$ in a metric space $(X,\rho)$ we have
\footnote{Engelking, General Topology, 4.1.11, p.254}
\[
  \overline{A} = \{ x: \rho(x,A)=0  \}.
\]
\end{corollary}
\begin{proof}
Assume that $x\in\overline{A}$, then for any $r>0$, $B(x,r)\cap A\neq\emptyset$, i.e.
there exists an $a\in A$ such that $\rho(x,a)<r$, thus $\rho(x,A)=0$. On the
other hand, if $x\notin \overline{A}$, then there exists a $r>0$ such that
$B(x,r)\cap A=\emptyset$, i.e., for all $a\in A$ we have $\rho(x,a)\ge r$ and
$\rho(x,A)\ge r>0$.
\end{proof}
\begin{proof}[Second Proof]
From Proposition \ref{P:conv}, $x\in \overline{A}$ iff there exists a sequence
$x_1,x_2,\cdots$ of points of $A$ that converges to $x$, i.e. $x=\lim{x_i}$.
From Theorem \ref{T:dist_cont}, let function $f:X\to R$ be 
$f(x)=\rho(x,A)$ for every $x\in X$, then this function is continous,
and thus we can exchange the function and the limit and get
\[
  \rho(x,A)=\rho(\lim{x_i},A)=\lim\rho(x_i,A)=0.
\]
\end{proof}


%%%%%%%%%%%%%%%%%%%%%%%%%%%%%%%%%%%%%%%%%%%%%%%
\begin{corollary} \label{C:g_delta}
Every closed subset of a metrizable space is a \textbf{$G_{\delta}$ set}, 
\index{$G_{\delta}$ set}
i.e. countable intersection of open sets.
Every open subset of a metrizable space is a \textbf{$F_{\sigma}$ set}, 
\index{$F_{\sigma}$ set}
i.e. countable union of closed sets.
\footnote{Engelking, General Topology, 4.1.12, p.254}
\end{corollary}
\begin{proof}
We only need to verify that any closed subset $A$ (=$\overline{A}$) is a countable
intersection of open sets. The second part follows directly after applying de 
Morgan's law. 

From Corollary \ref{C:closure_m}, and let $f(x)=\rho(x,A)$, we have 
\[
  A=f^{-1}(0)
   =f^{-1}\left( \bigcap_{i} \left( -\frac{1}{i}, \frac{1}{i} \right) \right)
   =\bigcap_{i} f^{-1}\left( \left( -\frac{1}{i}, \frac{1}{i} \right) \right).
\]
And from Theorem \ref{T:dist_cont}, 
$f^{-1}\left( \left( -\frac{1}{i}, \frac{1}{i} \right) \right)$ is an open set
in $X$ for any $i$.
\end{proof}





%%%%%%%%%%%%%%%%%%
\begin{definition}
Given a topological space $(X,T)$, i.e. a space $X$ and a topology $T$, the
members of the Borel family generated by topology $T$ (i.e. open sets) are
\textbf{Borel sets}
\index{Borel sets}
associated with $T$.
\footnote{Federer, 2.2.1, p.60}
\end{definition}


%%%%%%%%%%%%%%%%%%
\begin{proposition} \label{P:borel}
Given a topological space $(X,T)$, i.e. a space $X$ and a topology $T$, if every
open set is the union of a countable family of closed sets, for instance in case
$T$ is induced by a metric, 
\footnote{See Corollary \ref{C:g_delta}.}
then the Borel sets form the smallest class
satisfying (2) and (3) in Definition \ref{D:borel_fam}, and containing the class
of all closed sets; they also form the smalleset class satisfying (2)' and (3)
and containing $T$ (all open sets).
\end{proposition}
\begin{proof}
Let $F$ be the smallest familiy containing $T$ that satisfies (2) and (3) of
Defintion \ref{D:borel_fam}. For any open set $A\in T$ there are countably many
closed set $C_1,C_2,\cdots$ such that $A=\cup_{i=1}^{\infty} C_i$. For any $i$,
$X\setminus C_i$ is an open set in $T$ thus $X\setminus C_i\in F$ and
\[
  X\setminus A = X\setminus \left( \bigcup_{i=1}^{\infty} C_i \right)
               = \bigcap_{i=1}^{\infty} \left( X\setminus C_i \right) \in F.
\]
Hence from Proposition \ref{P:borel1}, $F$ is a Borel family, i.e. the Borel
sets. And from property (1) of Borel family, $F$ contains all closed sets.
\end{proof}


%%%%%%%%%%%%%%%%%%%%%%%%%%%%%%%%%%%%%%%%%%%%%%%%%%%%%%%%%%%%%%%%%%%%%%%%%%%
\section{Approximation by Closed Sets}

%%%%%%%%%%%%%%%%%%
\begin{theorem} \label{T:Fed2.2.2}
\footnote{Federer, 2.2.2, p.60}
Suppose $\phi$ measures a metric space $X$, all open sets are $\phi$-measurable,
and $B$ is a Borel set.
\begin{enumerate}
  \item[(1)] If $\phi(B)<\infty$ and $\epsilon>0$, then $B$ contains a closed
  set $C$ for which
  \[
    \phi(B\setminus C)<\epsilon.
  \]
  \item[(2)] If $B$ is contained in the union of countably many open sets $V_i$
  with $\phi(V_i)<\infty$, and if $\epsilon>0$, then $B$ is contained in an open
  set $W$ for which 
  \[
    \phi(W\setminus B)<\epsilon.
  \]
\end{enumerate}
\end{theorem}
\begin{proof} %%%%%
The approach we adopt here is to prove that the class of all sets with the
desired property is closed under countable union and countable intersection and 
contains all closed sets, thus contains all Borel sets.

Let $\psi=\phi|_B$ 
\footnote{
  It is easy to see that for any $A$, $\psi(A)=\phi|_B(A)=\phi(B\cap A)<\infty$.
  This is important for us to apply substraction safely and to guarantee that 
  all closed sets are in class $F$ below.
}
and consider the class $F$ of all subsets $A$ of $X$ with the
property: for every $\epsilon>0$, $A$ contains a closed set $C$ such that 
\[
  \psi(A\setminus C)<\epsilon.
\]

We verify that if $A_1,A_2,A_3,\cdots\in F$, then also 
$\cap_{i=1}^{\infty} A_i \in F$ and $\cup_{i=1}^{\infty} A_i \in F$.
Given $\epsilon>0$, we choose closed sets $C_i\subset A_i$ with 
$\psi(A_i\setminus C_i)<\epsilon 2^{-i}$, and estimate that
\[
  \psi \left( \bigcap_{i=1}^{\infty} A_i\setminus 
       \bigcap_{i=1}^{\infty} C_i \right)
    \leq \psi \left( \bigcup_{i=1}^{\infty} ( A_i\setminus C_i ) \right)
    < \sum_{i=1}^{\infty} \epsilon 2^{-i} = \epsilon,
\]
where we have used the following property
\[
  \bigcup_i ( A_i\setminus C_i )
    \supset \bigcup_i \left( A_i\setminus \bigcup_j C_j \right) 
    = \bigcup_i A_i\setminus \bigcup_j C_j,
\]

It is a little more difficult to verifty that $\cup_{i=1}^{\infty} A_i \in F$.
First notice that because all open sets are $\phi$-measurable, they are also 
$\psi$-measurable. Thus from Theorem \ref{T:meas_prop}, property (1) and (3), 
all closed sets are $\psi$-measurable and for any closed sets 
$C_1,C_2,C_3,\cdots$ we have
\[
  \psi \left( \bigcup_{i=1}^{\infty} C_i \right)
    = \lim_{n\to\infty} \psi\left( \bigcup_{i=1}^n C_i \right),
\]
Using the definition of measurable set we have
\begin{align*}
  \psi \left( \bigcup_{i=1}^{\infty} A_i\setminus \bigcup_{i=1}^{n} C_i \right)
    &= \psi \left( \bigcup_{i=1}^{\infty} A_i \right)
      - \psi \left( \bigcup_{i=1}^{n} C_i \right) \notag\\
    &\leq 
      \psi \left( \bigcup_{i=1}^{\infty} A_i
                  \setminus \bigcup_{i=1}^{\infty} C_i \right)
      + \psi \left( \bigcup_{i=1}^{\infty} C_i \right)
      - \psi \left( \bigcup_{i=1}^{n} C_i \right),
\end{align*}
hence 
\footnote{Federer skips the proof of this step.}
\[
  \lim_{n\to\infty} \psi \left( \bigcup_{i=1}^{\infty} A_i\setminus 
       \bigcup_{i=1}^{n} C_i \right)
  \leq \psi \left( \bigcup_{i=1}^{\infty} A_i\setminus 
       \bigcup_{i=1}^{\infty} C_i \right).
\]
Thus we have
\[
  \lim_{n\to\infty} \psi \left( \bigcup_{i=1}^{\infty} A_i\setminus 
       \bigcup_{i=1}^{n} C_i \right)
  \leq \psi \left( \bigcup_{i=1}^{\infty} A_i\setminus 
       \bigcup_{i=1}^{\infty} C_i \right)
    \leq \psi \left( \bigcup_{i=1}^{\infty} ( A_i\setminus C_i ) \right)
    < \sum_{i=1}^{\infty} \epsilon 2^{-i} = \epsilon,
\]
where we have used the following property
\[
  \bigcup_i ( A_i\setminus C_i )
    \supset \bigcup_i \left( \bigcap_j A_j\setminus C_i \right) 
    = \bigcap_j A_j\setminus \bigcap_i C_i.
\]
Moreover $\bigcap_{i=1}^{\infty}C_i$ and $\bigcup_{i=1}^{\infty}C_i$ are closed.

Thus $F$ satisfies the conditions (2) and (3) of Definition \ref{D:borel_fam}.
Since $F$ contains the class of all closed sets, from Proposition \ref{P:borel}
we conclude that it contains the class of all Borel sets. Hence $B\in F$, and
for any $\epsilon>0$ there exists a closed set $C$ such that $C\subset B$ and
$\psi(B\setminus C)<\epsilon$. Thus
\[
  \phi(B\setminus C) = \phi(B\cap (B\setminus C)) = \phi|_B(B\setminus C)
  = \psi(B\setminus C) < \epsilon.
\]

To prove (2), we choose closed sets $C_i\subset V_i\setminus B$ such that
\[
  \phi( (V_i\setminus B)\setminus C) < \epsilon 2^{-i},
\]
Since $V_i\cap B\subset V_i\setminus C_i$, we thus have
\[
  \bigcup_i (V_i\setminus C_i) \supset \bigcup_i (V_i\cap B) = B, 
\]
and 
\[
  \phi \left( \left( \bigcup_i (V_i\setminus C_i) \right)
              \setminus B \right)
  \leq \sum_i \phi( (V_i\setminus B)\setminus C) < \epsilon. 
\]
Thus $W=\cup_i (V_i\setminus C_i)$ is the desired open set in (2).
\end{proof}   %%%%%


%%%%%%%%%%%%%%%%%%%%%%%%
%%  Borel regular
%%%%%%%%%%%%%%%%%%%%%%%%
\begin{definition}
A measure $\phi$ over a topological space $X$ is called 
\textbf{Borel regular}
\index{Borel regular measure}
\footnote{Federer 2.2.3, p.61}
iff all open sets are $\phi$-measurable and each subset $A$ of $X$ is contained 
in a Borel set $B$ for which $\phi(A)=\phi(B)$.
\end{definition}

%%%%%%%%%%%%%%%%%%%%%%%%
\begin{proposition} \label{P:borel_reg}
\begin{enumerate}
\item[(1)] If $\phi$ is a Borel regular measure and $A$ is a $\phi$-measurable 
  set for which $\phi(A)<\infty$, then there exists Borel sets $B$ and $D$ such 
  that
  \[
    D\subset A\subset B, \qquad \phi(B\setminus D)=0.
  \]
\item[(2)] If $\phi$ is a Borel regular measure over a metric space, the
  statements (1) and (2) of Theorem \ref{T:Fed2.2.2} hold for every
  $\phi$-measurable set $B$.
  \footnote{Note that all open sets are $\phi$-measurable from the definition of
      the Borel regual measure.}
\item[(3)] If $\phi$ is a Borel regular measure and $A$ is a Borel set, then
  $\phi|_A$ is Borel regular.
\item[(4)] If $\phi$ is a measure over a topological space $X$ such that all
  Borel subsets of $X$ are $\phi$-measurable, and if 
  \[
    \psi(A) = \inf\{\phi(B): A\subset B \, \text{and $B$ is a Borel set} \}
  \]
  whenever $A\subset X$ , then $\psi$ is a Borel regular measure, and
  $\psi(A)=\phi(A)$ in case $A$ is a Borel set.
\end{enumerate}
\end{proposition}
\begin{proof}  %%%%%%%%%%%%%%%%%%%%%%%%%%%%
\begin{enumerate}
\item[(1)] Since $\phi$ is Borel regular, there exists a Borel set $B$ such that
$A\subset B$ and $\phi(B)=\phi(A)$. Since $A$ is $\phi$-measurable 
we have
\[
  \phi(A)=\phi(B)=\phi(A)+\phi(B\setminus A),
\]
and because $\phi(A)<\infty$ we have $\phi(B\setminus A)=0$.
Furthermore, there exists a Borel set $E$
such that $B\setminus A \subset E$ and $\phi(B\setminus A)=\phi(E)=0$.
Now let $D=B\setminus E$, it is easy to see that $D$ is Borel, $D\subset A$, 
and 
\[
  \phi(D)=\phi(B\setminus E)=\phi(B)=\phi(A).
\]

\item[(2)] This is a direct consequence of (1) and Theorem \ref{T:Fed2.2.2} by 
selecting Borel sets $D$ and $E$ such that $D\subset B\subset E$ and
$\phi(D)=\phi(B)=\phi(E)$.

\item[(3)] From Lemma \ref{L:meas_proj}, $\phi|_A$ is a measure, and all open 
sets are $\phi|_A$-measurable. Since $\phi$ is Borel regular, for any 
$B\subset X$, there exists a Borel set $D\supset B$ such that $\phi(D)=\phi(B)$.
Hence $\phi(D\setminus B)=0$ and
\[
  \phi|_A(D) = \phi(A\cap D) \leq \phi(A\cap B) + \phi(D\setminus B) 
    = \phi(A\cap B) = \phi|_A(B),
\]
thus $\phi|_A(D)=\phi|_A(B)$, and $\phi|_A$ is Borel regular.

\item[(4)] The proof is similar to Lemma \ref{L:regul}.

\end{enumerate}
\end{proof}   %%%%%


%%%%%%%%%%%%%%%%%%%%%%%%%%%%%%%%%%%%%%%%%%%%%
%%         Federer 2.2.4
%%%%%%%%%%%%%%%%%%%%%%%%%%%%%%%%%%%%%%%%%%%%%
\section{Nonmeasurable Sets}

%%%%%%%%%%%%%%%%%%%%
\begin{definition} \label{D:dense}
\footnote{Engelking, General Topology, 1.3, p. 25}
Given a topological space $X$, a subset $A\subset X$ is called \textbf{dense}
\index{dense set}
if $\overline{A}=X$, i.e. if the closure of $A$ is $X$. 
A subset $A\subset X$ is called \textbf{nowhere dense} 
\index{nowhere dense}
if $\overline{X\setminus \overline{A}}=X$.
The \textbf{density of space} $X$ is defined to be the
smallest cardinal number of its dense subsets. The space $X$ is called
\textbf{separable} 
\index{separable space} 
if its density is countable.
\end{definition}

%%%%%%%%%%%%%%%%%%%%
\begin{definition}
\footnote{Engelking, General Topology, 4.3, p. 268}
Let $(X,\rho)$ be a metric space and $\{x_i\}$ a sequence of points of $X$, we
say that $\{x_i\}$ is a \textbf{Cauchy sequence} 
\index{Cauchy sequence}
if for every $\epsilon>0$ there
exists a natural number $k$ such that $\rho(x_i,x_k)\leq\epsilon$ whenever 
$i\geq k$. A metric space $(X,\rho)$ is \textbf{complete} 
\index{complete space}
if every Cauchy 
sequence in $(X,\rho)$ converges to a point of $X$.
\end{definition}

%%%%%%%%%%%%%%%%%%%%
\begin{theorem}
\footnote{Federer 2.2.4, p.62} 
If $\phi$ is a Borel regular measure over a complete, separable metric space
$X$, $0<\phi(A)<\infty$, and $\phi(\{x\})=0$ whenever $x\in A$, then $A$ has a
$\phi$ nonmeasurable subset.
\end{theorem}
\begin{proof}   %%%%%
\footnote{Why the metric space has to be complete and separable? It does not
    seem to be used in the proof.}
We consider the class $\Gamma$ of all closed subsets $C$ of $A$ with positive
$\phi$-measure, i.e., $\phi(C)>0$, hence $|C|=2^{\aleph_0}$, i.e. $C$ is 
uncountable. This is because all single points in $A$ has zero $\phi$-measure,
and if $C$ is countable, from the countable subadditivity of measure we have
\[
  \phi(C) \le \sum_{x\in C} \phi(\{x\}) = 0,
\]
i.e. $C$ has zero $\phi$-measure, and this is in conflict with the positive
$\phi$-measure in the definition of $C$.

Noting that $|\Gamma |\le 2^{\aleph_0}$, we wellorder $\Gamma$ so that, for each
$C\in \Gamma$, the set $\Gamma_C$ of all predecessors of $C$ has cardinal
strictly less than $2^{\aleph_0}$. 
\footnote{Why is this possible?}
By induction with respect to this wellordering we define function $f$ and $g$ on
$\Gamma$ such that, for each $C\in\Gamma$, $f(C)$ and $g(C)$ are distinct
elements of
\[
  C\setminus [f(\Gamma_C) \bigcup g(\Gamma_C)];
\]
this is possible because
\[
  |f(\Gamma_C) \bigcup g(\Gamma_C)| =2 |\Gamma_C| < 2^{\aleph_0}=|C|.
\]
Noting also that for each distinct $C,D\in\Gamma$, we have $f(C)\neq f(D)$ and 
$g(C)\neq g(D)$, and we also have $\img(f)\cap \img(g)=\emptyset$.
Since both the image $\img(f)$ and $A\setminus\img(f)$ (which contains $\img(g)$) meet 
every member of $\Gamma$, neither set contains any member of $\Gamma$,
\footnote{For each $C\in\Gamma$, $g(C)\in C$, $f(C)\in C$, and $f(C)\neq g(C)$,
  $g(C)\notin \img(f)$, hence $C\nsubseteq \img(f)$. And 
  $f(C)\notin A\setminus\img(f)$, hence $C\nsubseteq A\setminus f(\Gamma)$.}
i.e. neither $\img(f)$ nor $A\setminus\img(f)$ contains any closed subsets with
positive measure. 

Using the property of Borel regular measure in Proposition \ref{P:borel_reg}(2),
if both $\img(f)$ and $A\setminus\img(f)$ are $\phi$-measurable, then for any
$\epsilon>0$ there exists a closed set $C\subset \img(f)$ (or 
$C\subset (A\setminus\img(f))$) such that $\phi(\img(f)\setminus C)<\epsilon$
(or $\phi(A\setminus\img(f)\setminus C)<\epsilon$), hence 
\[
  \phi(\img(f)) = \phi(A\setminus\img(f))=0,
\]
thus $\phi(A)=0$, this is in conflict with the assumption that $\phi(A)>0$,
hence either $\img(f)$ or $\img(g)$ is $\phi$-nonmeasurable.

\end{proof}   %%%%%




%%%%%%%%%%%%%%%%%%%%%%%%%%%%%%%%%%%%%%%%%%%%%
%%         Federer 2.2.5
%%%%%%%%%%%%%%%%%%%%%%%%%%%%%%%%%%%%%%%%%%%%%
\section{Radon Measure}

%%%%%%%%%%%%%%%%%%%%%%%%%%%%%%%%%%%%%%%%%%%%%
%%      subspace
%%%%%%%%%%%%%%%%%%%%%%%%%%%%%%%%%%%%%%%%%%%%%
\begin{definition}
\footnote{Engelking, General Topology, 2.1, p.66}
Given a topological space $(X,O)$, for any subset $M\subset X$, taking the 
family $\{M\cap U: U\in O\}$ as the family of open sets in $M$, we define a
topology on $M$, the set $M$ with this topology is called a
\textbf{subspace of the space $X$}, 
\index{subspace}
and the topology itself is called the \textbf{induced topology} or the
\textbf{subspace topology}.
\index{subspace topology}.
\end{definition}

%%%%%%%%%%%%%%%%%%%%%%%%%%%%%%%%%%%%%%%%%%%%%
%%   Hausdorff space
%%%%%%%%%%%%%%%%%%%%%%%%%%%%%%%%%%%%%%%%%%%%%
\begin{definition} \label{D:haus}
\footnote{Engelking, General Topology, 1.5, p.37}
A topological space $X$ is called a 
\textbf{$T_2$-space}, \index{$T_2$-space}
or a 
\textbf{Hausdorff space}, \index{Hausdorff space}
if for every pair of distinct points $x_1,x_2\in X$
there exist open sets $U_1,U_2$ such that $x_1\in U_1, x_2\in U_2$ and 
$U_1\cap U_2=\emptyset$.
\end{definition}

%%%%%%%%%%%%%%%%%%%%%%%%%%%%%%%%%%%%%%%%%%%%%
%%  cover, subcover
%%%%%%%%%%%%%%%%%%%%%%%%%%%%%%%%%%%%%%%%%%%%%
\begin{definition}
\footnote{Engelking, General Topology, 3.1, p.123}
A \textbf{cover}
\index{cover}
of a set $X$ is a family $\{A_s\}_{s\in S}$ of subsets of $X$ such that
$X\subset\bigcup_{s\in S} A_s$,
\footnote{Engelking uses $X=\bigcup_{s\in S} A_s$ in the definition of cover
    which is probably unnecessary.}
and if $X$ is a topological space, $\{A_s\}_{s\in S}$
is an open (or closed) cover of $X$ if all sets $A_s$ are open (closed).
A cover $\{A'_s\}_{s\in S'}$ of subsets of $X$ is a \textbf{subcover}
\index{subcover}
of another cover $\{A_s\}_{s\in S}$ if $S'\subset S$ and $A'_s=A_s$ for every
$s\in S'$.
\end{definition}

%%%%%%%%%%%%%%%%%%%%%%%%%%%%%%%%%%%%%%%%%%%%%
%%    compact space
%%%%%%%%%%%%%%%%%%%%%%%%%%%%%%%%%%%%%%%%%%%%%
\begin{definition}
\footnote{Engelking, General Topology, 3.1, p.123, note that some authors do not
use Hausdorff space in the definition of compact space.}
A topological space $X$ is called a \textbf{compact space} 
\index{compact space}
if $X$ is a Hausdorff
space and every open cover of $X$ has a finite cover, i.e., if for every open
cover $\{U_s\}_{s\in S}$ of the space $X$ there exisits a finite set 
$\{s_1,s_2,\cdots,s_k\}\subset S$ such that
$X=U_{s_1}\cup U_{s_2}\cup \cdots \cup U_{s_k}$.
A subset $K$ of $X$ is called compact provided that $K$ is compact under the
subspace topology inherited from $X$.
\end{definition}

%%%%%%%%%%%%%%%%%%%%%%%%%%%%%%%%%%%%%%%%%%%%%
%%  closed sets are compact
%%%%%%%%%%%%%%%%%%%%%%%%%%%%%%%%%%%%%%%%%%%%%
\begin{proposition} \label{P:compact1}
\footnote{Engelking, General Topology, 3.1.2, p.124, Royden, Real Analysis, 4th
    ed., Proposition 15, 11.5, p.234. Note this does not require $X$ to be a
    Hausdorff space.}
A closed subset $K$ of a compact topological space $X$ is compact.
\end{proposition}
\begin{proof}
Let $\{U_s\}_{s\in S}$ be an open cover on $K$, then 
$\{U_s\}_{s\in S} \bigcup (X\setminus K)$ is an open cover on $X$ because
$X\setminus K$ is open. Since $X$ is compact, there exists a finite subcover,
and, by possibly removing the set $X\setminus K$, the remaining collection is a
finite subcollection of $\{U_s\}_{s\in S}$ covering $K$. Hence $K$ is compact.
\end{proof}

%%%%%%%%%%%%%%%%%%%%%%%%%%%%%%%%%%%%%%%%%%%%%
%%  compact sets in Hausdorff space are closed.
%%%%%%%%%%%%%%%%%%%%%%%%%%%%%%%%%%%%%%%%%%%%%
\begin{proposition} \label{P:compact2}
\footnote{Engelking, General Topology, 3.1.8, p.125, Royden, Real Analysis, 4th
    ed., Proposition 16, 11.5, p.234.}
Every compact subspace $K$ of a Hausdorff space $X$ is a closed subspace of $X$.
\end{proposition}
\begin{proof}
Since $X$ is Hausdorff, for any $x\in K$ and any $y\in X\setminus K$ there
exists open sets $U_x$ and $V_x$ such that $x\in U_x$, $y\in V_x$, and 
$U_x\cap V_x=\emptyset$. Now since $K=\bigcup \{U_x: x\in K\}$ is compact, then
there exists a finite number of points $x_1,x_2,\cdots,x_n$ such that
\[
  K\subset \bigcup_{i=1}^n U_{x_i}.
\]
Now let $N_y=\bigcap_{i=1}^n V_{x_i}$, it is obvious $N_y$ is open, and 
$y\in N_y$. It is also easy to see that $K\cap N_y=\emptyset$, hence 
$N_y\subset X\setminus K$, thus 
\[
  X\setminus K=\bigcup\{N_y:y\in X\setminus K\}.
\]
Hence $X\setminus K$ is open, thus $K$ is closed.
\end{proof}

%%%%%%%%%%%%%%%%%%%%%%%%%%%%%%%%%%%%%%%%%%%%%
%%    locally compact space
%%%%%%%%%%%%%%%%%%%%%%%%%%%%%%%%%%%%%%%%%%%%%
\begin{definition}
\footnote{Engelking, General Topology, 3.3, p.148}
A topological space $X$ is called a \textbf{locally compact space}
\index{locally compact space}
if for every
$x\in X$ there exists a neighborhood $U$ of the point $x$ such that the closure
$\overline{U}$ is a compact subspace of $X$.
\end{definition}

%%%%%%%%%%%%%%%%%%%%%%%%%%%%%%%%%%%%%%%%%%%%%
%%    Radon measure
%%%%%%%%%%%%%%%%%%%%%%%%%%%%%%%%%%%%%%%%%%%%%
\begin{definition}
By a \textbf{Radon measure} 
\index{Radon measure} 
\footnote{Federer 2.2.5, p.62, some authors do not use locally compact Hausdorff space
    in their defintions of Radon measure.}
we mean a measure $\phi$ over a locally compact
Hausdorff space $X$, with the following three properties:
\begin{enumerate}
  \item[(1)] If $K$ is a compact subset of $X$, then $\phi(K)<\infty$.
  \item[(2)] If $V$ is an open subset of $X$, then $V$ is $\phi$-measurable and
    \[
      \phi(V)=\sup\{ \phi(K): \text{$K$ is compact}, K\subset V \}.
    \]
  \item[(3)] If $A$ is any subset of $X$, then
    \[
      \phi(A)=\inf\{ \phi(V): \text{$K$ is open}, A\subset V \}.
    \]
\end{enumerate}
\end{definition}

%%%%%%%%%%%%%%%%%%%%%%%%%%%%%%%%%%%%%%%%%%%%%
%%   support
%%%%%%%%%%%%%%%%%%%%%%%%%%%%%%%%%%%%%%%%%%%%%
\begin{definition}
Whenever $\phi$ measures the space of a topology $T$ one defines the
\textbf{support}
\index{support}
\footnote{Federer 2.2.1, p.60}
of $\phi$ as the closed set
\[
  \spt \phi = X\setminus\bigcup\{V: \text{$V\in T$ and $\phi(V)=0$} \}.
\]
\end{definition}


%%%%%%%%%%%%%%%%%%%%%%%%%%%%%%%%%%%%%%%%%%%%%
\begin{proposition}
\footnote{Federer 2.2.5, p.63}
If $\phi$ is a Radon measure, then
\[
  \phi(X\setminus\spt\phi) = 0.
\]
\end{proposition}
\begin{proof}
From the definiton of support it is easy to see that $X\setminus\spt\phi$ is an
open set and it is the union of all open sets with zero measure. Thus each of
its compact subsets is contained in the union of finitely open sets with
$\phi$-measure zero, hence has $\phi$-measure zero. 
Hence from (2) of the definiton of Radon measure we conclude that 
$X\setminus\spt\phi$ also has zero $\phi$-measure.
\end{proof}

%%%%%%%%%%%%%%%%%%%%%%%%%%%%%%%%%%%%%%%%%%%%%
\begin{proposition}
\footnote{Federer 2.2.5, p.63}
If $\phi$ is a Radon measure, $A$ is $\phi$-measurable, $\phi(A)<\infty$ and
$\epsilon>0$, then $A$ contains a compact set $K$ with 
$\phi(A\setminus K)<\epsilon$.
\end{proposition}
\begin{proof}
We choose an open set $V$ such that $A\subset V$ and 
$\phi(V\setminus A)<\epsilon/2$. 
\footnote{This is possible because from (3) of the definition of the Radon 
    measure, there exists an open set $V$ such that $A\subset V$ and
    $\phi(V)>\phi(A)+\epsilon/2$. And because $A$ is $\phi$-measurable, we then
    have $\phi(V)=\phi(A)+\phi(V\setminus A)$. And because $\phi(A)<\infty$, we
    can do substraction and get $\phi(V\setminus A)<\epsilon/2$.}
We then choose another open set $W$ such that 
$V\setminus A\subset W$ and $\phi(W)<\epsilon/2$. 
Next, because of (2) of the definition of Radon measure, we can choose a compact
set $C$ with $C\subset V$ and $\phi(V\setminus C)<\epsilon/2$. Now 
$C$ is closed because of Proposition \ref{P:compact2}, thus $C\setminus W$ is
closed too, and from Proposition \ref{P:compact1}, we conclude that 
$C\setminus W$ is compact too. Finally we have
\[
  C\setminus W \subset V\setminus W \subset A,
\]
and 
\[
  \phi(A\setminus (C\setminus W)) \le \phi(V\setminus (C\setminus W))
    \le \phi(V\setminus C) + \phi(W) = \epsilon.
\]
\end{proof}


%%%%%%%%%%%%%%%%%%%%%%%%%%%%%%%%%%%%%%%%%%%%%
%%         Federer 2.2.6
%%%%%%%%%%%%%%%%%%%%%%%%%%%%%%%%%%%%%%%%%%%%%
\section{The Space of Sequences of Positive Integers}

Let $\mathcal{P}$ be the set of all positive integers. The set
\[
  \mathcal{N} = \mathcal{P}^{\mathcal{P}}
\]
of all infinite sequences of positive integers 
(also called Baire space)
\index{Baire space}
is a cartesian product with all
factors equal to $\mathcal{P}$. Using the discrete topology
\footnote{the topology of all subsets}
on each factor $\mathcal{P}$ we put the cartesian product topology on
$\mathcal{N}$. This topology can also be induced by a metric, the distance
between two sequences $m$ and $n$ equals
\begin{equation} \label{E:dist_seq}
  \sum_{i=1}^{\infty} 2^{-i} \frac{|m_i-n_i|}{1+|m_i-n_i|}.
\end{equation}
The metric space $\mathcal{N}$ is complete and separable
\footnote{Federer 2.2.6, p.63}. The completeness is a special case of the 
following theorem, while the separability is the consequence of 
Proposition \ref{P:prod_sep} below.

%%%%%%%%%%%%%%%%%%%
\begin{theorem} \label{T:prod_complt}
\footnote{Kuratowski, Topology, vol.1, \S 33, III, Theorem 2, p.406}
The cartesian product $\mathcal{X}_1 \times\mathcal{X}_2 \times \cdots$ of a
sequence of complete spaces is a complete space, if it is metrized by the
formula
\[
  |z-y| = \sum_{i=1}^{\infty} 2^{-i} \frac{|z^i-y^i|}{1+|z^i-y^i|},
\]
where $z$ denotes the sequence $z^1,z^2,\cdots$ and $y$ denotes the sequence 
$y^1,y^2,\cdots$.
\end{theorem}
\begin{proof}
For any index $i$, and any $\epsilon>0$, let 
\[
  \epsilon_1 = \frac{\epsilon}{2^i (1+\epsilon)},
\]
For any Cauchy sequence $z_1,z_2,\cdots$ in product space
$\mathcal{X}_1 \times\mathcal{X}_2 \times \cdots$, there exists an index $n$
such that for $|z_k-z_n|<\epsilon_1$ for all $k>n$. It is easy to see that
\[
  |z_k^i-z_n^i| < \frac{2^i |z_n-z_k|}{1- 2^i |z_n-z_k|} = \epsilon,
\]
hence sequence $z_1^i,z_2^i,z_3^i,\cdots$ is a Cauchy sequence in space
$\mathcal{X}_i$. Since $\mathcal{X}_i$ is a complete space for all $i$, there 
exists a $z^i=\lim_{n\to\infty} z_n^i$.
Next we verify that sequence 
\[
  z= \{ z^1,z^2,z^3,\cdots\} 
   = \{\lim_{n\to\infty} z_n^1, \lim_{n\to\infty} z_n^2, \cdots \}
\]
is the limit of Cauchy
sequence $z_1,z_2,\cdots$. For any $\epsilon>0$ let 
\[
  \epsilon_2=\frac{\epsilon}{1-\epsilon},
\]
then for all $i$ there exists an index $n_i$ such that $|z_k^i-z^i|<\epsilon_2$
for any $k>n_i$. Now let
\[
  n = \sup\{n_1,n_2,\cdots\},
\]
then for any $k>n$ we have
\[
  |z_k-z| = \sum_{i=1}^{\infty} 2^{-i} \frac{|z_k^i-z^i|}{1+|z_k^i-z^i|}
          <  \sum_{i=1}^{\infty} 2^{-i} \epsilon = \epsilon.
\]
\end{proof}

%%%%%%%%%%%%%%%%%%%%%%%%%%%%%%%%%%%%%%%%%%%%%%%%%%
\begin{proposition}
\footnote{Engelking, General Topology, 1.1, p. 12}
Any base $\mathcal{B}$ of a topological space $(X,\mathcal{O})$ has the 
following properties:
\begin{enumerate}
  \item[(B1)]For any $U_1,U_2\in\mathcal{B}$ and every point $x\in U_1\cap U_2$
             there exists a $U\in\mathcal{B}$ such that 
             $x\in U\subset U_1\cap U_2$.
  \item[(B2)] For every $x\in X$ there exists a $U\in\mathcal{B}$ such that
              $x\in U$.
\end{enumerate}
\end{proposition}
\begin{proof}
It is easy to check that a family $\mathcal{B}$ of subsets of $X$ is a base for
the topological space $(X,\mathcal{O})$ iff $B\subset \mathcal{O}$ and for every
$x\in X$ and neighborhood $V$ of $x$ there exists a $U\in\mathcal{B}$ such that
$x\in U\subset V$. Then (B1) follows because $U_1\cap U_2$ is a neighborhood of
$x$, and (B2) follows because $X$ is an open set.
\end{proof}


%%%%%%%%%%%%%%%%%%%%%%%%%%%%%%%%%%%%%%%%%%%%%%%%%%
\begin{proposition}[Topology generated by a base]  \label{P:top_base}
\footnote{Engelking, General Topology, 1.2.1, p. 21}
Suppose we are given a set $X$ and a family $\mathcal{B}$ of subsets of $X$ 
which has properties (B1)-(B2). Let $\mathcal{O}$ be the family of all subsets
of $X$ that are unions of subfamilies of $\mathcal{B}$, i.e., let 
$U\in \mathcal{O}$ iff $U=\cup\mathcal{B}_0$ for a subfamily $\mathcal{B}_0$
of $\mathcal{B}$. The family $\mathcal{O}$ satisfies conditions (O1)-(O3) and 
the family $\mathcal{B}$ is a base for the topological space $(X,\mathcal{O})$.

The topology $\mathcal{O}$ is called the topology generated by the base
$\mathcal{B}$.
\index{topology generated by a base}
\end{proposition}
%%%%%%%%%%%%%%%%%%
\begin{proof}
Condition (O1) is satisfied because $\emptyset\subset\mathcal{B}$ and 
$\emptyset=\cup\emptyset$, and by (B2), $X=\cup_{x\in X}U_x$ where 
$x\in U_x\in\mathcal{B}$.

Take $U_1,U_2\in\mathcal{O}$; we then have $U_1=\cup_{s\in S}U_s$
and $U_2=\cup_{t\in T}U_t$ where $U_s,U_t\in\mathcal{B}$ for $s\in S$ and
$t\in T$. Since
\[
  U_1\cap U_2= \bigcup_{s\in S, t\in T} U_s\cap U_t,
\]
to show that condition (O2) is satisfied, it is enough to prove that
$U_s\cap U_t$ is the union of a subfamily of $\mathcal{B}$. By (B1), for
every $x\in U_s\cap U_t$ there exists a $U(x)\in\mathcal{B}$ such that
\[
  x\in U(x) \subset U_s\cap U_t,
\]
thus
\[
  U_s\cap U_t = \bigcup_{x\in U_s\cap U_t} U(x).
\]

Condition (O3) is satisfied by the definition of the family $\mathcal{O}$.

Clearly $\mathcal{B}$ is a base for the space $(X,\mathcal{O})$.
\end{proof}


%%%%%%%%%%%%%%%%%%%%%%%%%%%%%%%%%%%%%%%%%%%%%%%%%%%%
\begin{proposition}[Topology generated by functions] \label{P:top_fun}
\footnote{Engelking, General Topology, 1.4.8, p. 31}
Suppose we are given a set $X$, a family $\{Y_s\}_{s\in S}$ of topological
spaces and a family of mappings $\{f_s\}_{s\in S}$, where $f_s:X\to Y_s$ is
a mapping of $X$ to $Y_s$. In the class of all topologies on $X$ that make 
all the $f_s$'s continuous there exists a coarsest 
\footnote{A topology $O_2$ is coarser than topology $O_1$ if $O_2\subset O_1$}
topology; this is the 
topology $\mathcal{O}$ generated by the base consisting of all sets of the
form $\cap^k_{i=1} f^{-1}_{s_i}(V_{s_i})$, where $s_1,s_2,\dots,s_k\in S$
and $V_{s_i}$ is an open subset of $Y_{s_i}$ for $i=1,2,\dots,k$.

The topology $\mathcal{O}$ is called the topology generated by the family
of mappings $\{f_s\}_{s\in S}$. 
\index{topology generated by functions}
\end{proposition}
\begin{proof}
Let 
\[
  \mathcal{B} = \{ \bigcap^k_{i=1} f^{-1}_{s_i}(V_{s_i}): 
                   V_{s_i}\in (Y_{s_i},\mathcal{O}_{s_i}),
                   s_1,s_2,\dots,s_k\in S 
                \},
\]
First we verify that it satisfies property (B1), let $U_1,U_2\in\mathcal{B}$ 
and write
\[
  U_1 = \bigcap^k_{i=1} f^{-1}_{s_i}(V_{s_i}) 
        \qquad s_1,s_2,\dots,s_k\in S, V_{s_i}\in (Y_{s_i},\mathcal{O}_{s_i}),
\]
and
\[
  U_2 = \bigcap^j_{i=1} f^{-1}_{t_i}(V_{t_i}) 
        \qquad t_1,t_2,\dots,t_j\in S, V_{t_i}\in (Y_{t_i},\mathcal{O}_{t_i}),
\]
then
\[
  U_1\cap U_2 = \left(  \bigcap^k_{i=1} f^{-1}_{s_i}(V_{s_i}) \right)
               \bigcap \left( \bigcap^j_{i=1} f^{-1}_{t_i}(V_{t_i}) \right) 
    \in \mathcal{B},
\]
because for any $U,V\in (Y_s,\mathcal{O}_s)$ we have 
$U\cap V\in (Y_s,\mathcal{O}_s)$ and
\[
  f_s^{-1}(U) \cap f_s^{-1}(V) = f_s^{-1}(U\cap V).
\]

To verify that $\mathcal{B}$ satisfies (B2), note that for every $x\in X$ and 
every $s\in S$, we have $f_s^{-1}(Y_s)=X$ thus 
$x\in f_s^{-1}(Y_s)\in\mathcal{B}$.

Hence by Proposition \ref{P:top_base}, there is a topology $\mathcal{O}$ 
generated by base $\mathcal{B}$. It is trivial to verify that for every 
$s\in S$, function $f_s:(X,\mathcal{O})\to (Y_s,\mathcal{O}_s)$ is continuous.
Actually for every $U\in (Y_s,\mathcal{O}_s)$, we have 
$f_s^{-1}(U)\in \mathcal{B}\subset \mathcal{O}$.

To verify that $\mathcal{O}$ is the coarest (i.e. smallest) topology that makes
all functions $\{f_s\}_{s\in S}$ continuous, suppose $\mathcal{O}'$ is a 
topology on $X$ under which function $f_s$ is continuous for every $s\in S$. 
Then for every $U_s\in (Y_s,\mathcal{O}_s)$ we have 
$f_s^{-1}(U_s)\in \mathcal{O}'$, hence for every $s_1,s_2,\dots,s_k\in S$ and 
every $V_{s_i}\in (Y_{s_i},\mathcal{O}_{s_i})$ we have
\[
  \bigcap_{i=1}^k f^{-1}_{s_i}(V_{s_i})\in\mathcal{O}'.
\]
Hence $\mathcal{B}\subset\mathcal{O}'$, thus $\mathcal{O}\subset\mathcal{O}'$.
\end{proof}

%%%%%%%%%%%%%%%%%%%%%%%%%%%%%%%%%%%%%%%%%%%%%
\begin{definition}
\footnote{Engelking, General Topology, 2.3, p. 77}
\index{product topology} \index{Tychonoff topology}
Suppose we are given a family $\{X_s\}_{s\in S}$ of topological spaces; 
consider the Cartesian product $X=\prod_{s\in S} X_s$ and the family of 
mappings $\{p_s\}_{s\in S}$ where $p_s:X\to X_s$ and for any 
$x=\{x_s\}\in X$, $p_s(x)=x_s$. The set $X=\prod_{s\in S} X_s$ with the
topology generated by the family of mappings $\{p_s\}_{s\in S}$ is called
the Cartesian product of the spaces $\{X_s\}_{s\in S}$ and the topology 
itself is called the Tychonoff topology on $\prod_{s\in S} X_s$; the 
mappings $p_s:\prod_{s\in S}X_s \to X_s$ are called the projections of
$X=\prod_{s\in S} X_s$ onto $X_s$.
\end{definition}

%%%%%%%%%%%%%%%%%%%%%%%%%%%%%%%%%%%%%%%%%%%%%
\begin{proposition} \label{P:prod_base}
\footnote{Engelking, General Topology, 2.3.1, p. 77}
The family of all sets $\prod_{s\in S} W_s$, where $W_s$ is an open subset of 
$X_s$ and $W_s\neq X_s$ only for finitely many $s\in S$, is a base for the
Cartesian product $\prod_{s\in S} X_s$.
\end{proposition}
%%%%%%%%%%%%%%%%%%
\begin{proof}
By Proposition \ref{P:top_fun}, the family of all sets of the form 
$\bigcap_{i=1}^k p^{-1}_{s_i}(W_{s_i})$ where $s_1,s_2,\dots,s_k\in S$ and
$W_{s_i}$ is open in $X_{s_i}$, is a base for $\prod_{s\in S} X_s$. Hence 
to prove the proposition it suffices to observe that
\[
  p^{-1}_{s_i}(W_{s_i}) = \prod_{s\in S} V_{s_i,s}, 
\]
where
\[
  V_{s_i,s}=
    \begin{cases} 
      X_s     & \text{if $s\neq s_i$}    \\
      W_{s_i} & \text{if $s=s_i$}
    \end{cases},
\]
and that
\[
  \bigcap_{i=1}^k p^{-1}_{s_i}(W_{s_i})
    = \bigcap_{i=1}^k \prod_{s\in S} V_{s_i,s}
    = \prod_{s\in S} \left( \bigcap_{i=1}^k V_{s_i,s} \right),
\]
where
\[
  \bigcap_{i=1}^k V_{s_i,s} \in (X_s,\mathcal{O}_s),
\]
and there are at most $k$ (finite many) indice $s$ such that
\[
  \bigcap_{i=1}^k V_{s_i,s} \neq X_s.
\]
\end{proof}

%%%%%%%%%%%%%%%%%%%%%%%%%%%%%%%%%%%%%%%%%%%%%
\begin{proposition} \label{P:prod_sep}
Countable product of separable spaces is separable.
\footnote{Adapted from Henno Brandsma, Topology Atlas website.}
\end{proposition}
%%%%%%%%%%%%%%%%%%
\begin{proof}
Suppose $X_1,X_2,\dots$ are separable topological spaces, then there exists
for every $i\in N$ there exists a dense set $A_i$ of $X_i$ such that
$|A_i|=\aleph_0$.
For every $n\in N$, let 
\[
  B_n = \prod_{i\leq n} A_i \times \prod_{i>n} \{q_i\}
      \qquad \text{where $q_i\in X_i$},
\]
i.e. for any $\{x_i\}\in B_n$, $x_i\in A_i$ if $i\leq n$, and $x_i = q_i$ 
if $i>n$. It is easy to see that $|B_n|=\aleph_0$ (finite product of 
countable sets).

Now let $B=\cup_n B_n$, then $|B|=\aleph_0$ (countable union of 
countable sets is countable). To prove the proposition we only need to 
verify that $B$ is dense in $\prod_i X_i$. From Proposition \ref{P:prod_base}
we know that the family of all sets $\prod_i W_i$, where 
$W_i\in (X_i,\mathcal{O}_i)$ and $W_i\neq X_i$ only for finitely many $i\in N$,
is a base for $\prod_i X_i$. To verify $B$ is dense it suffices to verify that
$B\cap \prod_i W_i\neq \emptyset$. Indeed, let $n=\max\{i: W_i\neq X_i\}$, then
\[
  B_n \bigcap \prod_i W_i = \prod_{i\leq n} \left( W_i \bigcap A_i \right)
                     \times \prod_{i>n} \left( X_i \bigcap \{q_i\} \right)
                          \neq \emptyset,
\]
hence $B\cap \prod_i W_i\neq\emptyset$.
\end{proof}

%%%%%%%%%%%%%%%%%%%%%%%%%%%%%%
\begin{definition} \label{D:cont_metr}
A function $f:X\to Y$ between two metric spaces $(X,d_X)$ and $(Y,d_Y)$ is
called \textbf{continuous}
  \index{continuous function between metric spaces} 
  \footnote{Royden, Real Analysis, 4th ed., section 9.3, p.191, Rudin,
     Principles of Mathematical Analysis, Definiton 4.5, p.85,
     see also Proposition \ref{P:cont2}.}
at $p\in X$ if for every $\epsilon>0$ there exists a $\delta>0$ such that
\[
	d_Y (f(x), f(p)) < \epsilon
\]
for all points $x\in X$ for which $d_X(x,p)<\delta$.
\end{definition}

%%%%%%%%%%%%%%%%%%%%%%%%%%%%%%
\begin{definition}
A mapping $f$ from a topological space $X$ to a topological space $Y$ is
said to be a \textbf{homeomorphism}
	\index{homeomorphism} 
	\footnote{Royden, Real Analysis, 4th ed., section 11.4, p.231}
provide it is one-to-one and onto, and both $f$ and $f^{-1}$ are continuous.
\end{definition}

%%%%%%%%%%%%%%%%%%%%%%%%%%%%%%%%%%%%%%%%%%%%%
\begin{lemma} \label{P:prod_sep}
The space $\mathcal{N}$ is the union of the sets
\begin{equation}
  \mathcal{N}_j = \{ n\in\mathcal{N}: n_1=j \},
\end{equation}
corresponding to $j\in\mathcal{P}$. Each set $\mathcal{N}_j$ is open and closed
in $\mathcal{N}$, and homeomorphic with $\mathcal{N}$.
\footnote{Federer 2.2.6, p.63}
\end{lemma}
%%%%%%%%%%%%%%%%%%
\begin{proof}
We may use for $\mathcal{N}$ the metric defined in Eq. \ref{E:dist_seq}:
\[
	d_{\mathcal{N}}(m,n) 
	  = \sum_{i=1}^{\infty} 2^{-i} \frac{|m_i-n_i|}{1+|m_i-n_i|}.
\]
For any $n\in\mathcal{N}_j$, there exist an open ball $B(n,1/4)$ such that
$B(n,1/4)\subset\mathcal{N}_j$, hence $\mathcal{N}_j$ is open in $\mathcal{N}$.

Again, for any $n\in\mathcal{N}^C_j$, there exist an open ball $B(n,1/4)$ such that
$B(n,1/4)\subset\mathcal{N}^C_j$, hence $\mathcal{N}^C_j$ is open in
$\mathcal{N}$, i.e. $\mathcal{N}_j$ is closed in $\mathcal{N}$.

Define a function $f: \mathcal{N}_j \to \mathcal{N}$ such that for any 
$n=(j,n_2,n_3,\dots)\in\mathcal{N}_j$
\[
  f(n)=(n_2,n_3,\dots),
\]
it is easy to see that $f$ is one-to-one and onto, and both $f$ and 
$f^{-1}$ are continuous, hence $f$ is homeomorphic.
\end{proof}

%%%%%%%%%%%%%%%%%%%%%%%%%%%%%%%%%%%%%%%%%%%%%
\begin{proposition} \label{P:cont_base}
For a mapping of a topological space $(X,O)$ to a topological space $(Y,O')$ the
following conditions are equivalent:
  \footnote{Engelking, General Topology, Proposition 1.4.1, p.28}
\begin{enumerate}
  \item[(i)] The mapping $f$ is continuous.
  \item[(ii)] Inverse images of all members of a subbase $\mathcal{P}$ for $Y$
     are open in $X$.
  \item[(ii')] Inverse images of all members of a base $\mathcal{B}$ for $Y$
     are open in $X$.
\end{enumerate}
\end{proposition} 
%%%%%%%%%%%%%%%%%%
\begin{proof}
The implication $(i)\Rightarrow (ii)$ is obvious.

Next we verify $(ii)\Rightarrow (ii')$. Let $\mathcal{P}$ be a subbase of $Y$
such that $f^{-1}(V)$ is open in $X$ for every $V\in\mathcal{P}$. Consider the
base $\mathcal{B}$ for $Y$ consisting of all finite intersections 
$V_1\cap V_2\cap \dots\cap V_k$ of members of $\mathcal{P}$; since
\[
  f^{-1}(V_1\cap V_2\cap \dots\cap V_k)
  = f^{-1}(V_1) \cap f^{-1}(V_2) \cap \dots\cap f^{-1}(V_k),
\]
inverse images of all members are open in $X$.

Finally we verify $(ii')\Rightarrow (i)$. Suppose for any $V$ in $\mathcal{B}$,
$f^{-1}(V)$ is open in $X$. Then for any open set $U$ in $Y$, there exist 
$\mathcal{B}_U\subset \mathcal{B}$ such that $U=\bigcup_{V\in \mathcal{B}_U} V$.
Since
\[
  f^{-1}(U) = f^{-1}(\bigcup_{V\in \mathcal{B}_U} V)
            = \bigcup_{V\in \mathcal{B}_U} f^{-1}(V),
\]
we get that $f^{-1}(U)$ is open in $X$.
\end{proof}














%%%%%%%%%%%%%%%%%%%%%%%%%%%%%%%%%%%%%%%%%%%%%
\begin{proposition} \label{P:prod_conti1}
\footnote{Engelking, General Topology, 1.4.9, p.31}
A mapping $f$ of a topological space $X$ to a topological space $Y$ whereof the
topology is generated by a family of mappings $\{f_s\}_{s\in S}$, where $f_s$ is
a mapping of $Y$ to $Y_s$, is continuous iff the composition $f_s f$ is
continuous for every $s \in S$.
\end{proposition}
%%%%%%%%%%%%%%%%%%
\begin{proof}
If $f:X\to Y$ is continuous, then $f_s f$ is continuous as the composition of
two continuous mappings. 

Let us suppose that $f_s f: X\to Y_s$ is continuous for every $s\in S$. Since
the topology of $Y$ is generated by the family of functions $\{f_s\}_{s\in S}$, 
by Proposition \ref{P:top_fun}, there exists a subbase $\mathcal{P}$ of $Y$
consisting of all sets of the form $f_s^{-1}(V_s)$, where $V_s$ is open in
$Y_s$. By Proposition \ref{P:cont_base}, to verify that $f$ is continuous,
it suffices to show that inverse images of members of $\mathcal{P}$ under 
the mapping $f$ are open in $X$. This follows directly from the equality
\[
  f^{-1}(f_s^{-1}(V_s)) = (f_s f)^{-1}(V_s).
\]
\end{proof}

%%%%%%%%%%%%%%%%%%%%%%%%%%%%%%%%%%%%%%%%%%%%%
\begin{proposition} \label{P:prod_conti2}
\footnote{Engelking, General Topology, 2.3.6, p.78}
A mapping $f$ of a topological space $X$ to a Cartesian product 
$\prod_{s\in S} Y_s$ is continuous iff the composition $p_s f$ is continuous for
every $s\in S$.
\end{proposition}
%%%%%%%%%%%%%%%%%%
\begin{proof}
Directly from Proposition \ref{P:prod_conti1}.
\end{proof}


%%%%%%%%%%%%%%%%%%%%%%%%%%%%%%%%%%%%%%%%%%%%%
\begin{proposition} \label{P:prod_assoc}
\footnote{Engelking, General Topology, 2.3.7, pp.78-79}
Let $\{X_s\}_{s\in S}$ be a family of topological spaces. If 
$S=\bigcup_{t\in T} S_t$, where $S_t\bigcap S_{t'}=\emptyset$ for $t\neq t'$. 
Then the spaces $\prod_{s\in S} X_s$ and 
$\prod_{t\in T} (\prod_{s\in S_t} X_s)$ are
homeomorphic, i.e., the Cartesian product is associative.
\end{proposition}
%%%%%%%%%%%%%%%%%%
\begin{proof}
We define function $f:\prod_{s\in S} X_s \to \prod_{t\in T} (\prod_{s\in S_t} X_s)$ 
as follows. To the point $x=\{x_s\}\in \prod_{s\in S} X_s$ we assign the point
\[
  f(x) = \{x_t\} \in \prod_{t\in T} (\prod_{s\in S_t} X_s),
\]
where $x_t=\{x_s\} \in \prod_{s\in S_t} X_s$. It is easy to see that function
$f$ is both one-to-one and onto.

To verify that $f$ is continuous, from Proposition \ref{P:prod_conti2}, we only 
need to verify that the function 
$p_t f:\prod_{s\in S} X_s \to \prod_{s\in S_t} X_s$ is continuous for every
$t\in T$. Applying Proposition \ref{P:prod_conti2} again, this is true iff the function
$p_s p_t f:\prod_{s\in S} X_s \to X_s$ is continuous for every 
$t\in T, s\in S_t$. Now it is easy to see that $p_s p_t f=p_s$, hence it is 
continuous.

To verify that 
$f^{-1}:\prod_{t\in T} (\prod_{s\in S_t} X_s) \to \prod_{s\in S} X_s$
is continuous, again we apply Proposition \ref{P:prod_conti2}, and this is true
iff the function
$p_s f^{-1}:\prod_{t\in T} (\prod_{s\in S_t} X_s) \to X_s$ is continuous for
every $s\in S$. Now for every $s\in S$, there exists a $t\in T$ such that 
$s\in S_t$, hence $p_s f^{-1}=p_s p_t$, and thus is continuous.

Hence we have verified that $f$ is one-to-one and onto, and that both $f$ and
$f^{-1}$ are continuous, i.e. $f$ is a homeomorphism.
\end{proof}

%%%%%%%%%%%%%%%%%%%%%%%%%%%%%%%%%%%%%%%%%%%%%
\begin{lemma} \label{L:prod_baire}
The cartesian product of countably many factors all equal to $\mathcal{N}$ 
\[
  \mathcal{N}^{\mathcal{P}} = (\mathcal{P}^{\mathcal{P}})^{\mathcal{P}}
\]
is homeomorphic with $\mathcal{N}$.
\footnote{Federer 2.2.6, p.63}
\end{lemma}
%%%%%%%%%%%%%%%%%%
\begin{proof}
Directly from Proposition \ref{P:prod_assoc}.
%% the metric approach: difficult to prove continuity of f and f^{-1}
% First we define $u:\mathcal{P}\times\mathcal{P} \to\mathcal{P}$ as
% \[
%   u(m,n) = 2^{m-1} (2n-1),
% \]
% it is easy to see that $u$ is both one-to-one and onto. For any $k$, we define
% $l(k)$ and $u(k)$ as satisfying
% \[
%   u(l(k),r(k)) = k.
% \]
% 
% Now let $f:\mathcal{N}^{\mathcal{P}}\to \mathcal{N}$, defined by 
% $f(n_1,n_2,\cdots)=m$, where
% \[
%   m(k) = n_{l(k)}(r(k)),
% \]
% It is easy to see that $f$ is one-to-one and onto.
% 
% %We may use for $\mathcal{N}$ the metric defined in Eq. \ref{E:dist_seq}:
% %\[
% %	d_{\mathcal{N}}(m,n) 
% %	  = \sum_{i=1}^{\infty} 2^{-i} \frac{|m_i-n_i|}{1+|m_i-n_i|}.
% %\]
% %
% %TODO: use a different metric? e.g.
% We may use for $\mathcal{N}$ the following metric:
% \[
% 	d_{\mathcal{N}}(m,n) 
% 	  = \frac{1}{1+ \min\{k:m(k)\ne n(k)\} },
% \]
% it is easy to see that $0\le d_{\mathcal{N}}(m,n) \le \frac{1}{2}$.
% 
% Combine this metric with the product metric:
% \[
% 	d(x,y) 
% 	  = \sum_{i=1}^{\infty} 2^{-i} 
%         \frac{d_{\mathcal{N}}(x_i,y_i)}{1+d_{\mathcal{N}}(x_i,y_i)},
% \]
% 
% TODO: prove both $f$ and $f^{-1}$ are continuous.

\end{proof}











%%%%%%%%%%%%%%%%%%%%%%%%%%%%%%%%%%%%%%%%%%%%%
%%         Federer 2.2.7-9
%%%%%%%%%%%%%%%%%%%%%%%%%%%%%%%%%%%%%%%%%%%%%
\section{Lipschitzian Maps}

\begin{definition} \label{D:lipschitz}
\footnote{Federer 2.2.7, p.63}
A map $f:X\to Y$, where $X$ and $Y$ are metric spaces, is called
$\textbf{Lipschitzian}$ 
\index{Lipschitzian}
iff there exists a finite positive number $M$ such that
\[
  dist[f(a),f(b)] \leq M \cdot dist(a,b),
\]
whenever $a,b\in X$; one refers to $M$ as a $\textbf{Lipschitz constant}$ for
$f$. Every Lipschitzian function has a least Lipschitz constant, denoted
\[
  Lip(f).
\]
We say that $f$ is $\textbf{locally Lipschitzian}$
\index{locally Lipschitzian}
iff each point of $X$ has a
neighborhood $U$ such that $f|_U$ is Lipschitzian.
\end{definition}

%%%%%%%%%%%%%%%%%%%%%%%%%%%
%%    vector space
%%%%%%%%%%%%%%%%%%%%%%%%%%%
\begin{definition}
\footnote{Kolmogorov and Fomin, Introductory Real Analysis, 13.1, p.118}
A nonempty set $L$ is said to be a \textbf{linear space} 
\index{linear space}
(or 
\textbf{vector space}
\index{vector space}
if it satisfies the following:
\begin{enumerate}
  \item[(1)] Any two elements $x,y\in L$ uniquely determine a third element (the
      sum) $x+y\in L$, such that
    \begin{enumerate}
      \item[(a)] $x+y=y+x$ (commutativity);
      \item[(b)] $(x+y)+z=x+(y+z)$ (associativity);
      \item[(c)] There exists an element $0\in L$ such that $x+0=x$ for all
        $x\in L$.
      \item[(d)] For every $x\in L$ there exists an element $-x\in L$ such that
        $x+(-x)=0$.
    \end{enumerate}
  \item[(2)] Any number $a$ and any element $x\in L$ uniquely determine an
    element $ax\in L$ called the product of $a$ and $x$ such that
    \begin{enumerate}
      \item[(a)] $a(bx)=(ab)x$;
      \item[(b)] $1x=x$;
    \end{enumerate}
  \item[(3)] The operations of addition and multiplication obey two distributive
    laws:
    \begin{enumerate}
      \item[(a)] $(a+b)x=ax+bx$;
      \item[(b)] $a(x+y)=ax+ay$.
    \end{enumerate}
\end{enumerate}
\end{definition}

%%%%%%%%%%%%%%%%%%%%%%%%%%%
\begin{definition}
A subset $X$ of a vector space $L$ is called \textbf{convex}
\index{convex set}
if for any $x,y\in
X$, $ax+(1-a)y\in X$ for any number $a$ satisfying $1\geq a\geq 0$.
\end{definition}

%%%%%%%%%%%%%%%%%%%%%%%%%%%
\begin{definition}
A nonnegative real-valued function $\|\cdot\|$ defined on a linear space $X$ is
called a \textbf{norm}
\index{norm}
provided for all $u,v\in X$ and $a\in R$:
\[
  \|u\|=0 \Leftrightarrow u=0;
  \|u+v\|\le \|u\| + \|v\|;
  \|au\|= |a| \|u\|.
\]
\end{definition}

%%%%%%%%%%%%%%%%%%%%%%%%%%%
\begin{lemma}
\footnote{Federer 2.2.7, p.64}
If $X$ is a convex subset of a normed vector space, then $f$ is Lipschitzian 
with $Lip(f)\leq M$ iff for $x\in X$
\[
  \limsup_{z\to x} dist[f(x),f(z)]/|x-z| \leq M.
\]
\end{lemma}
\begin{proof}
The necessity is straightforward to verify.
To prove the sufficiency of this condition we suppose $a,b\in X$, $\mu>M$, since
$X$ is convex, $a+t(b-a)\in X$ for all $0\leq t\leq 1$, let
\[
  S=S_{a,b,\mu}=\{ t: 0\leq t\leq 1, dist[f(a),f(a+t(b-a))]\leq \mu t|b-a| \}.
\]
To prove that $f$ is Lipshitzian we just need to verify that $1\in S_{a,b,\mu}$
for all $a,b\in X$ and $\mu>M$. And to prove the latter we just need to verify
that $\sup S\in S$ and $\sup S=1$.

First we verify that $\tau=\sup S\in S$. It is easy to see that 
$0\leq\tau\leq 1$ from the definition of supremum. For any $t\in S$, 
$t\leq \tau$ we have
\begin{align*}
  &dist[f(a),f(a+\tau (b-a))]  \\ \notag
  &\leq dist[f(a),f(a+t(b-a))] + dist[f(a+t(b-a)),f(a+\tau (b-a))] \\ \notag
  &\leq \mu t |b-a| + dist[f(a+t(b-a)),f(a+\tau (b-a))] \notag.
\end{align*}
Let 
\[
  L_r(x) = \sup\{ dist[f(x),f(z)]/|x-z| : |z-x|<r, z\neq x \},
\]
and
\[
  L(x) = \limsup_{z\to x} dist[f(x),f(z)]/|x-z|,
\]
hence $L(x)=\lim_{r\to 0} L_r(x) < M$ for any $x\in X$.
Now let
\[
  \epsilon = \mu - M,
\]
then $\epsilon>0$, and there exists a $\delta>0$, for any $0<r<\delta$, 
\[
  L_r(a+\tau (b-a)) < L(a+\tau (b-a)) + \epsilon < M + \epsilon = \mu.
\]
Now we can choose a $t\in S$ such that 
\[
  (\tau-t)|b-a| \leq r,
\]
this is possible because $\tau=\sup S$. Hence
\begin{align*}
  & dist[f(a+t(b-a)),f(a+\tau (b-a))]  \notag \\
  &\leq L_r(a+\tau (b-a)) (\tau - t) |b-a|  \notag \\
  &\leq \mu (\tau-t) |b-a|  \notag
\end{align*}
and
\begin{align*}
  &dist[f(a),f(a+\tau (b-a))]  \\ \notag
  &\leq \mu t |b-a| + \mu (\tau -t)|b-a| \\ \notag
  &= \mu \tau |b-a|.  \notag
\end{align*}
thus $\tau\in S$.

Next we verify that $\tau=\sup S=1$. Suppose that $\tau<1$, there exist a
$t_1>\tau$ such that $t_1\leq 1$ and 
\[
  (t_1-\tau)|b-a| \leq r,
\]
and 
\begin{align*}
  & dist[f(a+t_1(b-a)),f(a+\tau (b-a))]  \notag \\
  &\leq L_r(a+\tau (b-a)) (t_1 - \tau) |b-a|  \notag \\
  &\leq \mu (t_1-\tau) |b-a|  \notag
\end{align*}
Hence
\begin{align*}
  &dist[f(a),f(a+ t_1 (b-a))]  \\ \notag
  &\leq dist[f(a),f(a+\tau (b-a))] + dist[f(a+t_1(b-a)),f(a+\tau (b-a))] 
    \\ \notag
  &\leq \mu \tau |b-a| + \mu (t_1 - \tau) |b-a| \notag \\
  &= \mu t_1 |b-a|, \notag
\end{align*}
i.e. $t_1\in S$. This is in conflict with $\tau=\sup S$, hence we conclude that
$\sup S=1$.
\end{proof}

%%%%%%%%%%%%%%%%%%%%%%%%%%%%%%%%%%
%%  Cantor intersection theorem
%%%%%%%%%%%%%%%%%%%%%%%%%%%%%%%%%%
\begin{theorem}[Cantor Intersection Theorem] \label{T:cantor_int}
\footnote{Kuratowski, Topology, vol.1, \S 34, II, p.413}
\index{Cantor Intersection Theorem}
Every sequence $A_1\supset A_2\supset\cdots$ of non-empty closed subsets of
the complete metric space $X$ such that $\lim_{n\to\infty} \delta(A_n)=0$, then 
the set $A_1\cap A_2\cap\cdots$ consists of a single point.
\end{theorem}
\begin{proof}
It is easy to see that $A_1\cap A_2\cap\cdots$ can contain at most one point.
Thus we only need to verify that it is non-empty.
By selecting a point $p_n$ from each set $A_n$, we obtain a Cauchy sequence 
$\{p_1,p_2,\cdots\}$. Since $X$ is complete, the limit of this Cauchy sequence
$p$ is in $X$. Since each $A_n$ is closed, it contains the limit of any
convergent sequence of points in $A_n$ (by Proposition \ref{P:conv}).
Thus each $A_n$ contains $p$, and $p\in A_1\cap A_2\cap \cdots$.
\end{proof}

%%%%%%%%%%%%%%%%%%%%%%%%%%%%%%%%%%%%%%%%%%%%%%%%%
%% separable metric space  <==> countable base
%%%%%%%%%%%%%%%%%%%%%%%%%%%%%%%%%%%%%%%%%%%%%%%%%
\begin{proposition} \label{P:metric_base}
\footnote{Royden, Real Analysis, 4th ed., Proposition 25, 9.6, p.204; 
    Kuratowski, Topology, vol.1, \S 21, II, Theorem 2, p.206;
    and more generally Engelking, General Topology, 4.1.15, p.255.}
A metric space $X$ is separable iff it has a countable base, i.e. there exists a
countable collection $\{O_n\}_{n=1}^{\infty}$ of open subsets of $X$ such that
any open subset of $X$ is the union of a subcollection of 
$\{O_n\}_{n=1}^{\infty}$. 
\end{proposition}
\begin{proof}
First we suppose $X$ is separable, i.e. there exists a $D\subset X$ such that 
$\overline{D}=X$ and $|D|\leq\aleph_0$. Thus for any open set $O\subset X$ we have 
$O\cap D\neq\emptyset$. Let $\{x_n\}$ be an enumeration of $D$. Then 
$\{B(x_n,1/m)\}_{n,m\in N}$
is a countable collection of open subsets of $X$. Let $O$ be an open subset of
$X$, and $x\in O$. Then there exists a $m\in N$ such that $B(x,1/m)\subset O$
because $O$ is open. Now we know that 
\[
  B(x,1/2m)\cap D\neq \emptyset
\]
because $D$ is dense, hence we can choose a $x_n\in D\cap B(x,1/2m)$, and
\[
  x\in B(x_n,1/2m) \subset B(x,1/m) \subset O.
\]
Hence any open set is a union of a subcollection of $\{B(x_n,1/m)\}_{n,m\in N}$,
i.e. $\{B(x_n,1/m)\}_{n,m\in N}$ is a countable base of $X$.

To prove the converse, we simply choose a point $x_n\in O_n$ for any $n$. Then
the set $\{x_n\}_{n=1}^{\infty}$ is countable and dense because any open subset
of $X$ is an union of the subcollection of $\{O_n\}_{n=1}^{\infty}$ thus
contains points from set $\{x_n\}_{n=1}^{\infty}$.
\end{proof}

%%%%%%%%%%%%%%%%%%%%%
\begin{lemma}
\footnote{Royden, Real Analysis, 4th ed., Proposition 2, 9.2, p.188.} 
Let $X$ be a subspace of the metric space $Y$ and $E$ a subset of $X$. Then $E$
is open in $X$ iff $E=X\cap O$, where $O$ is open in $Y$.
\end{lemma}
\begin{proof}
When dealing with subspace we must be careful what we mean by open balls.
First suppose that $E$ is open in $X$, hence for any $x\in X$, there is a 
$r_x>0$ such that
\[
  \{z\in X: d(z,x)<r_x\} \subset E.
\]
Hence
\[
  E= \bigcup_{x\in E} \{ z\in X : d(z,x) < r_x \} 
   = X \cap \bigcup_{x\in E} \{ z\in Y : d(z,x) < r_x \},
\]
and $O= \bigcup_{x\in E} \{ z\in Y : d(z,x) < r_x \}$ is an open set in $Y$.

To prove the converse, suppose $O$ is open in $Y$, then for any $x\in O$ there
exists a $r_x>0$ such that
\[
  \{z\in Y: d(z,x)<r_x\} \subset O.
\]
Hence
\[
  O= \bigcup_{x\in O} \{ z\in Y : d(z,x) < r_x \} 
\]
and
\[
  X\cap O= X\cap \bigcup_{x\in O} \{ z\in Y : d(z,x) < r_x \} 
    = \bigcup_{x\in O} \{ z\in X : d(z,x) < r_x \},
\]
i.e. $X\cap O$ is open in $X$.
\end{proof}

%%%%%%%%%%%%%%%%%%%%%
\begin{proposition} \label{P:metric_sep}
\footnote{Royden, Real Analysis, 4th ed., Proposition 26, 9.6, p.204.} 
Every subspace of a separable metric space is separable.
\end{proposition}
\begin{proof}
Let $E$ be a subspace of the separable metric space $X$. From Proposition
\ref{P:metric_base} we conclude that $X$ has a countable base, let's denote it 
$\{O_n\}_{n=1}^{\infty}$. It is easy to see that 
$\{E\cap O_n\}_{n=1}^{\infty}$ is a countable base of $E$. Again from
Proposition \ref{P:metric_base} we conclude that $E$ is separable too.
\end{proof}

%%%%%%%%%%%%%%%%%%%%%
\begin{theorem} \label{T:lipschitz_compltSep}
\footnote{Federer 2.2.8, p.64}
For every complete, separable, nonempty metric space $X$ there exists a locally
Lipschitzian map $g$ of $\mathcal{N}$ onto $X$.
\end{theorem}
\begin{proof}
It is easy to see that for any $A\subset X$ and any $y\in \overline{A}$ and any 
$r>0$, there exists a point $x\in A$ such that $d(x,y)<r$ (from Proposition 
\ref{P:clos}), i.e. $y\in B(x,r)$, thus
\[
  \overline{A} \subset \bigcup_{x\in A} B(x,r) 
          \subset \bigcup_{x\in A} \overline{B}(x,r) \qquad \forall r>0.
\]
Thus if $X$ is separable, i.e. there exists a countable $A\subset X$ such that 
$\overline{A}=X$, then $X$ is the union of countable closed sets,
\[
  X = \bigcup_{x\in A} \{y\in X: d(y,x)\leq r \} \qquad r>0.
\]
Furthermore, since any subspace of a separable metric space is separable
(Proposition \ref{P:metric_sep}), for any nonempty closed set $C\subset X$, the 
subspace $C$ is separable, hence there exists a set $A\subset X$, 
$\overline{A}=C$, $|A|=\aleph_0$, and 
\begin{align*}
  C &= \bigcup_{x\in A} \{y\in C: d(y,x)\leq r \} \qquad r>0, \notag \\
    &= \bigcup_{x\in A} \left( C\bigcap \{y\in X: d(y,x)\leq r \} \right)  
       \notag,
\end{align*}
it is easy to see that $C\bigcap \{y\in C: d(y,x)\leq r \}$ is a nonempty
closed set in $X$. 
Using this we can then proceed by induction
with respect to $k$ by associating every finite sequence $s_1,s_2,\cdots,s_k$
of positive integers a nonempty closed subest 
\[
  E(s_1,s_2,\cdots,s_k)
\]
of $X$, whose diameter does not exceed $2^{-k-2}$, subject to the conditions
\[
  X=\bigcup_{j\in\mathcal{P}} E(j) = E(1)\cup E(2)\cup\cdots,
\]
and
\begin{align*}
  E(s_1,s_2,\cdots,s_k) 
    &= \bigcup_{j\in\mathcal{P}} E(s_1,s_2,\cdots,s_k,j) \notag \\
    &= E(s_1,s_2,\cdots,s_k,1) \cup E(s_1,s_2,\cdots,s_k,2) \cup \cdots. \notag
\end{align*}
Thus for each infinite sequence $n=(n_1,n_2,\cdots)\in\mathcal{N}$ there
corresponds a sequence of nonempty closed sets
\[
  E(n_1)\supset E(n_1,n_2) \supset E(n_1,n_2,n_3) \supset \cdots
\]
whose diameters approach $0$, and whose intersection consists of exactly one
point of the complete metric space $X$ (Cantor Intersection Theorem 
\ref{T:cantor_int}), which we call $g(n)$.

It is easy to see that $\img(g)=X$ since for any $x\in X$ we can find a 
$n=(n_1,n_2,\cdots)\in\mathcal{N}$ such that $x\in E(n_1)$, $x\in E(n_1,n_2)$,
etc. Next we verify that $g:\mathcal{N}\to X$ is locally Lipschitzian. Note that
given the distance function between two distinct points $m,n\in \mathcal{N}$
given in Eq. \ref{E:dist_seq}, if $dist(m,n)<1/4$ there exists 
$k\in \mathcal{P}$ such that 
\[
  2^{-k-2}\leq dist(m,n) < 2^{-k-1}, 
    \qquad \text{hence $m_i=n_i$ for $i\leq k$},
\]
thus
\[
  dist[g(m),g(n)]\leq diam E(m_1,m_2,\cdots,m_k)\leq 2^{-k-2},
\]
and 
\[
  \frac{dist[g(m),g(n)]}{dist(m,n)} \leq 1.
\]
\end{proof}

%%%%%%%%%%%%%%%%%%%%%%%%
%%   interior
%%%%%%%%%%%%%%%%%%%%%%%%
\begin{definition}
\footnote{Engelking, General Topology, 1.1, p.14}
For every set $A$ in a topological space $(X,\mathcal{O})$ we define the 
\textbf{interior} 
\index{interior} 
$Int(A)$ of $A$ as the largest open set contained in $A$. 
Hence a set is open iff it is equal to its interior.
\end{definition}

%%%%%%%%%%%%%%%%%%%%%%
\begin{theorem} \label{T:interior_closure}
\footnote{Engelking, General Topology, Theorem 1.1.5, p.15}
For every $A\subset X$ we have $Int(A)= X\setminus \overline{X\setminus A}$.
\end{theorem}
\begin{proof}
Since $Int(A)\subset A$, we have $X\setminus A\subset X\setminus Int(A)$. Now
$X\setminus Int(A)$ is a closed set because $Int(A)$ is open, thus
\[
	\overline{X\setminus A} \subset X\setminus Int(A),
\]
hence $Int(A)\subset X\setminus \overline{X\setminus A}$.

Next note that 
\[
	A=X\setminus (X\setminus A) \supset X\setminus \overline{X\setminus A},
\]
and because $X\setminus \overline{X\setminus A}$ is open we thus have
\[
	Int(A)\supset X\setminus \overline{X\setminus A}.
\]
Thus we get $Int(A)= X\setminus \overline{X\setminus A}$.
\end{proof}

%%%%%%%%%%%%%%%%%%%%%%%%
\begin{proposition} \label{P:nodense}
Given a topological space $X$, a subset $A\subset X$ is nowhere dense iff
$Int(\overline{A})=\emptyset$, i.e. the interior of the closure of $A$ is 
empty. 
\end{proposition}
\begin{proof}
Straightforward from Defintion \ref{D:dense} and Theorem \ref{T:interior_closure}.
\end{proof}

%%%%%%%%%%%%%%%%%%%%%%%%
\begin{corollary} \label{C:clos}
\footnote{Engelking, General Topology, Corollary 1.1.2, p.14}
Given a topological space $X$, a subset $A\subset X$, if $U$ is an open set and
$U\cap A=\emptyset$, then also $U\cap \overline{A}=\emptyset$.
\end{corollary}
%%%%%%%%%%%%%%%%%%%
\begin{proof}
Suppose that there exists an $x\in U\cap\overline{A}$. Then $U$ is a
neighborhood of $x$ and by Proposition \ref{P:clos} $U\cap A\neq \emptyset$,
this contradicts the assumption, so that we have $U\cap \overline{A}=\emptyset$.
\end{proof}

%%%%%%%%%%%%%%%%%%%%%%%%
\begin{proposition} \label{P:nondense}
\footnote{Yeh, Real Analysis, 2ed, Proposition 15.45, p.346}
Let $(X,d)$ be a metric space and let $E$ be a subset of $X$.
\begin{enumerate}
  \item[(a)] $E$ is nowhere dense iff every non-empty open set $G$ in $X$
		contains a non-empty open set $G_0$ such that $G_0\cap E=\emptyset$.
  \item[(b)] $E$ is nowhere dense iff every open ball $B$ in $X$
		contains an open ball $B_0$ such that $B_0\cap E=\emptyset$.
\end{enumerate}
\end{proposition}
%%%%%%%%%%%%%%%%%%%%%%%%
\begin{proof}
First we verify (a). If $E$ is nowhere dense, then by Proposition 
\ref{P:nodense} $Int(\overline{E})=\emptyset$, hence for every nonempty open 
set $G$ we have $G\nsubset \overline{E}$, thus 
$G\setminus \overline{E}\neq\emptyset$. Now
$G\setminus \overline{E}$ is an open set contained in $G$, and 
$(G\setminus \overline{E})\cap E=\emptyset$. 

Conversely, if for every nonempty open set $G$ there exists a nonempty open set
$G_0\subset G$ such that $G_0\cap E=\emptyset$, then by Corollary \ref{C:clos}
we have $G_0\cap \overline{E}=\emptyset$, hence $G\nsubset\overline{E}$, and we
can conclude that $Int(\overline{E})=\emptyset$, i.e. $E$ is nowhere dense.

(b) follows from (a) by the fact that an open ball is an open set in the
topology generated by the metric and every non-empty open set contains an open
ball.
\end{proof}

%%%%%%%%%%%%%%%%%%%%%%%%%%%%%%
%%  Baire category theorem
%%%%%%%%%%%%%%%%%%%%%%%%%%%%%%
\begin{theorem}[Baire Category Theorem] \label{T:baire}
\footnote{Yeh, Real Analysis, 2ed, Theorem 15.47, pp.346-347}
Let $(X,d)$ be a complete metric space, then $X$ can not be represented by a
countable union of nowhere dense sets.
\end{theorem}
%%%%%%%%%%%%%%%%%%%%%%%%
\begin{proof}
We prove the theorem by showing that for any countable collection of nowhere
dense sets $\{E_n:n\in N\}$ we have $\bigcup_{n\in N} E_n\neq X$, i.e. there exists
$x\in X$ such that $x\notin E_n$ for every $n\in N$.

Let us write $B(x,r)=\{y\in X:d(x,y)<r\}$ and $C(x,r)=\{y\in X:d(x,y)\le r\}$
for an open ball and a closed ball in $(X,d)$.

Let $B(x_0,r_0)$ be an arbitrary open ball in $(X,d)$. Then by Proposition
\ref{P:nondense} $B(x_0,r_0)$ contains an open ball $B(x_1,r_1)$ such that
$B(x_1,r_1)\cap E_1=\emptyset$. We may choose $r_1<1$.

Consider open set $B(x_1,r_1/2)\subset C(x_1,r_1/2)\subset B(x_1,r_1)$. Since
$E_2$ is nowhere dense, again by Proposition \ref{P:nondense} $B(x_1,r_1/2)$
contains an open ball $B(x_2,r_2)$ such that $B(x_2,r_2)\cap E_2=\emptyset$. We
may choose $r_2<\frac{1}{2}$.

Consider open set $B(x_2,r_2/2)\subset C(x_2,r_2/2)\subset B(x_2,r_2)$. Since
$E_3$ is nowhere dense, again by Proposition \ref{P:nondense} $B(x_2,r_2/2)$
contains an open ball $B(x_3,r_3)$ such that $B(x_3,r_3)\cap E_3=\emptyset$. We
may choose $r_3<\frac{1}{3}$.

Thus continuing indefinitely, we have sequences $(B(x_n,r_n):n\in N)$,
$(C(x_n,r_n/2):n\in N)$, $(B(x_n,r_n/2):n\in N)$, such that 
$B(x_n,r_n)\cap E_n=\emptyset$ and $r_n<\frac{1}{n}$ for every $n\in N$ and
\begin{align*}
	        & B(x_1,r_1) \supset C(x_1,r_1/2) \supset B(x_1,r_1/2)  \\
	\supset & B(x_2,r_2) \supset C(x_2,r_2/2) \supset B(x_2,r_2/2)  \\
	\supset & B(x_3,r_3) \supset C(x_3,r_3/2) \supset B(x_3,r_3/2) \cdots  
\end{align*}
Now $C(x_1,r_1/2) \supset C(x_2,r_2/2) \supset C(x_3,r_3/2)\supset\cdots$ so
that $(C(x_n,r_n/2):n\in N$ is a decreasing sequence of closed sets. Moreover
\[
	\lim_{n\to\infty} diam(C(x_n,r_n/2))
	  = \lim_{n\to\infty} r_n
	  \le\lim_{n\to\infty}\frac{1}{n} = 0.
\]
Since $(X,d)$ is a complete metric space,
this implies that there exists $x\in X$ such that
$\bigcap_{n\in N} C(x_n,r_n/2)=\{x\}$ by Cantor Intersection Theorem
\ref{T:cantor_int}. Then since $B(x_n,r_n)\cap E_n=\emptyset$ for every 
$n\in N$, we have $x\notin E_n$ for every $n\in N$. Therefore
$x\notin\bigcup_{n\in N} E_n$ and thus $\bigcup_{n\in N} E_n\neq X$. 
\end{proof}

%%%%%%%%%%%%%%%%%%%%%%%%%%%%%%
\begin{definition}
\footnote{Engelking, General Topology, p.24}
A point $x$ in a topological space $X$ is called an 
\textbf{accumulation point} \index{accumulation point}
of a set $A\subset X$ if $x\in \overline{A\setminus \{x\}}$; 
the set of accumulation points of $A$ is called the 
\textbf{derived set} \index{derived set}
of $A$ and is denoted $A^d$. 
Points from $A\setminus A^d$ are called 
\textbf{isolated points} \index{isolated points}
of the set $A$. 
A set $A$ is said to be 
\textbf{dense in itself} \index{dense in itself} 
\footnote{Kuratowski, Topology, vol.1, \S 9, V, p.77.}
if it contains no isolated points, i.e. $A\subset A^d$.
\end{definition}

%%%%%%%%%%%%%%%%%%%%%%%%%%%%%%
\begin{proposition} \label{P:isolat}
\footnote{Engelking, General Topology, p.24}
A point $x$ is an isolated point of the topological space $X$ iff the one-point
set $\{x\}$ is open.
\end{proposition}
%%%%%%%%%%%%%%%%%%%%%
\begin{proof}
Set $\{x\}$ is open iff $X\setminus\{x\}$ is closed, and iff 
$X\setminus\{x\} =\overline{ X\setminus\{x\} }$, and iff 
$\{x\}=X\setminus \overline{ X\setminus\{x\} }$, i.e. 
$x\notin\overline{ X\setminus\{x\} }$.
\end{proof}

%%%%%%%%%%%%%%%%%%%%%%%%%%%%%%%%%%%%%%%%%
%%   T1-space
\begin{definition} 
A topological space $X$ is called a 
\textbf{$T_1$-space} \index{$T_1$-space}
\footnote{Engelking, General Topology, section 1.5, p.36}
if for every pair of distinct points $x_1,x_2\in X$ there exists an open set
$U\subset X$ such that $X_1\in U$ and $x_2\notin U$.
\end{definition} 

%%%%%%%%%%%%%%%%%%%%%%%%%%%%%%%%%%%%%
\begin{proposition} \label{P:T1}
$X$ is a $T_1$-space iff for every $x\in X$ the set $\{x\}$ is closed.
\footnote{Engelking, General Topology, section 1.5, p.37}
\end{proposition}
%%%%%%%%%%%%%%%%%%%%%
\begin{proof}
If $X$ is $T_1$, then for every $x\in X$ the set
\[
  X\setminus\{x\} 
  = \bigcup_{y\in X} \{ U_y\in \mathcal{O}: y\neq x, y\in U_y, x\notin U_y \}
\]
is open, hence the set $\{x\}$ is closed.
Conversely, if for every $x\in X$ the set $\{x\}$ is closed, then
$X\setminus\{x\}$ is open, and for every $y\in X$ and $y\neq x$, there exists an
open neighborhood $U_y$ such that $y\in U_y\subset X\setminus\{x\}$, hence
$x\notin U_y$, thus $X$ is a $T_1$-space.
\end{proof}

%%%%%%%%%%%%%%%%%%%%%%%%%%%%%%%%%%%%%
\begin{corollary} \label{C:metric_T1}
Any metric space $(X,d)$ is a $T_1$-space, i.e. for any $x\in X$, the set 
$\{x\}$ is closed.
\end{corollary}
%%%%%%%%%%%%%%%%%%%%%
\begin{proof}
For every $x\in X$, and for any element $y\in X\setminus \{x\}$ we can find an 
open ball $B(y,r)$ such that $B(y,r)\subset X\setminus \{x\}$, e.g. by choosing
$r=d(x,y)/2$. Hence $X\setminus \{x\}$ is an open set, i.e. $\{x\}$ is closed. 
Thus by Proposition \ref{P:T1} we get that $X$ is a $T_1$-space.
\end{proof}

%%%%%%%%%%%%%%%%%%%%%%%%%%%%%%%%%%%%%
\begin{corollary} \label{C:metric_T2}
Any metric space $(X,d)$ is a $T_2$-space (Hausdorff space).
\end{corollary}
%%%%%%%%%%%%%%%%%%%%%
\begin{proof}
For any two points $x,y\in X$, $x\neq y$, select a positive number $r<d(x,y)/2$
and two open balls $B(x,r)$ and $B(y,r)$, then $B(x,r)\cap B(y,r)=\emptyset$,
hence by Definition \ref{D:haus} we have that $X$ is Hausdorff.
\end{proof}

%%%%%%%%%%%%%%%%%%%%%%%%%%%%%%%%%%%%%%%%%
\begin{lemma} [Properties of derived set] \label{L:der_set}
We have the following properties of derived set of subsets of
topological space $X$:
\footnote{Kuratowski, Topology, vol.1, \S 9, III, pp.76-77.}
\begin{enumerate}
  \item[(1)] $A^d\subset \overline{A}$.
  \item[(2)] $\overline{A} = A \cup A^d$.
  \item[(3)] $(A\cup B)^d = A^d \cup B^d$.
  \item[(4)] If $A\subset B$, then $A^d \subset B^d$.
  \item[(5)] If $X$ is a $T_1$-space, then $\overline{A}^d = A^d$.
  \item[(6)] If $X$ is a $T_1$-space, then $A^{dd} \subset A^d$.
  \item[(7)] If $X$ is a $T_1$-space, then $A^d$ is closed, i.e.
             $\overline{A^d}=A^d$.
\end{enumerate}
\end{lemma}
%%%%%%%%%%%%%%%%
\begin{proof}
(1) From definition, for any $x\in A^d$ we have 
$x\in \overline{A\setminus\{x\}} \subset \overline{A}$,
hence $x\in\overline{A}$.

(2) From (1) we know that $A^d\subset \overline{A}$, thus we only need to verify
$\overline{A} \subset A \cup A^d$.
For any $x\in \overline{A}\setminus A$, $x\notin A$, hence 
$\overline{A\setminus\{x\}}=\overline{A}$, thus by definition
$x\in A^d$. 

(3) From definition, for any $x\in A^d$ we have 
$x\in \overline{A\setminus\{x\}} \subset \overline{(A\cup B)\setminus\{x\}}$,
i.e. $x\in (A\cup B)^d$.
Conversely, for any $x\in (A\cup B)^d$, 
$x\in \overline{(A\cup B)\setminus\{x\}} \subset
\overline{A\setminus\{x\}} \cup \overline{B\setminus\{x\}}$,
i.e. $x\in A^d \cup B^d$.

(4) From definition, for any $x\in A^d$ we have 
$x\in \overline{A\setminus\{x\}} \subset \overline{B\setminus\{x\}}$,
hence $x\in B^d$.

(5) From (4) we have $A^d\subset \overline{A}^d$. Hence we just need to verify
that $\overline{A}^d\subset A^d$. Now for any $x\in\overline{A}^d$, 
$x\in \overline{\overline{A}\setminus\{x\}}$,
since $X$ is $T_1$, by Proposition \ref{P:T1} the set $\{x\}$ is closed, hence
$\overline{A}\setminus\{x\}=\overline{A}\setminus\overline{\{x\}}
\subset \overline{A\setminus\{x\}}$,
hence
$\overline{\overline{A}\setminus\{x\}}\subset \overline{A\setminus\{x\}}$
by definition of closure.

(6) From (1) and (4), we get $A^{dd}\subset\overline{A}^d$. Now from (5) we know
that $\overline{A}^d=A^d$, hence $A^{dd}\subset A^d$.

(7) From (2) we get $\overline{A^d}=A^d\cup A^{dd}$, using (6) we then get
$\overline{A^d}=A^d$.
\end{proof}

%%%%%%%%%%%%%%%%%%%%%%%%%%%%%%
\begin{theorem} \label{dii_open}
If $X$ is a $T_1$-space and dense in itself, then all open subsets are dense in
itself.
\footnote{Kuratowski, Topology, vol.1, \S 9, V, Theorem 3, p.78.}
\end{theorem}
%%%%%%%%%%%%%%%%
\begin{proof}
From definition we have $X\subset X^d$. For any open subset $A\subset X$, then
$A=X\setminus B$ for some closed set $B$. From Lemma \ref{L:der_set}(1), we have
$B^d\subset\overline{B}=B$. And from Lemma \ref{L:der_set}(3) we get 
$X^d=A^d\cup B^d$. Hence $A=X\setminus B\subset X^d\setminus B^d\subset A^d$,
i.e. $A$ is dense in itself.
\end{proof}

%%%%%%%%%%%%%%%%%%%%%%%%%%%%%%
\begin{theorem} \label{dii_clo}
If $X$ is a $T_1$-space and $A\subset X$ is dense in itself, then $\overline{A}$
is also dense in itself
\footnote{Kuratowski, Topology, vol.1, \S 9, V, Theorem 1, p.77.}.
\end{theorem}
%%%%%%%%%%%%%%%%
\begin{proof}
Since $A$ is dense in itself, then $A\subset A^d$. From Lemma \ref{L:der_set}(2)
and (5) we have $\overline{A}=A\cup A^d=A^d=\overline{A}^d$, i.e. $\overline{A}$
is dense in itself.
\end{proof}

%%%%%%%%%%%%%%%%%%%%%%%%%%%%%%
\begin{corollary} \label{C:comp_uncount}
\footnote{Kolmogorov and Fomin, Introductory Real Analysis, 7.3, p.61}
A complete metric space $(X,d)$ without isolated points is uncountable.
\end{corollary}
%%%%%%%%%%%%%%%%
\begin{proof}
From Corollary \ref{C:metric_T1}, for any $x\in X$, the set $\{x\}$ is closed.
%It is not open because there does not exist an open ball contained in it. 
%And since there is no isolated points, $\{x\}$ can not be an open set, otherwise
%$\{x\}$ is a neighborhood of point $x\in X$ and 
%$\{x\}\bigcap X\setminus \{x\}=\emptyset$, i.e. $x$ is an isolated point, thus
%we have a contradiction.
And since there is no isolated points, by Proposition \ref{P:isolat} we have 
that $\{x\}$ is not an open set in $X$.
Hence the interior of its closure is the empty set, and by Proposition
\ref{P:nodense}
it is nowhere dense.
Thus by Baire Categrory Theorem \ref{T:baire} $X$ can not be the union of 
countable singletons, hence it is uncountable.
\end{proof}


%%%%%%%%%%%%%%%%%%%%%%%%%%%%%%%%%
\begin{proposition} \label{P:closed_complete}
  \footnote{Engelking, General Topology, Theorem 4.3.11, p.270}
If $(X,\rho)$ is a complete space, then for every closed subset $M$ of $X$ the
space $(M,\rho)$ is complete.
\end{proposition}
\begin{proof}
For any Cauchy sequence $(x_n)_n$ in $M$, it is also a Cauchy sequence in $X$.
Since $X$ is complete, $(x_n)_n$ converges to a point $x\in X$. Now because
$M$ is a closed set, it contains all its limit points, hence $x\in M$, thus
$M$ is also complete.
\end{proof}

%%%%%%%%%%%%%%%%%%%%%%%%%%%%%%
\begin{lemma} \label{L:binary_seq}
The set of all infinite binary sequences $Z=\{ z_n \}_{n=1,2,\cdots}$, where
\[
  z_n = (z_{n0}, z_{n1}, \cdots),  \forall i\in N, z_{ni}\in \{0,1\},
\]
is uncountable.
\end{lemma}
%%%%%%%%%%%%%%
\begin{proof}
Define a bijection $f:N\to Z$ where $\forall n\in N$,
\[
  f(n) = z_n,
\]
define a new sequence $q=(q_1, q_2, \cdots)$ where
$q_n=1-z_{nn}$. It is easy to see that the sequence $q$ is not in set $Z$, thus
by defintion $Z$ is uncountable.
\end{proof}

%%%%%%%%%%%%%%%%%%%%%%%%%%%%%%
\begin{lemma} \label{L:dii_uncount}
\footnote{Gelbaum, Problems in Real and Complex Analysis,  Problem 2.24, p.19, 
solution p.160}
If $X$ is a complete, nonempty metric space without isolated points, then any
nonempty open subset $U$ of $X$ is uncountable.
\end{lemma}
%%%%%%%%%%%%%%
\begin{proof}
Define a Cantor scheme $(U_s)_{s\in 2^{<N}}$ on $U$ by induction on $|s|$ such
that:
\footnote{cf. Tserunyan, Introduction to Descriptive Set Theory (version Nov.
  12, 2019), Theorem 4.2, p.15; 
  Srivastava, A Course on Borel Sets, Proposition 2.6.1, pp.75-76}
\begin{enumerate}
  \item[(i)] $U_s$ is nonempty open;
  \item[(ii)] $\diam(U_s)<1/|s|$;
  \item[(iii)] $\overline{U_{s^\wedge i}}\subset U_s$, for $i\in \{0,1\}$, 
               and $\overline{U_{s^\wedge 0}}\cap 
                    \overline{U_{s^\wedge 1}}=\emptyset$.
\end{enumerate}
We do this as follows: let $U_{\emptyset}=U$ and assume $U_s$ is defined. 
Since $U_s$ is open and nonempty, then $U_s$ is dense in itself (without
isolated points) by Theorem \ref{dii_open} and Corollary \ref{C:metric_T1},
and by Propsition \ref{P:isolat} we get that $U_s$ has at least two points 
$x\neq y$. Now by Corollary \ref{C:metric_T2} 
we know that $X$ is also a Hausdorff space (i.e. $T_2$), hence we can find
two distinct open neighborhoods $U_{s^\wedge 0}\ni x$ and 
$U_{s^\wedge 1}\ni y$ with small enough diameters so that the conditions (ii)
and (iii) are satisfied. This finishes the construction.

Now for any infinite binary sequence $s=(s_1,s_2,\cdots)$, define a function $g$
such that
\[
  g(s) = \bigcap_{n\in N} \overline{U_{s|n}},
\]
where $s|n=(s_1,s_2,\cdots,s_n)$.
And by Cantor Intersection Theorem \ref{T:cantor_int} we know that $g(s)$ is a
singleton in $X$. Hence we conclude that $g$ is a bijection. Now by Lemma
\ref{L:binary_seq} the set of all infinite binary sequences is uncountable,
hence $U$ is uncountable too.


% there exists an open ball $B(x_0,r_0)\subset U$,
% and $B(x_0,r_0)$ has at least two elements (since every singleton in $X$ is not 
% open by Proposition \ref{P:isolat}). 
% Now by Corollary \ref{C:metric_T2} we know 
% that $X$ is also a Hausdorff space (i.e. $T_2$), 
% hence there exists two distinct 
% points $x_{00},x_{01}\in B(x_0,r_0)$ and a positive number $r_1$ such that
% there exits two (nonempty) open balls
% $B(x_{00},r_1)\cap B(x_{01},r_1)=\emptyset$ and 
% $B(x_{00},r_1)\cup B(x_{01},r_1)\subset B(x_0,r_0)$.

% By induction, for each dyadic rational number $\sum_{k=1}^K \epsilon_k 2^{-k}$,
% there can be defined a point $x_{\epsilon_1,\dots,\epsilon_k}$ and for each $K$
% a positive $r_K$ so that $r_K\to 0$,
% \[
%   B(x_{\epsilon_1,\epsilon_{K-1},0},r_K)
%     \cap B(x_{\epsilon_1,\epsilon_{K-1},1},r_K) =\emptyset,
% \]
% and
% \[
%   B(x_{\epsilon_1,\epsilon_{K-1},0},r_K)
%     \cup B(x_{\epsilon_1,\epsilon_{K-1},1},r_K) 
%     \subset B(x_{\epsilon_1,\epsilon_{K-1}},r_{K-1}).
% \]
% The cardinality of the closure of the set of all points $x_{\dots}$ is
% $\mathfrak{c}$, hence the set $U$ is uncountable.
% 
% TODO: this does not seem to be completely rigorous

\end{proof}

%%%%%%%%%%%%%%%%%%%%%%%%%%%%%%%%%%%%%%%%%%%%%%%%%%%%%%%%%%%%%
%% Simon says: already proved in Corollary \label{C:g_delta}
% \begin{lemma}
% \footnote{Tserunyan, Introduction to Descriptive Set Theory (version Nov. 12,
% 2019), Lemma 1.8, p.5}
% If $(X,d)$ is a metric space, then any nonempty open set $U$ is a $F_{\sigma}$
% set, i.e. countable union of closed sets.
% \end{lemma}
% %%%%%%%%%%%%
% \begin{proof}
% For any $n\in\mathbb{N}$ let
% \[
%   F_n = \{ x\in U: d(x,X\setminus U)\ge 2^{-n} \};
% \]
% Now $F_n$ is closed, and $U=\bigcup_{n\in\Bbb N} F_n$. Thus, $U$ is a
% $F_{\sigma}$ set.
% \end{proof}

%%%%%%%%%%%%%%%%%%%%%%%%%%%%%%
\begin{definition}
A topological space $X$ is \textbf{completely metrizable} if it admits a compatiable
metric $d$ such that $(X,d)$ is complete. A separble completely metrizable space
is called \textbf{Polish}.
\index{Polish space}
\end{definition}

%%%%%%%%%%%%%%%%%%%%%%%%%%%%%%
\begin{definition}
A \textbf{Lusin scheme}
\footnote{
  Tserunyan, Introduction to Descriptive Set Theory (version Nov. 12, 2019), 
  Definition 2.11, p.9;
  Kechris, Classical Descriptive Set Theory, Definition 7.5, p.36;
}
on a set $X$ is a family 
$(A_s)_{s\in \mathcal{P}^{<\mathcal{P}}}$ of subsets of $X$ such that
\index{Lusin scheme}
\begin{enumerate}
  \item[(i)] $A_{s^\wedge i}\bigcap A_{s^\wedge j}=\emptyset$, 
             $\forall s\in \mathcal{P}^{<\mathcal{P}}, i\neq j$;
  \item[(ii)] $A_{s^\wedge i}\subset A_s$, 
              $\forall s\in \mathcal{P}^{<\mathcal{P}}, i\in \mathcal{P}$.

If $(X,d)$ is a metric space and we additionally have

  \item[(iii)] $\lim_{n\to\infty} \diam(A_{x|n})=0$, $\forall x\in\mathcal{N}$,
\end{enumerate}
we say that $(A_s)_{s\in \mathcal{P}^{<\mathcal{P}}}$ has \textbf{vanishing
diameter}. In this case, we let
\[
  D=\left\{
      x\in\mathcal{N}: \bigcap_{n\in\mathcal{P}} A_{x|n}\neq\emptyset
    \right\}
\]
and define $f:D\to X$ by $\{f(x)\}=\bigcap_n A_{x|n}$. This $f$ is called the 
\textbf{associated map}.
\index{associated map}
\end{definition}

%%%%%%%%%%%%%%%%%%%%%%%%%%%%%%
\begin{proposition}  \label{P:Lusin_sche}
\footnote{
  Tserunyan, Introduction to Descriptive Set Theory (version Nov. 12, 2019), 
  Proposition 2.12, p.10.
}
Let $(A_s)_{s\in \mathcal{P}^{<\mathcal{P}}}$ be a Lusin scheme on a metric
space $(X,d)$ that has vanishing diameter and let $f:D\to X$ be the associated
map. 
\begin{enumerate}
  \item[(a)] $f$ is injective and continuous.
  \item[(b)] If $A_{\emptyset}=X$ and $A_s=\bigcup_i A_{s\wedge i}$
             for each $s\in \mathcal{P}^{<\mathcal{P}}$, then $f$ is surjective.
  \item[(c)] If $(X,d)$ is complete and $\overline{A_{s\wedge i}}\subset A_s$ 
             for each $s\in \mathcal{P}^{<\mathcal{P}}, i\in\mathcal{P}$, then 
             for all $x\in \mathcal{N}$
\[
  x\notin D \iff \exists n\in\mathcal{P}, A_{x|n}=\emptyset.
\]
Thus $D$ is closed, and moreover, if each $A_s$ is nonempty, then
$D=\mathcal{N}$.
\end{enumerate}
\end{proposition}
%%%%%%%%%%%%%%%%%%
\begin{proof}
In part (a), to verify the injectivity (or $f$ is one-to-one) we note that for 
any two elements of $x$ and $p$ in $\mathcal{N}$, $x\neq p$, there exists
one positive integer $n$ such that $x_n\neq p_n$. 
By (i) of the definition of Lusin scheme, we have $A_{x|n}\neq A_{p|n}$, hence
$f(x)\neq f(p)$.

To verify the continuouness, note that since the Lusin scheme has vanishing
diameter, then for any $x\in\mathcal{N}$ and any $\epsilon>0$, there exist a 
positive integer $n$ such that $A_{x|n}<\epsilon$. It is easy to see that for
any $p\in\mathcal{N}$ which has the property $p|n=x|n$ (i.e. 
$p_1=x_1, p_2=x_2, \dots, p_n=x_n$), then $d(f(p),f(x))<\epsilon$.
We may use for $\mathcal{N}$ the metric defined in Eq. \ref{E:dist_seq}:
\[
	d_{\mathcal{N}}(m,n) 
	  = \sum_{i=1}^{\infty} 2^{-i} \frac{|m_i-n_i|}{1+|m_i-n_i|}.
\]
and note that $d_{\mathcal{N}}(p,x)<1-\frac{1}{2^n}$. Thus we conclude that $f$
is continuous.

To verify (b), note that for any $y\in X$, there is some $x_1$ such that 
$y\in A_{x_1}$, then there is some $x_2$ such that 
$y\in A_{x_1,x_2}$, etc. Thus there exists one $x\in \mathcal{N}$ such that
$f(x)=y$, i.e. $f$ is surjective.

To verify (c), first we note that
\[
  \bigcap_n A_{x|n} = \bigcap_n \overline{A_{x|n}}
\]
since $\overline{A_{x|n+1}}\subset A_{x|n}$. If $A_{x|n}$ is nonempty for all
$n\in\mathcal{P}$, then by the Cantor Intersection Theorem \ref{T:cantor_int}, 
$\bigcap_n A_{x|n}$ is nonempty, hence $x\in D$. 
Thus it is easy to see that
\[
  x\in D^c \iff \exists n\in\mathcal{P}, A_{x|n}=\emptyset.
\]

To verify that $D$ is closed, note that for any $x\in D^c$, there exists some
$n$ such that $A_{x|n}=\emptyset$. Again we use 
the metric defined in Eq. \ref{E:dist_seq} for $\mathcal{N}$, and it is easy to
see that open ball $B_{1-\frac{1}{2^n}}(x)\subset D^c$, hence $D^c$ is open, we thus 
conclude that $D$ is closed.
\end{proof}

%%%%%%%%%%%%%%%%%%%%%%%%%%%%%%
\begin{lemma} \label{L:sep_fsigma}
Let $X$ be a separable metric space, if $A$ is an $F_{\sigma}$-set in $X$ and
$\epsilon>0$, then $A$ is the union of a sequence $\{B_n\}$ of pairwise disjoint
$F_{\sigma}$-sets, each having diameter no more than $\epsilon$, and such that
$\overline{B_n}\subset A$ for each $n$.
\end{lemma}
%%%%%%%%%%%%%%%%
\begin{proof}
We first claim that $A$ is the union of a sequence of nonempty closed sets, each
having diameter no more than $\epsilon$. To see this, let 
$A=\bigcup_{n\in\mathcal{P}} F_n$, where each $F_n$ is closed. Since $X$ is
separable, by definition there is a countable dense subset 
$\{x_1,x_2,\dots\}$ of $X$. Let $C_m$ be the close ball of radius $\epsilon/2$
around $x_m$. Hence $X=\bigcup_{m\in\mathcal{P}} C_m$, and
\[
  A = \bigcup_{n\in\mathcal{P}} F_n
    = \bigcup_{n\in\mathcal{P}} ( F_n \bigcap (\bigcup_{m\in\mathcal{P}} C_m) )
    = \bigcup_{n\in\mathcal{P}} \bigcup_{m\in\mathcal{P}} D_{n,m},
\]
where $D_{n,m}=F_n\bigcap C_m$. It is easy to see that each $D_{n,m}$ is closed 
and has diameter no more than $\epsilon$.

Next we write $A=\bigcup_{n\in\mathcal{P}} C_n$, where $C_n$ is closed with
diameter no more than $\epsilon$. Now let $B_1=C_1$ and 
$B_{n+1}=C_{n+1}\setminus\bigcup_{m=1}^n C_m$ for $n\geq 1$. Since 
$B_n\subset C_n$ we have $\diam B_n\leq\epsilon$ and 
$\overline{B_n}\subset C_n\subset A$. Now observe that the set 
$D=\bigcup_{m=1}^n C_m$ is closed so its complement, being open, is an
$F_{\sigma}$. Therefore each $B_n$ is also an $F_{\sigma}$. Furthermore the sets
$B_n$ are pairwise disjoint and $A=\bigcup_{n\in\mathcal{P}} B_n$.
\end{proof}

%%%%%%%%%%%%%%%%%%%%%%%%%%%%%%
\begin{theorem} \label{T:pol_bij}
Let $X$ be a Polish space. Then there is a closed set $D\subset \mathcal{N}$ and a
continuous bijection $f: D\to X$.
\footnote{
  Tserunyan, Introduction to Descriptive Set Theory (version Nov. 12, 2019), 
  Theorem 5.7, pp.18-19;
  Kechris, Classical Descriptive Set Theory, Theorem 7.9, pp.38-39;
  Aliprantis and Border, Infinite Dimensional Analysis: A 
  Hitchhiker's Guide, Theorem 3.66, pp.104-105;
}
\end{theorem}
%%%%%%%%%%%%%%%
\begin{proof}
We fix a complete compatible metrix $d$ on $X$ and define a Lusin scheme 
$(F_s)_{s\in \mathcal{P}^{<\mathcal{P}}}$ on $X$ by induction such that:
% \begin{enumerate}
%   \item[(i)] $F_{\emptyset}=X$;
%   \item[(ii)] $F_s$ is an $F_{\sigma}$ set;
%   \item[(iii)] $F_s=\bigcup_{i\in\mathcal{P}} F_{s^\wedge i} = \bigcup_{i\in\mathcal{P}} \overline{F_{s^\wedge i}}$;
%   \item[(iv)] $\diam(F_s)<2^{-|s|}$;
% \end{enumerate}
\begin{enumerate}
   \item[(i)] $F_{\emptyset}=X$ and 
              $F_s=\bigcup_{i\in\mathcal{P}} F_{s^\wedge i}$;
   \item[(ii)] $\overline{F_{s^\wedge i}}\subset F_s$, for $i\in\mathcal{P}$;
   \item[(iii)] $F_s$ is an $F_{\sigma}$ set;
   \item[(iv)] $\diam(F_s)<2^{-|s|}$.
\end{enumerate}
The existence of this Lusin scheme is guaranteed by Lemma \ref{L:sep_fsigma},
and the theorem follows from Proposition \ref{P:Lusin_sche}.
\end{proof}
%%%%%%%%%%%%%%%%%%%%


For any $z=(z_1,z_2,\cdots)\in \mathcal{N}$, we denote 
$z|n=(z_1,z_2,\cdots,z_n)$. 

%%%%%%%%%%%%%%%%%%%%%%%%%%%%%
\begin{lemma} \label{L:lusin}
\footnote{Kuratowski and Mostowski, Set Theory with an Introduction to
	Descriptive Set Theory, North Holland(1976), pp.409-410; 
    Kechris, Classical Descriptive Set Theory, Exercise 25.12, pp.199-200.
}
Let $\{U_z\}$ be a Lusin scheme on $X$, then 
\[
	\bigcup_{z\in\mathcal{N}} \bigcap_{k\in\mathcal{P}} U_{z|k}
	  =\bigcap_{k\in\mathcal{P}} \bigcup_{z\in\mathcal{N}} U_{z|k}.
\]
\end{lemma}
%%%%%%%%%%%%%%%%%%%%
\begin{proof}
Note first that for any $z\in\mathcal{N}$ we have
\[
	\bigcap_{k\in\mathcal{P}} U_{z|k}
	  \subset\bigcap_{k\in\mathcal{P}} \bigcup_{z\in\mathcal{N}} U_{z|k}.
\]
hence
\[
	\bigcup_{z\in\mathcal{N}} \bigcap_{k\in\mathcal{P}} U_{z|k}
	  \subset\bigcap_{k\in\mathcal{P}} \bigcup_{z\in\mathcal{N}} U_{z|k}.
\]

To verify the other direction, for any 
$p\in\bigcap_{k\in\mathcal{P}} \bigcup_{z\in\mathcal{N}} U_{z|k}$, then
for any $k\in\mathcal{P}$, we have $p\in\bigcup_{z\in\mathcal{N}} U_{z|k}$.
Now by the defintion of Lusin scheme, if $z|k\neq y|k$ then 
$U_{z|k}\cap U_{y|k}=\emptyset$, there exists
an unique $z\in\mathcal{N}$ such that $p\in U_{z|k}$. Again by the definition
of Lusin scheme we have
\[
	U_{z|1} \supset U_{z|2} \supset \cdots,
\]
we conclude that $p\in\bigcap_k U_{z|k}$, hence
\[
	p\in\bigcup_{z\in\mathcal{N}} \bigcap_{k\in\mathcal{P}} U_{z|k},
\]
i.e.
\[
	\bigcap_{k\in\mathcal{P}} \bigcup_{z\in\mathcal{N}} U_{z|k}
	  \subset\bigcup_{z\in\mathcal{N}} \bigcap_{k\in\mathcal{P}} U_{z|k}.
\]
\end{proof}
%%%%%%%%%%%%%%%%%%%%


%%%%%%%%%%%%%%%%%%%%%%%%%%%%%%
\begin{theorem}
If $X$ is a complete, nonempty metric space without isolated points, then 
$X$ has a Borel subset $\Gamma$ which is homeomorphic to 
$\mathcal{N}$. 
\footnote{Federer, 2.2.9, p.65. cf. Bruckner et.al., Real Analysis, 2ed.,
	Theorem 11.4, p.438, which proves the case with perfect, complete, 
	separable metric space.}
% \begin{enumerate}
%   \item[(1)] $X$ has a Borel subset $\Gamma$ which is homeomorphic to 
%              $\mathcal{N}$, \footnote{Federer, 2.2.9, p.65}
%   \item[(2)] $\mathcal{N}$ has a closed subset $D$ which is homeomorphic to $X$.
% \end{enumerate}
\end{theorem}
%%%%%%%%%%%%
\begin{proof}
For each $k\in\mathcal{P}$ we construct disjoint nonempty open sets
% \footnote{\textdbend How could we prove that such sets always exist? 
% 	So far we know that any complete metric space without isolated points is
% 	uncountable. But is $U(s)$ nonempty and uncountable?
% }
$U(s)$ with closures $E(s)$ and diameters less than $2^{-k-2}$ corresponding to
all sequences $s\in\mathcal{P}^k$; in passing from $k$ to $k+1$ we require that 
\[
	U(s_1,\dots,s_k) \supset \bigcup_{j\in\mathcal{P}} E(s_1,\dots,s_k,j).
\]

% Using mathematical induction, 
% given nonempty open set $U(s_1,\dots,s_k)$, from
% Theorem \ref{dii_open} we get that it is dense in itself (without isolated
% points), and from Theorem \ref{dii_clo} we know that $E(s_1,\dots,s_k)$ is dense
% in itself too. Also by Proposition \ref{P:closed_complete} we conclude that
% $E(s_1,\dots,s_k)$ is a complete space. Hence by Corollary 
% \ref{C:comp_uncount} we have that $E(s_1,\dots,s_k)$ is uncountable.

Define a Lusin scheme 
%\index{Lusin scheme}
\footnote{cf. Aliprantis and Border, Infinite Dimensional Analysis: A 
  Hitchhiker's Guide, Theorem 3.66, pp.104-105;
  Kechris, Classical Descriptive Set Theory, Theorem 7.9, pp.38-39;
  Tserunyan, Introduction to Descriptive Set Theory (version Nov. 12, 2019), 
  Theorem 5.7, pp.18-19
}
$(U_s)_{s\in \mathcal{P}^{<\mathcal{P}}}$ on $X$ by 
induction such that:
\begin{enumerate}
  \item[(i)] $U_s$ is nonempty open;
  \item[(ii)] $\diam(U_s)<2^{-|s|-2}$;
  \item[(iii)] $\bigcup_{i\in\mathcal{P}} \overline{U_{s^\wedge i}}
                \subset U_s$;
  \item[(iv)] $U_{s^\wedge i}\bigcap U_{s^\wedge j}=\emptyset$, 
              $\forall i\neq j$.
\end{enumerate}
We do this as follows: let $U_{\emptyset}=X$ and assume $U_S$ is defined. Since
$U_s$ is nonempty and open, by Lemma \ref{L:dii_uncount} we know that $U_s$ is
uncountable. Select a point $x_1\in U_s$, there exists an open neighborhood 
$U_{s^\wedge 1}$ of $x_1$ such that $\diam(U_{s^\wedge 1})<2^{-|s|-3}$ and 
$\overline{U_{s^\wedge 1}}\subset U_s$. 
Now $U_s\setminus \overline{U_{s^\wedge 1}}$ is open and nonempty, similary we
can select a point $x_2\in U_s\setminus \overline{U_{s^\wedge 1}}$ and an open
neighborhood $U_{s^\wedge 2}$ of $x_2$ such that 
$\diam(U_{s^\wedge 2})<2^{-|s|-3}$ and 
$\overline{U_{s^\wedge 2}}\subset U_s\setminus\overline{U_{s^\wedge 1}}$. 
Continuing this, we will be able to construct a sequence of nonempty open sets
$(U_{s^\wedge n})_{n\in\mathcal{P}}$ satisfying all four conditions above
(i),(ii),(iii),(iv).

Thus for each infinite sequence $n=(n_1,n_2,\cdots)\in\mathcal{N}$ there
corresponds a sequence of nonempty closed sets
\[
  E(n_1)\supset E(n_1,n_2) \supset E(n_1,n_2,n_3) \supset \cdots
\]
whose diameters approach $0$, and whose intersection consists of exactly one
point of the complete metric space $X$ (Cantor Intersection Theorem 
\ref{T:cantor_int}), which we call $g(n)$.

Now we verify that $g$ maps $\mathcal{N}$ homeomorphically onto the Borel set
\[
	\Gamma = \bigcap_{k\in\mathcal{P}} \bigcup_{s\in\mathcal{P}^k} U(s).
\]
First note that for any $n=(n_1,n_2,\cdots)\in\mathcal{N}$ we have
\begin{align*}
	\cdots 
	&\subset U(n_1,n_2,\cdots,n_k) \subset E(n_1,n_2,\cdots,n_k)
  	\subset U(n_1,n_2,\cdots,n_{k-1})  \\
	&\subset E(n_1,n_2,\cdots,n_{k-1})
  	\subset \cdots \subset U(n_1) \subset E(n_1) \subset X,
\end{align*}
it is easy to see that 
\[
	\bigcap_{k=1}^{\infty} U(n_1,n_2,\cdots,n_k) 
	  \subset \bigcap_{k=1}^{\infty} E(n_1,n_2,\cdots,n_k),
\]
and
\[
	\bigcap_{k=1}^{\infty} E(n_1,n_2,\cdots,n_k)
	= \bigcap_{k=2}^{\infty} E(n_1,n_2,\cdots,n_k)
	\subset \bigcap_{k=1}^{\infty} U(n_1,n_2,\cdots,n_k),
\]
hence we have
\[
	\bigcap_{k=1}^{\infty} U(n_1,n_2,\cdots,n_k) 
	= \bigcap_{k=1}^{\infty} E(n_1,n_2,\cdots,n_k).
\]

% For any $z=(z_1,z_2,\cdots)\in \mathcal{N}$, we denote 
% $z|n=(z_1,z_2,\cdots,z_n)$. We shall verify that
% \footnote{cf. Kuratowski and Mostowski, Set Theory with an Introduction to
% Descriptive Set Theory, North Holland(1976), pp.409-410. We also adopt the 
% notation in Kuratowski's book.}
% \[
% 	\bigcup_{z\in\mathcal{N}} \bigcap_{k\in\mathcal{P}} U_{z|k}
% 	  =\bigcap_{k\in\mathcal{P}} \bigcup_{z\in\mathcal{N}} U_{z|k}.
% \]
% Note first that for any $z\in\mathcal{N}$ we have
% \[
% 	\bigcap_{k\in\mathcal{P}} U_{z|k}
% 	  \subset\bigcap_{k\in\mathcal{P}} \bigcup_{z\in\mathcal{N}} U_{z|k}.
% \]
% hence
% \[
% 	\bigcup_{z\in\mathcal{N}} \bigcap_{k\in\mathcal{P}} U_{z|k}
% 	  \subset\bigcap_{k\in\mathcal{P}} \bigcup_{z\in\mathcal{N}} U_{z|k}.
% \]
% To verify the other direction, for any 
% $p\in\bigcap_{k\in\mathcal{P}} \bigcup_{z\in\mathcal{N}} U_{z|k}$, then
% for any $k\in\mathcal{P}$, we have $p\in\bigcup_{z\in\mathcal{N}} U_{z|k}$.
% Now since if $z|k\neq y|k$ then $U_{z|k}\cap U_{y|k}=\emptyset$, there exists
% an unique $z\in\mathcal{N}$ such that $p\in U_{z|k}$. And because
% \[
% 	U_{z|1} \supset U_{z|2} \supset \cdots,
% \]
% we conclude that $p\in\bigcap_k U_{z|k}$, hence
% \[
% 	p\in\bigcup_{z\in\mathcal{N}} \bigcap_{k\in\mathcal{P}} U_{z|k},
% \]
% i.e.
% \[
% 	\bigcap_{k\in\mathcal{P}} \bigcup_{z\in\mathcal{N}} U_{z|k}
% 	  \subset\bigcup_{z\in\mathcal{N}} \bigcap_{k\in\mathcal{P}} U_{z|k}.
% \]

Since $\{U_z\}$ is a Lusin scheme on $X$, by Lemma \ref{L:lusin} we have
\[
	\bigcup_{z\in\mathcal{N}} \bigcap_{k\in\mathcal{P}} U_{z|k}
	  =\bigcap_{k\in\mathcal{P}} \bigcup_{z\in\mathcal{N}} U_{z|k}.
\]


To verify that $g:\mathcal{N}\to \Gamma$ where
\[
	\Gamma=\bigcap_{k\in\mathcal{P}} \bigcup_{z\in\mathcal{N}} U_{z|k},
\]
is homeomorphic we need to verify that $g$ is one-to-one and onto and
both $g$ and $g^{-1}$ are continuous. 
First note that it is easy to verify that $g$ is one-to-one and onto.
\footnote{Follow the same arguments that we used to prove
$\bigcup_{z\in\mathcal{N}} \bigcap_{k\in\mathcal{P}} U_{z|k}
		=\bigcap_{k\in\mathcal{P}} \bigcup_{z\in\mathcal{N}} U_{z|k}$.}
Next we verify that both $g$ and $g^{-1}$ are continuous.
Here for $\mathcal{N}$ we use the metric defined in Eq. \ref{E:dist_seq}:
\[
	d_{\mathcal{N}}(m,n) 
	  = \sum_{i=1}^{\infty} 2^{-i} \frac{|m_i-n_i|}{1+|m_i-n_i|}.
\]
Note that for any $p,x\in\mathcal{N}$, the first $k$ terms of $p$ and $x$ 
match, i.e. $x_1=p_1,x_2=p_2, \dots, x_k=p_k$
iff $g(x),g(p)\in U_{p|k}$. Note also that if the first $k$ terms of $p$ and $x$
match, then $d_{\mathcal{N}}(p,x) < 1-\frac{1}{2^k}$. Also if 
$g(x),g(p)\in U_{p|k}$, then $d_{\Gamma}(g(x),g(p)) < 2^{-k-2}$. And the
continuity of $g$ and $g^{-1}$ follows.
% First we verify $g$ is continuous. For any $\epsilon>0$, choose
% $k_{\epsilon}\in\mathcal{P}$ such that $2^{-k_{\epsilon}-2}<\epsilon$. 
% Then for any $p\in\mathcal{N}$ we know
% \[
% 	diam(U_{p|k_{\epsilon}}) < 2^{-k_{\epsilon}-2} < \epsilon,
% \]
% thus if we choose 
% \[
% 	\delta=1- \frac{1}{2^{k_{\epsilon}}} > 0,
% \]
% then for all $x\in\mathcal{N}$ and $d_{\mathcal{N}}(x,p)<\delta$ we have
% $x_1=p_1, x_2=p_2, \dots, x_{k_{\epsilon}}=p_{k_{\epsilon}}$, hence
% $g(x)\in U_{p|k_{\epsilon}}$ and $g(p)\in U_{p|k_{\epsilon}}$, thus
% \[
% 	d_{\Gamma}(g(x),g(p)) < \epsilon,
% \]
% i.e. $g$ is continuous.
% Next we verify that $g^{-1}$ is continuouse. For any $\epsilon>0$, we choose
% $k_{\epsilon}\in\mathcal{P}$ such that $1-\frac{1}{2^{k_{\epsilon}}} <\epsilon$,


% Next we verify that $g$ maps a closed subset $D$ of $\mathcal{N}$ 
% homeomorphically to $X$ 
% \footnote{cf. Aliprantis and Border, Infinite Dimensional Analysis: A 
%   Hitchhiker's Guide, Theorem 3.66, pp.104-105}
% where
% \[
% 	D = \{ 
%           n\in\mathcal{N} : U_{n|k}\neq\emptyset, \forall k\in\mathcal{P}
%         \}
% \]
% 
% First we verify that $D$ is closed in $X$. For any $n\in D^C$, there is some $k$
% for which $U_{n|k}=\emptyset$. Now if 
% $d_{\mathcal{N}}(n,m)\leq 1-\frac{1}{2^k}$, then $m|k=n|k$, hence 
% $U_{m|k}=U_{n|k}=\emptyset$, thus $m\in D^C$. This means that 
% $B_{1-\frac{1}{2^k}}(n)\subset D^C$, i.e. $D^C$ is open, thus $D$ is closed.
% 
% To verify that $g:D\to X$ is homeomorphic we need to verify that $g$ is 
% one-to-one and onto and both $g$ and $g^{-1}$ are continuous. 
% First note that $X=\cup_{n\in\mathcal{P}} U_n$, hence each point of $X$ belongs
% to some $U_{n_1}$, and inductively belongs to some $U_{n_1,n_2}, etc., 

\end{proof}

%%%%%%%%%%%%%%%%%%%%%%%%%%%%%%%%%%%%%%%%%%%%%
%%         Federer 2.2.10
%%%%%%%%%%%%%%%%%%%%%%%%%%%%%%%%%%%%%%%%%%%%%
\section{Suslin Sets}

%%%%%%%%%%%%%%%%%%%%%%%%%%%%%%%%%
\begin{definition} \label{D:suslin_set}
Consider a topological space $X$ and the projection
\[
	p: X\times \mathcal{N} \to X, \qquad 
	\text{$p(x,n) = x$ whenever $x\in X, n\in\mathcal{N}$}.
\]
By a \textbf{Suslin subset}
\index{Suslin subset}
\footnote{Federer 2.2.10, p.65}
of $X$ we mean the $p$ image of some closed subset of $X\times \mathcal{N}$.
\end{definition}

%%%%%%%%%%%%%%%%%%%%%%%%%%%%%%%%%
\begin{definition}
\footnote{Engelking, General Topology, 1.6, pp.49-50. The concept is a
  generalization of the concept of sequence.}
A \textbf{net} \index{net}
in a topological space $X$ is an arbitrary function from a non-empty directed
set to the space $X$. We will denote a net by 
$S=\{x_{\sigma}, \sigma\in\Sigma\}$, where $x_{\sigma}$ is the point of $X$
assigned to the element $\sigma$ of the directed set $\Sigma$.

A point $x$ is called a \textbf{limit of a net}
\index{limit of a net}
$S=\{x_{\sigma}, \sigma\in\Sigma\}$ if for every neighborhood $U$ of $x$ there
exists a $\sigma_0\in\Sigma$ such that $x_{\sigma}\in U$ for every 
$\sigma\succeq\sigma_0$; we say that the net $S$ converges to $x$. A net can
converge to many points; the set of all limits of the net
$S=\{x_{\sigma}, \sigma\in\Sigma\}$ is denoted by $\lim S$ or
$\lim_{\sigma\in\Sigma} x_{\sigma}$. 
\end{definition}

%%%%%%%%%%%%%%%%%%%%%%%%%%%%%%%%%
\begin{proposition} \label{P:haus_diag}
\footnote{cf. a more general version is given by Kuratowski, Topology, vol.1, 
	\S 16, IV, Theorem 2, p.153}
A topological space $X$ is Hausdorff if and only if the diagonal of $X\times X$
\[
	D= \{ (x,x)\in X\times X: x\in X \}
\]
is closed.
\end{proposition}
\begin{proof}
Suppose first that $D$ is closed in $X\times X$. Then $(X\times X)\setminus D$
is open, and for any $(x,y)\in (X\times X)\setminus D$, we have $x\neq y$ and
there exists an open set $B$ such that 
$(x,y)\in B \subset (X\times X)\setminus D$. By Proposition
\ref{P:prod_base}, there exists open sets $U,V\subset X$ such that
$(x,y)\in U\times V \subset B \subset (X\times X)\setminus D$, thus 
$U\cap V=\emptyset$, i.e. $X$ is Hausdorff.

Conversely, suppose that $X$ is Hausdorff. Then for any $x,y\in X$ and $x\neq y$
there exist open sets $U,V\subset X$ such that $x\in U$, $y\in Y$ and 
$U\cap V=\emptyset$. Hence $(x,y)\in U\times V \subset (X\times X)\setminus D$,
we conclude that $(X\times X)\setminus D$ is an open set in $X\times X$, thus
$D$ is closed.
\end{proof}

%%%%%%%%%%%%%%%%%%%%%%%%%%%%%%%%%
\begin{lemma} \label{L:haus_graph}
\footnote{cf. Aliprantis and Border, Infinite Dimensional Analysis: A 
	  Hitchhiker's Guide, Theorem 2.58, Closed Graph Theorem, p.51}
Let $f:X\to Y$ be a continuous function from topological space $X$ to Hausdorff 
space $Y$, then the graph of $f$
\[
  \gr f = \{ (x,y)\in X\times Y : y=f(x) \}
\]
is a closed set in product topology $X\times Y$.
\end{lemma}
%%%%%%%%%%%%%%
\begin{proof}[Proof 1]
\footnote{Adapted from Henno Brandsma, Topology Atlas website, Apr. 2, 2004}
For any $(x,y)\in (X\times Y) \setminus \gr f$, we have $y\neq f(x)$, since $Y$
is Hausdorff, by Defintion \ref{D:haus}, there are two open sets $U$ and
$V$ such that $f(x)\in U$, $y\in V$, and $U\cap V=\emptyset$.
Since $f$ is continuous, by Definition \ref{D:cont}, $f^{-1}(U)$ is an
non-empty open set in $X$. By Proposition \ref{P:prod_base}, $f^{-1}(U)\times V$
is an open set in $X\times Y$, and it is easy to see that 
$f^{-1}(U)\times V \cap \gr f=\emptyset$, hence 
\[
  (x,y)\in f^{-1}(U)\times V \subset (X\times Y) \setminus \gr f,
\]
thus we conclude that $(X\times Y) \setminus \gr f$ is an open set in 
$X\times Y$, thus $\gr f$ is closed.
\end{proof}
%%%%%%%%%%%%%%
\begin{proof}[Proof 2]
	\footnote{Adapted from Henno Brandsma, Topology Atlas website, Apr. 2, 2004}
We define a function $F:X\times Y\to Y\times Y$ as $F((x,y))=(f(x),y)$. Since
$f$ is continuous, we conclude that $F$ is continuous too. By Proposition
\ref{P:haus_diag}, $Y$ is Hausdorff iff $D=\{ (y,y):y\in Y \}$ is closed. Now $F$
is continuous, thus $\gr f=F^{-1}(D)$ is closed too.
\end{proof}

This lemma can be easily extended to countable product of topological spaces:

%%%%%%%%%%%%%%%%%%%%%%%%
\begin{proposition}
Let $f_t:X_t\to Y$ be a continous function for every $t\in T$, where $X_t$ is a
topological space and $Y$ is Hausdorff, then the set
\[
  B=\{ (y,x_1,x_2,\dots)\in Y\times \prod_{t\in T} X_t: y=f_1(x_1)=f_2(x_2)=\dots \}
\]
is a closed set in product topology $Y\times \prod_{t\in T} X_t$.
\end{proposition}

%%%%%%%%%%%%%%%%%%%%%%%%%%%%%%%%%
\begin{proposition} \label{P:suslin_haus}
  \footnote{Federer 2.2.10, p.65}
	If $X$ is a Hausdorff space and $f:\mathcal{N}\to X$ is continuous, then
	$\img f$ is a Suslin subset of $X$.
\end{proposition}
\begin{proof}
Similar to Lemma \ref{L:haus_graph}, we have that set
\[
	G = \{ (f(n),n): n\in\mathcal{N} \}
\]
is a closed set in product topology $X\times \mathcal{N}$. Now $\img f$ is a
Suslin subset of $X$ because
\[
	\img f = \{ f(n): n\in \mathcal{N} \} = p(G).
\]
\end{proof}


%%%%%%%%%%%%%%%%%%%%%%%%%%%%%%%%%
\begin{proposition} \label{P:suslin_compltSep}
  \footnote{Federer 2.2.10, p.65}
	For each nonempty Suslin subset $S$ of a complete separable metric space $X$
	there exists a continuous map $h$ of $\mathcal{N}$ onto $S$.
\end{proposition}
\begin{proof}
By Proposition \ref{P:prod_sep} and Theorem \ref{T:prod_complt}, 
$X \times\mathcal{N}$ is a complete separable space. By Definition
\ref{D:suslin_set} there exists a closed subset $C\subset X\times \mathcal{N}$
such that $S=p(C)$. Now by Proposition \ref{P:metric_sep} and Proposition
\ref{P:closed_complete} we know that $C$ is also a complete separable metric
space. And following Theorem \ref{T:lipschitz_compltSep}, there is a locally
Lipschitzian map $g$ from $\mathcal{N}$ onto $C$. It is easy to see from the
definitions that $g$ is also a continuous map. Thus $h=p\circ g$ is a continuous
function from $\mathcal{N}$ onto $S$.
\end{proof}

%%%%%%%%%%%%%%%%%%%%%%%%%%%%%%%%%
\begin{corollary}
  \footnote{Federer 2.2.10, p.65}
	If $Y$ is a Hausdorff space, $S$ is a Suslin subset of some complete separable 
	metric space, and $f:S\to Y$ is continuous, then $f(S)$ is a Suslin subset of
	$Y$.
\end{corollary}
\begin{proof}
	By Proposition \ref{P:suslin_compltSep}, there exists a continuous
	$h:\mathcal{N}\to S$, thus $g=f\circ h$ is a continuous function from
	$\mathcal{N}$ onto Hausdorff space $Y$. Now following Proposition
	\ref{P:suslin_haus} we conclude that $f(S)=\img (f\circ h)$ is a Suslin subset
	of $Y$.
\end{proof}

%%%%%%%%%%%%%%%%%%%%%%%%%%%%%%%%%
%%%  inverse image of continuous function of a Suslin subset is Suslin
\begin{corollary} \label{C:suslin_cont}
  \footnote{Federer 2.2.10, p.66}
	If $f:X\to Y$ is continuous and $S$ is a Suslin subset of $Y$, then
	$f^{-1}(S)$ is a Suslin subset of $X$.
\end{corollary}
\begin{proof}
	By definition, there is a closed set $C_Y\in Y\times\mathcal{N}$ such that
	$S=p_Y\{(y,n)\in C_Y\}$. Now let
	\[
		C_X = \{(x,n)\in X\times\mathcal{N}: (f(x),n)\in C_Y\},
	\]
  and $C_X$ is closed in $X\times\mathcal{N}$ because $f$ is continuous.
	Note that $f^{-1}(S)=p_X(C_X)$, thus $f^{-1}(S)$ is Suslin in $X$. 
\end{proof}

Next we study relations between Suslin sets and Borel sets. Let $F$ be the
family consisting of all sets $p(C)$ such that $C$ is closed in 
$X\times \mathcal{N}$ and $p|_C$ is univalent.

%%%%%%%%%%%%%%%%%%%%%
\begin{lemma}
If $X$ is a metric space and $S$ is open in $X$, then $S\in F$.
\end{lemma}
%%%%%%%%%%%%%%%%%%%%%
\begin{proof}
Let 
\[
  A = \{ (x,t)\in (X\times R): t \cdot \dist(x,X\setminus S) = 1 \},
\]
For any $(x,t)\in (X\times R)\setminus A$, there exists $\epsilon,\delta>0$ such
that open set $B(x,\epsilon)\times (t-\delta,t+\delta)$
does not contain any point of $A$.
%\footnote{e.g. $\delta>t$, and $\epsilon>\frac{\delta}{t(\delta-t)}$.}
%$(x-\epsilon,x+\epsilon)\times (t-\delta,t+\delta)$ (of 
%$X\times R$) does not contain point $(x,t)$. 
Hence $(x\times R)\setminus A$ is open in $X\times R$, thus $A$ is closed 
in $X\times R$.

Since $R$ is a Polish space, by Theorem \ref{T:pol_bij}, there is a closed set
$B\subset\mathcal{N}$ and a continuous bijection $\psi:B\to R$. Let
\[
  C = \{ (x,n)\in (X\times\mathcal{N}) : n\in B, (x,\psi(n))\in A \},
\]
we verify that $C$ is a closed set in $X\times\mathcal{N}$.
Note that for any $(x,n)\in (X\times\mathcal{N})\setminus C$, we have
$(x,\psi(n))\in (X\times R)\setminus A$, 
there exists $\epsilon,\delta>0$ such that open set 
$B_X(x,\epsilon)\times B_R(\psi(n),\delta)\subset (X\times R)\setminus A$.
Using the defintion \ref{D:cont_metr} of continuous function, we get that there
exist $\delta'>0$ such that 
$\psi(B_{\mathcal{N}}(n,\delta'))\subset B_R(\psi(n),\delta)$. Hence the open
set
$B_X(x,\epsilon)\times B_{\mathcal{N}}(n,\delta')\subset
(X\times\mathcal{N})\setminus C$,
i.e., $(X\times\mathcal{N})\setminus C$ is open, thus $C$ is closed
in $X\times\mathcal{N}$.

Next we verify that $p|_C$ is univalent. Suppose that $(x,n),(x,m)\in C$ such
that $p((x,n))=x$ and $p((x,m))=x$, then $(x,\psi(n))\in A$ and 
$(x,\psi(m))\in A$. It is easy to see that $\psi(n)=\psi(m)$, and because $\psi$
is a bijection, we have $n=m$, thus $p|_C$ is univalent.

Finally, for any $(x,n)\in C$, we have $p((x,n))=x$ and $(x,\psi(n))\in A$, 
hence $\psi(n) \cdot \dist(x,X\setminus S)=1$, we conclude that $x\in S$. Thus 
$p(C)=S$.

Hence $S\in F$.
\end{proof}


%%%%%%%%%%%%%%%%%%%%%%%
\begin{definition}
Let $(P_s)_{s\in\mathcal{P}^{<\mathcal{P}}}$ be a \textbf{Souslin scheme}
\index{Souslin scheme}
on a set $X$, i.e., a family of subsets of $X$ indexed by
$\mathcal{P}^{<\mathcal{P}}$. The \textbf{Souslin operation}
\index{Souslin operation}
\footnote{Kechris, Classical Descriptive Set Theory, Definition 25.4, p.198}
$\mathcal{A}$ applied to such a scheme produces the set
\[
  \mathcal{A} P_s 
    = \bigcup_{x\in\mathcal{N}} \bigcap_{n\in\mathcal{P}} P_{x|n}.
\]
which is called the \textbf{kernel of the Souslin operation}
\index{kernel of the Souslin operation}.

Given any collection $\Gamma$ of subsets of a set $X$ we denote by
$\mathcal{A} \Gamma$ the class of sets $\mathcal{A}_s P_s$, where
$P_s \subset X$ are in $\Gamma$.
\end{definition}


%%%%%%%%%%%%%%%%%%%%%%%
\begin{proposition} \label{P:souslin_idem}
Let $X$ be a set and $\Gamma\subset 2^X$. Then 
$\mathcal{A}\mathcal{A}\Gamma=\mathcal{A}\Gamma$.
  \footnote{Kechris, Classical Descriptive Set Theory, Proposition 25.6, 
    pp. 198-199; Srivastava, A Course on Borel Sets, Theorem 1.13.1, pp.35-36}
\end{proposition}
%%%%%%%%%%%%%%%%%%%%%%%
\begin{proof}
It is trivial that for any $\Gamma$, $\Gamma\subset\mathcal{A}\Gamma$. 
\footnote{
  For any $A\in\Gamma$, define Souslin scheme $(P_s)$ with $P_s=A$ for each
  $s\in \mathcal{P}^{<\mathcal{P}}$, then 
  $A=\mathcal{A}_s P_s\in\mathcal{A}\Gamma$.
}
So it is enough to show that 
$\mathcal{A}\mathcal{A}\Gamma\subset \mathcal{A}\Gamma$.
Let $A=\mathcal{A}P_s$, with $P_s\in\mathcal{A}\Gamma$, so that 
$P_s=\mathcal{A}Q_{s,t}$, with $Q_{s,t}\in\Gamma$. Then
\begin{align*}
  x\in A 
    & \iff \exists y\in\mathcal{N} \forall m (x\in P_{y|m}) \\
    & \iff \exists y\in\mathcal{N} \forall m 
        \exists z\in\mathcal{N} \forall n (x\in Q_{y|m, z|n}) \\
    & \iff \exists y\in\mathcal{N} \exists(z_m)\in\mathcal{N}^{\mathcal{P}}
        \forall m \forall n (x\in Q_{y|m, z_m|n}).
\end{align*}

We aim to find some Souslin scheme $(R_s)$ such that 
$x\in\mathcal{A}R_s$, or roughly speaking, to find some mapping
from $(y,(z_m))$ to some $w\in\mathcal{N}$, and some mapping from
$(m,n)$ to some $k\in\mathcal{P}$, such that $R_{w|k}=Q_{y|m,z_m|n}$.

Fix now a bijection $\langle m,n\rangle$ of $\mathcal{P}\times \mathcal{P}$ 
with $\mathcal{P}$, so that $m\le\langle m,n \rangle$ and 
($p<n \Rightarrow \langle m,p \rangle < \langle m,n \rangle$)
(e.g. $\langle m,n \rangle = 2^m (2n+1) - 1$). Let also for $k\in \mathcal{P}$,
$(k)_0$, $(k)_1$ be such that $\langle (k)_0,(k)_1 \rangle=k$. Then encode
$(y,(z_m))\in \mathcal{N}\times\mathcal{N}^{\mathcal{P}}$ by $w\in\mathcal{N}$
given by 
\[
  w(k)=\langle y(k), z_{(k)_0}((k)_1)\rangle. 
\]
This gives a bijection 
$f:\mathcal{N}\times\mathcal{N}^{\mathcal{P}}\to\mathcal{N}$. 
To see this, take $(y,(z_m))\neq (y',(z'_m))$ and let 
$w=f(y,(z_m))$ and $w'=f(y',(z'_m))$. Now if
$y\neq y'$, then there exists a $k$ such that $y(k)\neq y'(k)$, thus
$w(k)\neq w'(k)$ because $\langle\rangle$ is a bijection. On the other hand,
if $(z_m)\neq (z'_m)$, then there exists $i,j$ that $z_i(j)\neq z'_i(j)$,
hence $w(\langle i,j\rangle)\neq w'(\langle i,j\rangle)$.

Note that knowing $w|\langle m,n \rangle$ determines $y|m$ and $z_m|n$, 
by the above properties of $\langle\rangle$, i.e. there are functions 
$\varphi,\psi: \mathcal{P}^{<\mathcal{P}} \to \mathcal{P}^{<\mathcal{P}}$ 
such that if $w$ encodes $(y,(z_m))$ and $s=w|\langle m,n\rangle$, then
$\varphi(s)=y|m$ and $\psi(s)=z_m|n$. 
Actually, for any $s\in \mathcal{P}^{<\mathcal{P}}$, let $m=(|s|)_0$ and 
$n=(|s|)_1$, we define
\[
  \varphi(s)=((s_1)_0, (s_2)_0, \cdots, (s_m)_0)
\]
and
\[
  \psi(s)= ((s_{\langle m,1\rangle})_1, (s_{\langle m,2\rangle})_1, \cdots,
            (s_{\langle m,n\rangle})_1)
\]
it is easy to see that if $w$ encodes $(y,(z_m))$ and $s=w|\langle m,n\rangle$, 
then
\begin{align*}
  y|m &= (y_1, y_2, \cdots, y_m) \\
      &=((w_1)_0, (w_2)_0, \cdots, (w_m)_0) \\
      &=((s_1)_0, (s_2)_0, \cdots, (s_m)_0) \\
      &= \varphi(s)
\end{align*}
and
\begin{align*}
  z_m|n &= (z_m(1), z_m(2), \cdots, z_m(n)) \\
        &= ((w_{\langle m,1\rangle})_1, (w_{\langle m,2\rangle})_1, \cdots,
            (w_{\langle m,n\rangle})_1)  \\
        &= ((s_{\langle m,1\rangle})_1, (s_{\langle m,2\rangle})_1, \cdots,
            (s_{\langle m,n\rangle})_1)  \\
        &= \psi(s)
\end{align*}

It follows that
\begin{align*}
  x\in A 
    & \iff \exists w\in\mathcal{N} \forall k (x\in R_{w|k}) \\
    & \iff x\in \mathcal{A} R_s,
\end{align*}
where $R_s=Q_{\varphi(s),\psi(s)}$ is in $\Gamma$.
\end{proof}

Next we prove the equivalence between Souslin operation and projection operator.
For any $s\in\mathcal{P}^{<\mathcal{P}}$, let
\[
  I_s = \{ t\in\mathcal{N}: s\prec t \}.
\]

%%%%%%%%%%%%%%%%%%%
\begin{lemma} \label{L:souslin_proj},
  \footnote{Fremlin, Measure Theory, vol.4, 421Ce;
    Srivastava, A Course on Borel Sets, Proposition 1.12.1, p.33.}
Let $(P_s)_{s\in\mathcal{P}^{<\mathcal{P}}}$ be a Souslin scheme of set $X$, 
and let
\[
  B = \bigcap_{n\in\mathcal{P}} \bigcup_{s\in\mathcal{P}^n}
      P_s \times I_s,
\]
then $\mathcal{A} P_s = \pi_X(B)$, where $\pi_X: X\times\mathcal{N}\to X$ is the
projection map.
\end{lemma}
%%%%%%%%%%%%%%%%%%%
\begin{proof}
We have
\begin{align*}
  (x,t)\in B 
    & \iff \forall n \exists s\in\mathcal{P}^n (x\in P_s, t\in I_s) \\
    & \iff \forall n (x\in P_{t|n}),
\end{align*}
thus
\begin{align*}
  x\in \pi_x(B)
	& \iff \exists t\in\mathcal{N} ((x,t)\in B) \\
	& \iff \exists t\in\mathcal{N} \forall n\in\mathcal{P} (x\in P_{t|n}) \\
	& \iff x\in \mathcal{A} P_s.
\end{align*}
  
\end{proof}

There is a particularly simple description of sets obtainable by Souslin
operation from closed sets in a topological space.
%%%%%%%%%%%%%%%%%%%
\begin{lemma} \label{L:souslin_close}
\footnote{Fremlin, Measure Theory, vol.4, 421I.}
Let $X$ be a topological space and $C\subset X\times\mathcal{N}$ a closed set.
Then for any $A\subset\mathcal{N}$,
\[
  \pi_{X,A}(C) = \bigcup_{s\in A} \bigcap_{n\in\mathcal{P}}
             \overline{ \pi_{X,I_{s|n}}(C) }.
\]
where the projection map $\pi_{X,A}:X\times \mathcal{N}\to X$ satisfies
\[
  \pi_{X,A}(C) = \{ x\in X: \exists t\in A ((x,t)\in C)  \}
\]
\end{lemma}
%%%%%%%%%%%%%%%%%%%
\begin{proof}
Let 
\[
  B = \bigcup_{s\in A} \bigcap_{n\in\mathcal{P}} \overline{ \pi_{X,I_{s|n}}(C) }.
\]

(i) We verify that $\pi_{X,A}(C)\subset B$. Note that $t\in I_{t|n}$ for any
$t\in\mathcal{N}$ and $n\in\mathcal{P}$, hence
\begin{align*}
  x\in\pi_{X,A}(C)
    & \iff \exists t\in A ((x,t)\in C)  \\
    & \iff \exists t\in A \forall n\in\mathcal{P} \exists s\in I_{t|n} ((x,s)\in C)  \\
    & \iff \exists t\in A \forall n\in\mathcal{P} (x\in \pi_{X,I_{t|n}}(C))  \\
    & \iff x\in \bigcup_{t\in A} \bigcap_{n\in\mathcal{P}} \pi_{X,I_{t|n}}(C)
	  \subset B.
\end{align*}

(ii) We verify that $B\subset\pi_{X,A}(C)$. Note that
\[
  x\in B \iff \exists s\in A \forall n\in\mathcal{P} ( x\in \overline{\pi_{X,I_{s|n}}(C)} ).
\]
Now if $(x,s)\notin C$, then (because $C$ is closed in $X\times \mathcal{N}$)
there exists an open set $G\subset X$, and $t\in \mathcal{P}$ such that 
$s\in I_t$, $x\in G$, and $(I_t\times G)\cap C=\emptyset$.
But this means that $G\cap \pi_{X,I_t}(C)=\emptyset$, 
so $G\cap \overline{\pi_{X,I_t}(C)} =\emptyset$, and $x\notin \overline{\pi_{X,I_t}(C)}$.
However this contradicts with the fact that $s\in I_t$ and that 
$x\in \overline{\pi_{X,I_{s|n}}(C)}$ for any $n\in\mathcal{P}$.
Hence $(x,s)\in C$, and so $x\in \pi_{X,A}(C)$.
\end{proof}

We can now establish the equivalence of Souslin sets and Souslin operation on
family of closed sets:

%%%%%%%%%%%%%%%%%%%
\begin{proposition} \label{P:souslin_equiv}
\footnote{Fremlin, Measure Theory, vol.4, Proposition 421J.}
Let $X$ be a topological space, and $\mathcal{F}$ the family of closed subsets
of $X$. Then a set $A\subset X$ is the kernel of a Souslin scheme in  
$\mathcal{F}$ iff there is a closed set $C\in X\times\mathcal{N}$ such that $A$
is the projection of $C$ on $X$.
\end{proposition}
%%%%%%%%%%%%%%%%%%%
\begin{proof}
Let $(F_s)_{s\in\mathcal{P}^{<\mathcal{P}}}$ be a Souslin scheme in
$\mathcal{F}$, and let $A=\mathcal{A} F_s$ be the kernel of Souslin operation 
on it. Set
\[
  C = \bigcap_{n\in\mathcal{P}} \bigcup_{s\in\mathcal{P}^n} F_s \times I_s,
\]
now $C$ is closed in $X\times\mathcal{N}$, and from Lemma \ref{L:souslin_proj},
$A=\pi_X(C)$.

Conversely, suppose that $C\in X\times\mathcal{N}$ is a closed set such that
$A=\pi_X(C)$. Using Lemma \ref{L:souslin_close}, $A$ is the kernel of Souslin
scheme $(\overline{\pi_{X,I_s}(C)})_{s\in\mathcal{P}^{<\mathcal{P}}}$.
\end{proof}

%%%%%%%%%%%%%%%%%%%
\begin{proposition} \label{P:souslin_oper}
The set of all Souslin sets is closed under Souslin operation.
  \footnote{
    Kechris, Classical Descriptive Set Theory, Corrollary 25.8, p.199;
    Srivastava, A Course on Borel Sets, Proposition 4.1.14, p.133.}
\end{proposition}
%%%%%%%%%%%%%%%%%%%
\begin{proof}
By Proposition \ref{P:souslin_idem}, the Souslin operation is idempotent; i.e.,
for any family $\mathcal{F}$ of sets, 
$\mathcal{A}(\mathcal{A}(\mathcal{F})) =\mathcal{A}(\mathcal{F})$. Since the set
of all Souslin sets $\Sigma_1^1=\mathcal{A}(\mathcal{F})$, where $\mathcal{F}$
is the family of closed sets, the result follows.
\end{proof}

Let $X$ be a set and $\mathcal{F}$ a family of subset of X, and
\[
	\mathcal{F}_{\sigma} 
	= \{  
	  \bigcup_{n\in\mathcal{P}} A_n: A_n\in\mathcal{F}
	\}
\]
the family of countable unions of sets in $\mathcal{F}$,
\[
	\mathcal{F}_{\delta} 
	= \{  
	  \bigcap_{n\in\mathcal{P}} A_n: A_n\in\mathcal{F}
	\}
\]
the family of countable intersections of sets in $\mathcal{F}$.

%%%%%%%%%%%%%%%%%%%
\begin{proposition} \label{P:souslin_fsigma}
\footnote{Srivastava, A Course on Borel Sets, Proposition 1.12.2, p.33}
For every family $\mathcal{F}$ of subsets of $X$,
\[
	\mathcal{F}_{\sigma}, \mathcal{F}_{\delta} \subset \mathcal{A}(\mathcal{F}).
\]
\end{proposition}
%%%%%%%%%%%%%%%%%%%
\begin{proof}
Let $(A_n)$ be a sequence in $\mathcal{F}$. 

(i) $\mathcal{F}_{\sigma} \subset \mathcal{A}(\mathcal{F})$. For every
$s=(s_1,s_2,\dots,s_m)\in\mathcal{P}^{<\mathcal{P}}$, let $B_s=A_{s_1}$, we have
\[
  \mathcal{A} B_s 
    = \bigcup_{x\in\mathcal{N}} \bigcap_{n\in\mathcal{P}} B_{x|n}
    = \bigcup_{x\in\mathcal{N}} \bigcap_{n\in\mathcal{P}} A_{x_1}
    = \bigcup_{n\in\mathcal{P}} A_n.
\]

(ii) $\mathcal{F}_{\delta} \subset \mathcal{A}(\mathcal{F})$. For every
$s=(s_1,s_2,\dots,s_m)\in\mathcal{P}^{<\mathcal{P}}$, let $C_s=A_{|s|}$, we have
\[
  \mathcal{A} C_s 
    = \bigcup_{x\in\mathcal{N}} \bigcap_{n\in\mathcal{P}} C_{x|n}
    = \bigcup_{x\in\mathcal{N}} \bigcap_{n\in\mathcal{P}} A_n
    = \bigcap_{n\in\mathcal{P}} A_n.
\]
\end{proof}

%%%%%%%%%%%%%%%%%%%
\begin{proposition} \label{P:borel_gen1}
\footnote{Srivastava, A Course on Borel Sets, Proposition 3.1.9, p.82;
  cf. Proposition \ref{P:borel}. }
The Borel $\sigma$-algebra $\mathcal{B}_X$ of a metrizable space $X$ equals the
smallest family $\mathcal{B}$ of subsets of $X$ that contains all open sets and
that is closed under countable intersections and countable unions.
\end{proposition}
%%%%%%%%%%%%%%%%%%%
\begin{proof}
(i) $\mathcal{B}\subset\mathcal{B}_X$. This is obvious because 
$\mathcal{B}_X$ contains all open sets, is closed under countable unions and 
countable intersections, and $\mathcal{B}$ is the smallest family satisfying 
these conditions.

(ii) $\mathcal{B}_X\subset\mathcal{B}$. Let
\[
	\mathcal{D} = \{ A\in\mathcal{B}: X\setminus A\in\mathcal{B} \}.
\]
It is obvious that $\mathcal{D}\subset\mathcal{B}$.
Since every closed set $C$ in a metrizable space is a $G_{\delta}$ set (Corollary 
\ref{C:g_delta}), $C$ is in $\mathcal{B}$, hence every open set is in
$\mathcal{D}$. Now suppose $A_1,A_2,\dots$ are in $\mathcal{D}$. Then
$A_i,X\setminus A_i\in\mathcal{B}$ for all $i$. As
\[
	X\setminus\bigcup_i A_i = \bigcap_i (X\setminus A_i), \qquad
	X\setminus\bigcap_i A_i = \bigcup_i (X\setminus A_i),
\]
$\bigcup_i A_i$ and $\bigcap_i A_i$ belong to $\mathcal{D}$. Thus 
$\mathcal{D}$ contains all open sets, is closed under countable unions and 
countable intersections, and $\mathcal{B}$ is the smallest family satisfying 
these conditions, hence $\mathcal{B}\subset\mathcal{D}$, thus 
$\mathcal{B}=\mathcal{D}$.
Now $\mathcal{B}$ contains all open sets, is closed under complementation,
countable unions and countable intersections, and $\mathcal{B}_X$ is the smallest 
family satisfying these conditions, hence $\mathcal{B}_X\subset\mathcal{B}$.
\end{proof}

%%%%%%%%%%%%%%%%%%%
\begin{proposition} \label{P:borel_gen2}
\footnote{Srivastava, A Course on Borel Sets, Proposition 3.1.10, p.83}
The Borel $\sigma$-algebra $\mathcal{B}_X$ of a metrizable space $X$ equals the
smallest family $\mathcal{B}$ of subsets of $X$ that contains all closed sets and
that is closed under countable intersections and countable unions.
\end{proposition}
%%%%%%%%%%%%%%%%%%%
\begin{proof}
Similar to the proof of Proposition \ref{P:borel_gen1}, except using the fact
that every open set in a metrizable space is a $F_{\sigma}$ set (Corollary 
\ref{C:g_delta}). 
\end{proof}

The condition of closure under countable unions may be reduced to countable
disjoint union.
%%%%%%%%%%%%%%%%%%%
\begin{proposition} \label{P:borel_gen3}
\footnote{Srivastava, A Course on Borel Sets, Proposition 3.1.11, p.84}
The Borel $\sigma$-algebra $\mathcal{B}_X$ of a metrizable space $X$ equals the
smallest family $\mathcal{B}$ of subsets of $X$ that contains all open sets and
that is closed under countable intersections and countable disjoint unions.
\end{proposition}
%%%%%%%%%%%%%%%%%%%
\begin{proof}
(i) $\mathcal{B}\subset\mathcal{B}_X$. This is obvious because 
$\mathcal{B}_X$ contains all open sets, is closed under countable disjoint unions and 
countable intersections, and $\mathcal{B}$ is the smallest family satisfying 
these conditions.

(ii) $\mathcal{B}_X\subset\mathcal{B}$. Let
\[
	\mathcal{D} = \{ A\in\mathcal{B}: X\setminus A\in\mathcal{B} \}.
\]
It is obvious that $\mathcal{D}\subset\mathcal{B}$.
Since every closed set $C$ in a metrizable space is a $G_{\delta}$ set (Corollary 
\ref{C:g_delta}), $C$ is in $\mathcal{B}$, hence every open set is in
$\mathcal{D}$. Now suppose $A_1,A_2,\dots$ are in $\mathcal{D}$. Then
$A_i,X\setminus A_i\in\mathcal{B}$ for all $i$. It is easy to see
$\bigcap_i A_i\in\mathcal{B}$. Let $B_1=X\setminus A_1$, 
$B_2=(X\setminus A_2)\bigcap A_1$, 
$B_3=(X\setminus A_3)\bigcap A_1 \bigcap A_2$, etc. It is easy to see that 
$B_i\in\mathcal{B}$ for each $i$, $B_i\bigcap B_j=\emptyset$ for each $i\ne j$,
and
\[
	X\setminus\bigcap_i A_i = \bigcup_i (X\setminus A_i)
	  = \bigcup_i B_i \in \mathcal{B},
\]
hence $\bigcap_i A_i\in\mathcal{D}$.

Next suppose $A_1,A_2,\dots$ are in $\mathcal{D}$, and 
$A_i\bigcup A_j=\emptyset$ for any $i\ne j$. It is easy to see that 
$\bigcup_i A_i\in\mathcal{B}$. And
\[
	X\setminus\bigcup_i A_i = \bigcap_i (X\setminus A_i) \in \mathcal{B},
\]
hence $\bigcup_i A_i\in\mathcal{D}$.
Thus $\mathcal{D}$ contains all open sets, is closed under countable disjoint unions and 
countable intersections, and $\mathcal{B}$ is the smallest family satisfying 
these conditions, hence $\mathcal{B}\subset\mathcal{D}$, thus 
$\mathcal{B}=\mathcal{D}$.
Now $\mathcal{B}$ contains all open sets, is closed under complementation,
countable disjoint unions and countable intersections, and $\mathcal{B}_X$ is the smallest 
family satisfying these conditions, hence $\mathcal{B}_X\subset\mathcal{B}$.
\end{proof}

%%%%%%%%%%%%%%%%%%%
\begin{definition}
We define the \textbf{Cantor ternary set} 
\index{Cantor ternary set} 
in the following way:
we start from $C_{\emptyset}=[0,1]$,
and for any $s\in\{0,1\}^{<\mathcal{P}}$, let $C_s=[a,b]$, we define
\[
	C_{s^\wedge 0} = \left[ a, a+\frac{b-a}{3} \right], \quad 
	C_{s^\wedge 1} = \left[ b-\frac{b-a}{3}, b \right],
\]
and for any $n\in\mathcal{P}$
\[
	K_n=\bigcup_{|s|=n} C_s,
\]
and finally the Cantor ternary set
\[
	C=\bigcap_{n\in\mathcal{P}} K_n
\]
\end{definition}

%%%%%%%%%%%%%%%%%%%
\begin{lemma} \label{L:cantor_base}
\footnote{cf
\url{https://math.stackexchange.com/questions/2946172/the-topology-of-the-cantor-set}
}
$\{C\cap C_s: s\in\{0,1\}^{<\mathcal{P}} \}$ is a base of the Cantor ternary
set $C$.
\end{lemma}
%%%%%%%%%%%%%%%%%%%
\begin{proof}
First note that from the definition of the Cantor ternary set, it is easy to see 
that for any $s,t \in\{0,1\}^{<\mathcal{P}}$ and $s\neq t$, if $s\prec t$, then 
$C_s\supset C_t$, otherwise $C_s\cap C_t=\emptyset$.

Let $\mathcal{B}=\{C\cap C_s: s\in\{0,1\}^{<\mathcal{P}} \}$. It is easy to
see that $\mathcal{B}$ satisfies condition (B1). And for any $x\in C$, we have
$x\in K_1$, hence $x\in C\cap C_{(0)}$ or $x\in C\cap C_{(1)}$, thus
$\mathcal{B}$ also satisfies condition (B2).
Hence by Proposition \ref{P:top_base} we conclude that $\mathcal{B}$ is a base of
$C$.
\end{proof}

%%%%%%%%%%%%%%%%%%%
\begin{lemma}
For any $s\in\{0,1\}^{<\mathcal{P}}$, $C\cap C_s$ is clopen.
\end{lemma}
%%%%%%%%%%%%%%%%%%%
\begin{proof}
By Lemma \ref{L:cantor_base}, $C\cap C_s$ is open, we only need to verify
that it is closed. Note that
\[
  C\setminus (C\cap C_s) 
	= \bigcup_{ |t|=|s|, t\neq s } (C\cap C_t),
\]
hence $C\setminus (C\cap C_s)$ is open, thus $C\cap C_s$ is closed.
\end{proof}

For any family $\mathcal{F}$, let $\mathcal{F}_+$ denote the family of countable
disjoint unions of sets in $\mathcal{F}$.

%%%%%%%%%%%%%%%%%%%
\begin{lemma} \label{L:cantor_fplus}
Let $\mathcal{F}$ be the set of closed subsets of $\mathbb{R}$. For any $x\in C$, 
$C\setminus \{x\}\in \mathcal{F}_+$.
\end{lemma}
%%%%%%%%%%%%%%%%%%%
\begin{proof}
By Corollary \ref{C:metric_T1}, singleton $\{x\}$ is closed, hence 
$C\setminus \{x\}$ is open. By Lemma \ref{L:cantor_base}, 
$\{C\cap C_s: s\in\{0,1\}^{<\mathcal{P}} \}$ is a base of $C$, hence
there exists a $S\subset \{0,1\}^{<\mathcal{P}}$ such that 
\[
  C\setminus \{x\} = \bigcup_{s\in S} (C\cap C_s).
\]
Note that $C\cap C_s$ is clopen, and that for any 
$s,t\in \{0,1\}^{<\mathcal{P}}$, either $C_s\cap C_t=\emptyset$, or (without
loss of generality) $C_s\subset C_t$, hence we can safely make all
subsets in $S$ disjoint, thus $C\setminus \{x\}\in \mathcal{F}_+$.
\end{proof}
%%%%%%%%%%%%%%%%%%%


%%%%%%%%%%%%%%%%%%%
\begin{lemma} \label{L:cantor_fplus2}
\footnote{Srivastava, A Course on Borel Sets, Lemma 3.1.13, pp.84-85}
Let $\mathcal{F}$ be the set of closed subsets of $\mathbb{R}$. 
Then $(0,1]\in \mathcal{F}_{+\delta +}$.
\end{lemma}
%%%%%%%%%%%%%%%%%%%
\begin{proof}
Let $D$ be the set of all end points of the middle-third intervals removed from
$[0,1]$ to construct the Cantor ternary set $C$. So, 
$D=\{\frac{1}{3},\frac{2}{3},\frac{1}{9},\frac{2}{9},\frac{7}{9},\frac{8}{9},\dots
\}$. Let $E=D\cup \{0\}$ and $P=C\setminus E$. Note that
\[
  (0,1]\setminus P = \left[ \frac{1}{3},\frac{2}{3} \right] \bigcup 
                     \left[ \frac{1}{9},\frac{2}{9} \right] \bigcup
                     \left[ \frac{7}{9},\frac{8}{9} \right] \bigcup \dots,
\]
i.e. $(0,1]\setminus P$ is the union of the closures of the middle-third
intervals removed to form $C$. These intervals are, of course, disjoint.
Therefore, the lemma will be proved if we show that $P$ is in
$\mathcal{F}_{+\delta}$. Now
\[
  P=\bigcap_{x\in E} (C\setminus \{ x \}).
\]
By Lemma \ref{L:cantor_fplus}, $C\setminus \{ x \} \in \mathcal{F}_+$. Since $E$
is countable, we have $P\in \mathcal{F}_{+\delta}$.
\end{proof}
%%%%%%%%%%%%%%%%%%%


%%%%%%%%%%%%%%%%%%%
\begin{proposition}[Urysohn's lemma on metric space]  \label{P:uryshohn}
\footnote{Srivastava, A Course on Borel Sets, Proposition 2.1.18, pp.44-45}
Suppose $A,B$ are two nonempty, disjoint closed subsets of a metrizable space
$X$. Then there is a continuous function $u:X\to [0,1]$ such that
\[
  u(x)=
    \begin{cases} 
      0     & \text{if $x\in A$}    \\
      1     & \text{if $x\in B$}
    \end{cases}
\]
\end{proposition}
%%%%%%%%%%%%%%%%%%%
\begin{proof}
Let $d$ be a compatible metric on $X$. Take
\[
  u(x)=\frac{d(x,A)}{d(x,A)+d(x,B)},
\]
since $d(x,A)$ and $d(x,B)$ are continuous, $u$ is continuous, and it is easy to
check that $u(x)=0$ if $x\in A$ and $u(x)=1$ if $x\in B$.
\end{proof}
%%%%%%%%%%%%%%%%%%%

%%%%%%%%%%%%%%%%%%%
\begin{proposition} \label{P:metric_contin}
\footnote{Srivastava, A Course on Borel Sets, Proposition 2.1.19, p.45}
For every nonempty closed subset $A$ of a metrizable space $X$ there is a
continuous function $f:X\to [0,1]$ such that $A=f^{-1}(0)$.
\end{proposition}
%%%%%%%%%%%%%%%%%%%
\begin{proof}
By Corollary \ref{C:g_delta}, $A$ is the countable intersection of open sets,
i.e. $A=\bigcap_n U_n$, where $U_n$ are open. Now by Uryshohn's Lemma on metric
space (Proposition \ref{P:uryshohn}), for each $n\in \mathcal{P}$, there is a
continuous function $f_n:X\to [0,1]$ such that 
\[
  f_n(x)=
    \begin{cases} 
      0     & \text{if $x\in A$}    \\
      1     & \text{if $x\in X\setminus U_n$}
    \end{cases}
\]
Now take
\[
	f(x) = \sum_{n=0}^{\infty} \frac{1}{2^{n+1}} f_n(x).
\]
and $f$ is continuous. It is easy to see that $f(A)=0$, and for any 
$x\in X\setminus A$, there is a $n$ such that $x\in X\setminus U_n$, hence
$f_n(x)=1$, and thus $f(x)\neq 0$. Thus we conclude that $f^{-1}(0)=A$.
\end{proof}


%%%%%%%%%%%%%%%%%%%
\begin{proposition} \label{P:borel_gen4}
\footnote{Srivastava, A Course on Borel Sets, Proposition 3.1.12, pp.84-85;
	Bruckner et.al., Real Analysis, 2ed., Theorem 3.4, pp.108-109.}
The Borel $\sigma$-algebra $\mathcal{B}_X$ of a metrizable space $X$ equals the
smallest family $\mathcal{B}$ of subsets of $X$ that contains all closed sets and
that is closed under countable intersections and countable disjoint unions.
\end{proposition}
%%%%%%%%%%%%%%%%%%%
\begin{proof}
Apparently we only need to show that any open set $U$ of $X$ is in 
$\mathcal{B}$. By Proposition \ref{P:metric_contin}, there is a continuous
function $f:X\to [0,1]$ such that $U=f^{-1}((0,1])$. Now by Lemma
\ref{L:cantor_fplus2}, $(0,1]\in\mathcal{F}_{+\delta +}$, where $\mathcal{F}$ is
the set of closed subsets of $\mathbb{R}$. Hence
\[
  U=f^{-1}((0,1]) \in f^{-1}(\mathcal{F}_{+\delta +})
   = f^{-1}(\mathcal{F})_{+\delta +}. 
\]
And because $f^{-1}(\mathcal{F})$ is a family of closed subsets of $X$, we
conclude that $U \in \mathcal{B}$.
\end{proof}


%%%%%%%%%%%%%%%%%%%
\begin{proposition}
Every Borel set is Souslin.
  \footnote{cf. Krantz, Geometric Integration Theory, Theorem 1.7.9, p.46}
\end{proposition}
%%%%%%%%%%%%%%%%%%%
\begin{proof}
First by Proposition \ref{P:souslin_equiv}, any Souslin set is the kernel of
Souslin operation on closed sets, thus all closed sets are Souslin. And by
Proposition \ref{P:souslin_oper}, the set of all Souslin sets is closed under 
Souslin operation. Hence by Proposition \ref{P:souslin_fsigma}, the set of all
Souslin sets is closed under countable union and countable intersection. Thus by
Proposition \ref{P:borel_gen2}, it contains all Borel sets.
\end{proof}

%%%%%%%%%%%%%%%%%%%%%%%
\begin{definition}
Two subsets P and Q of a topological space Y are said to be 
\textbf{Borel separated} 
\index{Borel separated}
\footnote{Federer 2.2.10, p.66}
if there exist disjoint Borel sets A and B of Y with 
$P\subset A$ and $Q\subset B$.
\end{definition}

%%%%%%%%%%%%%%%%%%%%%%%%%%%%%%%%%%%%%
\begin{lemma}
\footnote{Aliprantis and Border, Infinite Dimensional Analysis: A 
  Hitchhiker's Guide, p.448}
If two subsets A and B of a topological space can be written as 
$A=\bigcup_{n\in\mathcal{P}} A_n$ and 
$B=\bigcup_{m\in\mathcal{P}} B_m$, where each pair $A_n$ and $B_m$ are Borel
separated, then the pair $A$ and $B$ are also Borel separated.
\end{lemma}
%%%%%%%%%%%%%%%%%%%
\begin{proof}
For each $n,m$, we choose disjoint Borel sets $A'_{n,m}$ and $B'_{n,m}$ such
that $A_n\subset A'_{n,m}$ and $B_m\subset B'_{n,m}$. Let
$A'=\bigcup_n \bigcap_m A'_{n,m}$ and $B'=\bigcup_m \bigcap_n B'_{n,m}$,
it is easy to see that $A\subset A'$ and $B\subset B'$, since 
$A_n\subset\bigcap_m A'_{n,m}$ and $B_m\subset\bigcap_n B'_{n,m}$.
Also $A'\bigcap B'=\emptyset$ because 
$(\bigcap_j A'_{n,j})\bigcap(\bigcap_k B'_{k,m})=\emptyset$.
Thus $A$ and $B$ are Borel separated.
\end{proof}
%%%%%%%%%%%%%%%%%%%


%%%%%%%%%%%%%%%%%%%%%%%%%%%%%%%%%%%%%
\begin{lemma} \label{L:sep_lusin}
\footnote{Federer, 2.2.10, p.67;
  cf. Bruckner et.al., Real Analysis, 2ed., Theorem 11.14, pp.443-444}
Assuming that X is a complete, separable metric space, Y is a Hausdorff space
and $f:X\to Y$ is continuous, if C and D are closed subsets of X such that 
$f(C)$ and $f(D)$ are disjoint, then $f(C)$ and $f(D)$ are Borel separated in Y. 
\end{lemma}
%%%%%%%%%%%%%%%%%%%
\begin{proof}
First we carve the space X according to the same method used in 
Theorem \ref{T:lipschitz_compltSep}, $X=\bigcup_{j\in\mathcal{P}} E(j)$.
Suppose that $f(C)$ and $f(D)$ are not Borel separated, then there exist
$m_1,n_1\in \mathcal{P}$ such that $f(C\bigcap E(n1))$ and $f(D\bigcap E(m_1))$
are not Borel separated. By induction for any $k\in\mathcal{P}$, there exists
$(n_1,n_2,\dots,n_k)$ and $(m_1,m_2,\dots,m_k)$ such that 
$f(C\bigcap E(n_1,n_2,\dots,n_k))$ and $f(D\bigcap E(m_1,m_2,\dots,m_k))$
are not Borel separated. 

Note that the intersections 
$C\cap \bigcap_{k\in\mathcal{P}} E(n_1,n_2,\dots,n_k)$ and
$D\cap \bigcap_{k\in\mathcal{P}} E(m_1,m_2,\dots,m_k)$ would consist of single
points $u\in C$ and $v\in D$. And because Y is Hausdorff, there exist disjoint
open sets $U_1$ and $U_2$ in Y such that $f(u)\in U_1$ and $f(v)\in U_2$.
Now because $f$ is continuous, $f^{-1}(U_1)$ is open in X, and because X is a
metric space, there exist $r>0$ such that open ball $B(u,r)\subset f^{-1}(U_1)$,
hence there exist a $i\in\mathcal{P}$ that 
$C\bigcap E(n_1,n_2,\dots,n_i)\subset B(u,r)\subset f^{-1}(U_1)$, and thus
$f(C\bigcap E(n_1,n_2,\dots,n_i))\subset U_1$. Similarly there exists a 
$j\in \mathcal{P}$ such that
$f(D\bigcap E(m_1,m_2,\dots,m_j))\subset U_2$. Let $k=max(i,j)$, we conclude
that $f(C\bigcap E(n_1,n_2,\dots,n_k))$ and $f(D\bigcap E(m_1,m_2,\dots,m_k))$
are Borel separated, which is a contraditon. Hence $f(C)$ and $f(D)$ are Borel 
separated in Y. 
\end{proof}
%%%%%%%%%%%%%%%%%%%

%%%%%%%%%%%%%%%%%%%%%%%%%%%%%%%%%%%%%
\begin{corollary} \label{C:sep_souslin}
\footnote{Federer, 2.2.10, p.67; cf. Bruckner et.al., Real Analysis, 2ed., 11.15, p.444}
Assuming that X is a complete, separable metric space, Y is a Hausdorff space
and $f:X\to Y$ is continuous, 
if $S_1,S_2,\dots,$ are Souslin subsets of X such that $f(S_1),f(S_2),\dots$ are
disjoint, then there exist disjoint Borel sets $B_1,B_2,\dots$ of Y with 
$f(S_j)\subset B_j$ for $j=1,2,\dots$.
\end{corollary}
%%%%%%%%%%%%%%%%%%%
\begin{proof}
For any $j\in\mathcal{P}$, $S_j$ is Souslin, by definition, there exists a
closed set $C_j$ of $X\times\mathcal{N}$ such that $S_j=p(C_j)$, where p is the
projection function. It is easy to see that function 
$f\circ p:X\times \mathcal{N}\to Y$ is continuous, and
$f(S_j)=f(p(C_j))=f\circ p (C_j)$. Now by Lemma \ref{L:sep_lusin}, for any 
unequal $i,j\in\mathcal{P}$, there exist disjoint Borel sets $B_{i,j}$ and
$B_{j,i}$ such that $f(S_i)\subset B_{i,j}$ and $f(S_j)\subset B_{j,i}$.
Now let $B_i=\bigcap_{k\in\mathcal{P}} B_{i,k}$, then $f(S_i)\subset B_i$, and
$B_i\bigcap B_j=\emptyset$ for any $i\neq j$.
\end{proof}
%%%%%%%%%%%%%%%%%%%


%%%%%%%%%%%%%%%%%%%%%%%%%%%%%%%%%%%%%
\begin{lemma}
\footnote{Federer, 2.2.10, p.67;
  cf. Bruckner et.al., Real Analysis, 2ed., Theorem 11.16, pp.444-445; 
  Kechris, Classical Descriptive Set Theory, Theorem 15.1, p.89;
  Tserunyan, Introduction to Descriptive Set Theory (version Sep. 12, 2022), 
  Theorem 13.3, pp.51-52}
Assuming that X is a complete, separable metric space, Y is a Hausdorff space
and $f:X\to Y$ is continuous, if C is a closed subset of X and $f|C$ is 
univalent, then $f(C)$ is a Borel subset of Y.
\end{lemma}
%%%%%%%%%%%%%%%%%%%
\begin{proof}
Again we use the family of closed sets as defined in 
Theorem \ref{T:lipschitz_compltSep} with 
$X=\bigcup_{j\in\mathcal{P}} E(j)$, and
\[
	E(s_1,\dots,s_k)=\bigcup_{j\in\mathcal{P}} E(s_1,\dots,s_{k-1},j),
\]
with $\diam(E(s_1,\dots,s_k)<2^{-k-2}$.
We turn this into a Lusin scheme of Borel sets by letting 
$A(s_1)=E(s_1)\setminus \bigcup_{j<s_1} E(j)$, and
\[
	A(s_1,\dots,s_k) = A(s_1,\dots,s_{k-1}) \bigcap E(s_1,\dots,s_{k-1})
	  \setminus \bigcup_{j<n_k} E(s_1,\dots,s_{k-1},j),
\]

Since f is one-one, the sets $f(C\cap A(s))$ are disjoint for distinct 
$s=(s_1,\dots,s_k)\in\mathcal{P}^k$. Consequently, using Corollary
\ref{C:sep_souslin}, there exist disjoint Borel sets $B'(s)$ that
$f(C\cap A(s))\subset B'(s)$. Now let
\begin{align*}
  B(\emptyset) &= B'(\emptyset), \\
  B(s_1) &= B'(s_1) \bigcap \overline{f(C\cap E(s_1))}, \\
  B(s) &= B'(s) \bigcap B(s_1,\dots,s_{k-1}) \bigcap \overline{f(C\cap E(s))},
\end{align*}
it is easy to see that $B(s)$ are disjoint and 
\[
  f(C\cap A(s))\subset B(s) \subset B(s_1,\dots,s_{k-1}) 
    \bigcap \overline{f(C\cap E(s))}
\]

Let 
\[
	T=\bigcup_{n\in\mathcal{N}} \bigcap_{k\in\mathcal{P}} B(n|k),
\]
since $\{B_s\}$ is a Lusin scheme on $Y$, by Lemma \ref{L:lusin} we have
\[
	T= \bigcap_{k\in\mathcal{P}} \bigcup_{s\in\mathcal{P}^k} B(s),
\]
thus $T$ is a Borel set in $Y$.

It remains only to see that $T=f(C)$. Note first that since $\{A_s\}$ is a 
Lusin scheme we have
\[
  \bigcup_{n\in\mathcal{N}} \bigcap_{k\in\mathcal{P}} A(n|k)
  = \bigcap_{k\in\mathcal{P}} \bigcup_{s\in\mathcal{P}^k} A(s)
  = X,
\]
hence
\[
  f(C)\subset 
    \bigcup_{n\in\mathcal{N}} \bigcap_{k\in\mathcal{P}} f(C\cap A(n|k)) 
	\subset \bigcup_{n\in\mathcal{N}} \bigcap_{k\in\mathcal{P}} B(n|k)
	= T.
\]
In the opposite direction, we check that
\[
  T\subset 
    \bigcup_{n\in\mathcal{N}} \bigcap_{k\in\mathcal{P}} 
	  \overline{ f(C\cap E(n|k)) }
	\subset f(C).
\]
To see this let 
$y\in\bigcup_{n\in\mathcal{N}} \bigcap_{k\in\mathcal{P}}
\overline{f(C\cap E(n|k))}$,
then there exists a $n\in\mathcal{N}$ such that for any $k\in\mathcal{P}$, 
$y\in\overline{f(C\cap E(n|k))}$, which means there exists a 
$x_k\in C\cap E(n|k)$ such that $d(f(x_k),y)<1/k$.
Now $\{x_k\}$ is a Cauchy sequence in C so $x_k\to x$ for some point $x\in C$.
By the continuity of $f$, then $f(x_k)\to f(x)$ so $y=f(x)\in f(C)$. 
\end{proof}
%%%%%%%%%%%%%%%%%%%

%%%%%%%%%%%%%%%%%%%%%%%%%%%%%%%%%%%%%%%%%%%%%%%%%%%%%%%
%%%  existence of Souslin subset that is not Borel  %%%
%%%%%%%%%%%%%%%%%%%%%%%%%%%%%%%%%%%%%%%%%%%%%%%%%%%%%%%
Here we construct a Suslin set which is not a Borel set.

%%%%%%%%%%%%%%%%%%%%%%%%%%%%%%%%%%%%
%%%  diagonal map is continuous  %%%
%%%%%%%%%%%%%%%%%%%%%%%%%%%%%%%%%%%%
\begin{lemma} \label{L:diag_map}
  For any topological space $(X,\mathcal{O}$, the diagonal map 
  $f:X\to X\times X$, defined by $f(x)=(x,x)$, is continuous.
\end{lemma}
%%%%%%%%%%%%%%%%%%%
\begin{proof}
  By Proposition \ref{P:prod_base}, $U_1\times U_2$, where $U_1$ and $U_2$
  are open sets of $X$, is a base of the product space $X\times X$. Hence
  any open set $V\in X\times X$ can be written as
  \[
    V = \bigcup_{U_1,U_2\in \mathcal{O}} U_1\times U_2.
  \]
  Now
  \[
    f^{-1}(V) = f^{-1} \big( \bigcup_{U_1,U_2\in \mathcal{O}} U_1\times U_2 \big)
      = \bigcup_{U_1,U_2\in \mathcal{O}} f^{-1}(U_1\times U_2)
      = \bigcup_{U_1,U_2\in \mathcal{O}} U_1\bigcap U_2,
  \]
  hence $f^{-1}(V)$ is an open set of $(X,\mathcal{O})$, thus $f$ is continuous.
\end{proof}

%%%%%%%%%%%%%%%%%%%
\begin{theorem}
  The space $\mathcal{N}$ contains a Suslin subset that is not Borel.
  \footnote{Federer, 2.2.11, p.68;
    cf. Bruckner et.al., Real Analysis, 2ed., Theorem 11.17, pp.446-448}
\end{theorem}
%%%%%%%%%%%%%%%%%%%
\begin{proof}
First we observe that if $Z$ is a topological space with a countable base 
$U(1),U(2),U(3),\dots$, then the set
\[
  C=\{(z,n)\in Z\times\mathcal{N}: z\notin \bigcup_{i\in\mathcal{P}} U(n_i) \}
\]
is a close set in $Z\times\mathcal{N}$, and every closed subset of $Z$ occurs
among the slices
\[
  C_n = \{ z\in Z: (z,n)\in C \}
      = \{ z\in Z: z\notin \bigcup_{i\in\mathcal{P}} U(n_i) \}.
\]

It is easy to see that for any closed set $A\subset Y$, there exists
$n\in \mathcal{N}$ such that $A=C_n$.
Next we verify that $(Z\times\mathcal{N})\setminus C$ is open.
Note that for any $(z,n)\in (Z\times\mathcal{N})\setminus C$, we have
$z\in\bigcup_{i\in\mathcal{P}} U(n_i)$, hence there exists $j\in\mathcal{P}$
such that $z\in U(n_j)$.
Let
\[
  A_j=\{m\in\mathcal{N}: m_j=n_j\},
\]
using the metric defined in Eq. \ref{E:dist_seq}, 
\[
	d_{\mathcal{N}}(m,n) 
	  = \sum_{i=1}^{\infty} 2^{-i} \frac{|m_i-n_i|}{1+|m_i-n_i|}.
\]
it is easy to see that for any $m\in A_j$, there is an open ball
$B(m,\frac{1}{2^{j+1}})\subset A_j$, hence $A_j$ is an open set in 
$\mathcal{N}$. Thus $U(n_j)\times A_j$ is open in $Z\times\mathcal{N}$. 
Since it is also a subset of $(Z\times\mathcal{N})\setminus C$, we 
conclude that $(Z\times\mathcal{N})\setminus C$ is open in 
$Z\times\mathcal{N}$.

Taking $Z=Y\times\mathcal{N}$, we rewrite
\[
  C=\{(y,m,n)\in Y\times\mathcal{N}\times\mathcal{N}:
    (y,m)\notin \bigcup_{i\in\mathcal{P}} U(n_i) \},
\]
and
\[
  C_n = \{ (y,m)\in Y\times\mathcal{N}: (y,m,n)\in C \}
\]
Letting
\[
  S = \{(y,n)\in Y\times\mathcal{N}: \exists m\in\mathcal{N} ((y,m,n)\in C) \},
\]
and for any $n\in\mathcal{N}$
\[
  S_n = \{ y\in Y: ((y,n)\in S) \}
      = \{ y\in Y: \exists m\in\mathcal{N} ((y,m)\in C_n) \}.
\]
It is easy to see that $S$ is a Suslin subset of  $Y\times\mathcal{N}$
since $S=p_{1,3}(C)$. And since $S_n=p_Y(C_n)$,
we conclude that every Suslin subset of $Y$ occurs among the slices $S_n$, 

Letting $Y=\mathcal{N}$ and consider set
\[
  T = \{ n\in\mathcal{N}: (n,n)\in S \}
    = \{ n\in\mathcal{N}: n\in S_n \}.
\]
By Lemma \ref{L:diag_map}, the diagonal map 
$f:\mathcal{N}\to\mathcal{N}\times\mathcal{N}$, defined by $f(n)=(n,n)$,
is continuous. Then by Corollary \ref{C:suslin_cont}, we get that
$T=f^{-1}(S)$ is a Suslin subset of $\mathcal{N}$.
Moreover its complement
\[
  \mathcal{N}\setminus T = \{ n\in\mathcal{N}: n\notin S_n \}
\]
is not a Suslin subset of $\mathcal{N}$, because if it were, then
there exists some $m\in \mathcal{N}$ such that 
$\mathcal{N}\setminus T=S_m$, and this would cause a contradiction
if we consider whether $m$ is in $T$ or $\mathcal{N}\setminus T$.
\footnote{cf. Cohn, Measure Theory, 2nd ed., Corollary 8.2.17, p.254}
Now, suppose $m\in T$, then we have $m\in S_m=\mathcal{N}\setminus T$,
which is impossible; on the other hand, suppose
$m\in \mathcal{N}\setminus T$, then we have $m\in S_m$, and hence
$m\in T$, again impossible. Therefore $T$ is not a Borel set.
\end{proof}






%%%%%%%%%%%%%%%%%%%%%%%%%%%%%%%%%%%%%
\newpage
List of important results:
\begin{enumerate}
  \item Every closed subset of a metrizable space is a $G_{\delta}$ set.
        Every open subset of a metrizable space is a $F_{\sigma}$ set.
    (Corollary \ref{C:g_delta}).
  \item Countable product of separable spaces is separable (Proposition
	\ref{P:prod_sep}) .
  \item	Countable product of complete metric spaces is complete (Theorem 
	\ref{T:prod_complt}).
  \item Every subspace of a separable metric space is separable (Proposition
	\ref{P:metric_sep}).
  \item Every closed subset of a complete space is complete (Proposition
	\ref{P:closed_complete}).
  \item Every metric space $(X,d)$ is a $T_1$-space (Corollary
        \ref{C:metric_T1}).
  \item Every metric space $(X,d)$ is a $T_2$-space (Hausdorff space) (Corollary
        \ref{C:metric_T2}).
\end{enumerate}



